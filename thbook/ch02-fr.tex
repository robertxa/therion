\eject
\vbox{\rightline{\pic[100mm]{agrippa.jpg}}
\kern-20mm
\eightit\baselineskip10pt\rightskip105mm\leftskip0mm\parindent1.05cm
Nous n'avons \'ecrit ce travail que pour vous, enfants de doctrine et d'ap-prentissage. 
Examinez ce livre, r\'efl\'echis-sez au sens que nous avons dispers\'e et rassembl\'e \`a nouveau ; 
ce que nous avons cach\'e \`a un endroit, nous l'avons r\'ev\'el\'e \`a un autre, 
afin que cela puisse \^etre compris par votre sagesse.\par
\kern-10pt
\rightline{\hfil\eightrm ---Henricus C. agrippa ab Nettesheym, \eightmit 1533}
}

\chapter Cr\'eation de fichiers de donn\'ees.

\subchapter Les Bases.

Les fichiers d'entr\'ee pour Therion sont au format texte. 
Il y a quelques r\`egles de base sur la forme d'un tel fichier :

\list 
* Il existe deux types de commandes : 
  Commandes sur une ligne et commandes sur plusieurs lignes.  

* Une commande sur une seule ligne est termin\'ee par un caract\`ere de fin de ligne. 
   La syntaxe de celle-ci est :

  |command arg1 ... argN [-option1 value1 -option2 value2 ...]|

  o\`u {\it arg1 ... argN} sont des arguments obligatoires et les paires commen\c{c}ant par un tiret 
  {\it -option value} sont des options facultatives. 
  Les arguments et les options disponibles d\'ependent de la commande. 
  Un exemple pourrait \^etre :

  |point 643.5 505.0 gradient -orientation 144.7|

   avec trois arguments obligatoires (au d\'ebut) et une paire optionnelle `option / valeur' (apr\`es le tiret). 
   Les options peuvent n'avoir aucune ou plusieurs valeurs. 
  
* Les commandes multi-lignes commencent de la m\^eme mani\`ere que les commandes \`a une seule ligne, 
   mais continuent sur les lignes suivantes jusqu'\`a la fin de la commande explicite (endcommande). 
   Ces lignes peuvent contenir des donn\'ees ou des options, qui sont appliqu\'ees aux donn\'ees suivantes. 
   Si une ligne de donn\'ees commence par un mot r\'eserv\'e \`a une option, vous devez ins\'erer `|!|' Devant celle-ci. La syntaxe est :

  |command arg1 ... argN [-option1 value1 -option2 value2 ...]
  ...
  optionX valueX
  data
  ...
endcommand|

Encore une fois, pour une meilleure compr\'ehension, voici un exemple r\'eel avec la commande ``|line|'' (ligne) :

|line wall -id walltobereferenced
  1174.0 744.5
  1194.0 756.5 1192.5 757.5 1176.0 791.0
  smooth off
  1205.5 788.0 1195.5 832.5 1173.5 879.0
endline|

   Cette commande line a un argument obligatoire, un type de ligne (paroi de cavit\'e  dans ce cas), suivi d'une option. 
   Les deux lignes suivantes contiennent des donn\'ees (coordonn\'ees des courbes de B\'ezier \`a tracer).
   La ligne suivante (``|smooth off|'' pour lisser) sp\'ecifie une option qui s'applique aux donn\'ees suivantes 
   (c'est-\`a-dire pas pour la ligne enti\`ere, contrairement \`a l'option |-id| dans la premi\`ere ligne) 
   et la derni\`ere ligne contient quelques donn\'ees suppl\'ementaires.

* si la valeur d'une option ou d'un argument contient des espaces, vous devez inclure cette valeur entre \hbox{|" "|} ou \hbox{|[ ]|}. 
  Si vous souhaitez ins\'erer un double guillemet |"| texte inclus \hbox{|" "|}, vous devez l'ins\'erer deux fois. 
  Les guillemets sont utilis\'es pour les cha\^ines, les crochets pour les valeurs num\'eriques et les mots-cl\'es (keywords).

* chaque ligne se terminant par une barre oblique invers\'ee ou anti-slash (|\|) est consid\'er\'ee comme continuant sur la ligne suivante, 
   comme s'il n'y avait ni saut de ligne ni barre oblique invers\'ee.
   
* tout ce qui suit |#| (di\`ese), jusqu'\`a la fin de la ligne, m\^eme \`a l'int\'erieur d'une commande, est consid\'er\'e comme un commentaire et est donc ignor\'e.

* \NEW{5.4}les commentaires multilignes dans un bloc |comment| ... |endcomment| sont autoris\'es dans les fichiers de donn\'ees et de configuration
\endlist


\subchapter Types de donn\'ees.

Therion utilise les types de donn\'ees suivants :

\list
* {\it keyword/mot-clef} = une s\'equence de caract\`eres  |a-Z|, |a-z|, |0-9| et |_-/|
             (ne commen\c{c}ant pas par `|-|').
* {\it ext\_keyword/mot-clef\_\'etendu} = mot-cl\'e pouvant aussi contenir des caract\`eres  |+*.,'|, 
            Mais pas sur la premi\`ere position.
* {\it date} = une date (ou un d'intervalle de temps) dans un format sp\'ecifique : \hfil\break
          |YYYY.MM.DD@HH:MM:SS.SS - YYYY.MM.DD@HH:MM:SS.SS| ou `|-|'
          pour laisser une date ind\'etermin\'ee.
* {\it person/personne} =nom et pr\'enom d'une personne s\'epar\'es par des espaces. 
          Utilisez `|/|' ' pour s\'eparer le pr\'enom et le nom s'il y a plusieurs noms.
* {\it string/cha\^ine de charact\`ere} = une s\'equence de caract\`eres quelconques. 
    \NEW{5.3}Les cha\^ines peuvent contenir une balise sp\'eciale |<lang:XX>| pour s\'eparer les traductions. 
    Dans les cha\^ines multilingues, seul le texte compris entre |<lang:XX>|
    (o\`u |XX| est la langue s\'electionn\'ee dans le fichier d'initialisation ou de configuration) 
    et la balise |<lang:YY>| suivante est affich\'e \`a la sortie. 
    Si aucune correspondance n'est trouv\'ee, tout ce qui pr\'ec\`ede toute balise
    |<lang:ZZ>|  est affich\'e.
* {\it units} = unit\'es de longueur prises en charge :
           meter[s] (m\`etres), centimeter[s] (centim\`etres), inch[es] (pouces), feet[s] (pieds), yard[s]
           (aussi m, cm, in, ft, yd).
           Unit\'es d'angle prises en charge : degree[s], minute[s] (aussi deg, min), 
	   grad[s], mil[s], percent[age] (uniquement pour le clinom\`etre). 
           Une valeur en degr\'es peut \^etre entr\'ee en notation d\'ecimale
	   ($x.y$) ou en notation sp\'eciale pour les degr\'es, minutes et secondes
	   ($deg[{:}min[{:}sec]]$).
\endlist



\subchapter Syst\`emes de coordonn\'ees.

Therion prend en charge les transformations de coordonn\'ees dans les syst\`emes de coordonn\'ees g\'eod\'esiques. 
Vous pouvez sp\'ecifier l'option |cs| (CS pour |C|oordinate |S|ystem) dans les objets  |centreline|, |surface|, |import| and 
|layout| (ligne de cheminement, surface, importation et mise en page), puis saisir des donn\'ees XY dans un syst\`eme donn\'e. 
Vous pouvez \'egalement sp\'ecifier la sortie |cs| dans le fichier de configuration. 

Si vous ne sp\'ecifiez pas de |cs| dans votre jeu de donn\'ees, 
il est suppos\'e que vous travaillez dans un syst\`eme de coordonn\'ees local et aucune conversion n'est effectu\'ee. 
Si vous sp\'ecifiez |cs| n'importe o\`u dans les donn\'ees, vous devez le sp\'ecifier pour toutes les donn\'ees de g\'eolocalisation 
(|fix|, |origin| in |layout| etc.). 

|cs| s'applique \`a toutes les donn\'ees d'emplacement suivantes jusqu'\`a ce qu'un autre |cs| soit sp\'ecifi\'e ou sinon jusqu'\`a la fin de l'objet en cours, 
selon celui qui est indiqu\'e en premier.

Les syst\`emes de coordonn\'ees suivants sont pris en charge :

\list
* |UTM1| -- |UTM60| = Universal Transverse Mercator pour l'h\'emisph\`ere nord et dans une zone donn\'ee, WGS84 datum.
* |UTM1N| -- |UTM60N| = idem \`a |UTM1| -- |UTM60|
* |UTM1S| -- |UTM60S| = UTM pour l'h\'emisph\`ere sud, WGS84 datum.
* |lat-long|, |long-lat| = latitude (N positive, S n\'egative) et 
  longitude (E positive, W n\'egative) dans un ordre donn\'e en degr\'es
  ($deg[{:}min[{:}sec]]$ authoris\'e), WGS84 datum. Par d\'efaut, non pris en charge en sortie.
* |EPSG:<nombre>| = La plupart des syst\`emes de coordonn\'ees EPSG. 
  Presque tous les syst\`emes de coordonn\'ees utilis\'es dans le monde ont leur propre num\'ero EPSG. 
  Pour trouver le num\'ero de votre syst\`eme, voir le fichier |extern/proj4/nad/epsg| dans la distribution de la source Therion.
* |ESRI:<nombre>| = Similaire \`a EPSG, mais standard ESRI.
%* |EUR79Z30| = UTM zone 30, EUR79 datum.  % ad-hoc addition not worth mentioning??
* |JTSK|, |iJTSK| = Syst\`eme tch\'ecoslovaque S-JTSK utilis\'e depuis les ann\'ees 1920 avec les axes sud et ouest (JTSK) 
  et sa version modifi\'ee avec les axes orient\'es est et nord et les nombres n\'egatifs (iJTSK). 
  JTSK n'est pas pris en charge en sortie (mais iJTSK lui l'est).
* |JTSK03|, |iJTSK03| = nouvelle r\'ealisation S-JTSK introduite en Slovaquie en 2011.
* |OSGB:<H, N, O, S or T><a-Z except I>| =Grille nationale British Ordnance Survey.\NEW{5.4}
* |S-MERC| = la projection sph\'erique de Mercator, utilis\'ee par divers sites de cartographie en ligne.
\endlist

\subchapter D\'eclinaison magn\'etique, automatique ou manuelle.

Therion int\`egre le mod\`ele de champ g\'eomagn\'etique Terre IGRF\[Voir \www{https://www.ngdc.noaa.gov/IaGa/vmod/}], 
valable pour la p\'eriode 1900-2020\NEW{5.4}. Il est utilis\'e automatiquement pour le calcul de la d\'eclinaison magn\'etique 
si la cavit\'e est situ\'ee dans l'espace avec une station fix (g\'eolocalis\'ee) utilisant l'un des syst\`emes de coordonn\'ees g\'eod\'esiques 
pris en charge en m\^eme temps que la commande date des lignes de cheminement. 
La d\'eclinaison calcul\'ee est alors r\'epertori\'ee dans le fichier journal pour plus d'informations (LOG file).


Si l'utilisateur d\'efinit une |declination| (d\'eclinaison magn\'etique) sp\'ecifique pour le cheminement, cette valeur a priorit\'e sur le calcul automatique.

\subchapter Format des donn\'ees.

La syntaxe des fichiers d'entr\'ee est expliqu\'ee dans la description des commandes individuelles. 
L'\'etude des exemples de fichiers distribu\'es avec Therion vous aidera \`a comprendre. 
Voir aussi un exemple en {\it annexe}.

Chacune des sections suivantes d\'ecrit une commande Therion \`a l'aide de la structure suivante :
The syntax of input files is explained in the description of

{\it Description :} notes concernant cette commande.

{\it Syntaxe :} description de la syntaxe sch\'ematique.

{\it Contexte :} : sp\'ecifie le contexte dans lequel cette commande est autoris\'ee. 
Le contexte de relev\'e topo ({\it survey}) signifie que la commande doit \^etre entour\'ee par la paire
|survey ... endsurvey| pair. Le contexte de  {\it scrap} signifie que la commande doit \^etre incluse dans la paire 
|scrap ... endscrap|. Context {\it all} signifie que la commande peut \^etre utilis\'ee n'importe o\`u.

{\it Arguments :} une liste des arguments obligatoires avec leurs explications.

{\it Options :} une liste des options utilisables.

{\it Command-like options :} options pour les commandes multi-lignes, qui peuvent \^etre sp\'ecifi\'ees parmi les lignes de donn\'ees.

%%%%%%%%%%%%%%%%%%%%%%%%%%%%%%%%%%%%%%%%


\subsubchapter `encoding'.

\description
  D\'efinit l'encodage du fichier d'entr\'ee. Cela permet d'utiliser d'autres caract\`eres que les caract\`eres ASCII dans les fichiers d'entr\'ee.
\enddescription

\syntax
  |encoding <nom-encodage>|
\endsyntax

\context
  Ce devrait \^etre la toute premi\`ere commande du fichier.
\endcontext

\arguments
* |<encoding-name>| = pour voir une liste de tous les noms de codage pris en charge, ex\'ecutez Therion avec l'option 
  |--print-encodings|. `UTF-8' (Unicode) et `aSCII' (7\,bit) sont toujours support\'es.
\endarguments

%%%%%%%%%%%%%%%%%%%%%%%%%%%%%%%%%%%%%%%%


\subsubchapter `input'.

\description
 Ins\`ere le contenu d'un fichier \`a la place de la commande. 
 L'extension par d\'efaut est `.th' et peut \^etre omise. 
 Pour une compatibilit\'e optimale, utilisez les chemins relatifs et les barres Unix `|/|', 
 et non les barres obliques inverses Windows `|\|', comme s\'eparateurs de r\'epertoires.
\enddescription

\syntax
  |input <file-name>|
\endsyntax

\context
  all
\endcontext

\arguments
*  |<nom-du-fichier>|
\endarguments

%%%%%%%%%%%%%%%%%%%%%%%%%%%%%%%%%%%%%%%%


\subsubchapter `survey'.

\description
 Survey est la structure de donn\'ees principale. 
 Les topographies peuvent \^etre imbriqu\'ees, ce qui permet de construire une structure hi\'erarchique. 
 Habituellement, chaque niveau de cette \'etude de structure hi\'erarchique repr\'esente des cavit\'es, 
 des zones karstiques de niveaux sup\'erieurs et de niveaux inf\'erieurs, par exemple des galeries.
 % Each data object must belong to a   survey. 

  Chaque topographie a son propre espace de noms sp\'ecifi\'e par son argument  |<id>|. Les objets 
  (tels que les stations topos ou les scraps ; voir ci-dessous 
  qui appartiennent \`a une sous-topographie (subsurvey) de la topographie en cours sont r\'ef\'erenc\'es comme suit :
  
  |<object-id>@<subsurvey-id>|,
  
  ou, s'il y a plusieurs niveaux de classement :
  
  |<object-id>@<subsubsurvey-id>.<subsurvey-id>|.\[Note : il est impossible de faire r\'ef\'erence 
  \`a un objet rang\'e parmi les topographies de niveau sup\'erieur.]
  
  Cela signifie que les identificateurs d'objet ne doivent \^etre uniques que dans le cadre d'une seule et m\^eme topographie. 
  En cons\'equence, les noms des stations topographiques peuvent \^etre identiques si elles se trouvent dans des topographies diff\'erentes. 
  Cela permet aux stations d'\^etre num\'erot\'ees \`a partir de 0 dans chaque relev\'e ou d'assembler deux cavit\'es 
  dans un m\^eme syst\`eme karstique sans renommer les stations du relev\'e.
\enddescription

\syntax
      |survey <id> [OPTIONS]
       ... other therion objects ...
       endsurvey [<id>]|
\endsyntax

\context
  aucun, survey
\endcontext

\arguments
*|<id>| = identifiant de la survey/topographie
\endarguments

\options 
* |namespace <on/off>| = sp\'ecifie si survey cr\'ee un espace pour les noms (|on| 
  par d\'efault)
* |declination <specification>| = d\'efinit la d\'eclinaison magn\'etique qui sera utilis\'ee par d\'efaut pour toutes les donn\'ees des objets de cette topographie 
  (qui peut \^etre remplac\'ee par les d\'efinitions de d\'eclinaison dans les sous-niveaux). La |<specification>| 
  peut prendre trois formes :

  1. |[]| une cha\^ine vide. Cela r\'einitialisera la d\'efinition de la d\'eclinaison magn\'etique.

  2. |[<value> <units>]| d\'efinira une valeur unique de d\'eclinaison magn\'etique (aussi pour les topographies non dat\'ees).

  3. |[<date1> <value1> [<date2> <value2> ... ] <units>]| 
     d\'efinira la d\'eclinaison magn\'etique pour plusieurs dates diff\'erentes. 
     Ensuite, la d\'eclinaison de chaque vis\'ee sera d\'efinie en fonction de la date sp\'ecifi\'ee par les donn\'ees de l'objet. 
     Si vous souhaitez d\'efinir explicitement la d\'eclinaison pour les donn\'ees de topographies non dat\'ees, utilisez `|-|' au lieu de la date.

  Si aucune d\'eclinaison magn\'etique n'est sp\'ecifi\'ee mais qu'un syst\`eme de coordonn\'ees g\'eod\'esique est d\'efini, 
  la d\'eclinaison est automatiquement calcul\'ee \`a l'aide du mod\`ele g\'eomagn\'etique int\'egr\'e.
    
  {\bf N.B. :} La d\'eclinaison magn\'etique est positive lorsque le nord magn\'etique est \`a l'Est du nord vrai.

* |person-rename <ancien nom> <nouveau nom>| = renomme une personne dont le nom a \'et\'e chang\'e.

* |title <string/blabla>| = description de l'objet.

* |entrance <nom-de-station>| = sp\'ecifie l'entr\'ee principale de la cavit\'e repr\'esent\'ee par cette topographie. 
   Si elle n'est pas sp\'ecifi\'e et qu'il y a au moins une station marqu\'ee dans cette topo, celle-ci est \'egalement consid\'er\'ee comme une cavit\'e. 
   Cette information est utilis\'ee pour l'exportation dans |cave-list|.
\endoptions


%%%%%%%%%%%%%%%%%%%%%%%%%%%%%%%%%%%%%%%%


\subsubchapter `centreline'.

\description
  Donn\'ees du relev\'e topographique (ligne de cheminement). 
  La syntaxe est emprunt\'ee \`a Survex avec des modifications mineures. 
  Le manuel Survex peut \^etre utile en tant que r\'ef\'erence suppl\'ementaire pour l'utilisateur. 
  Un terme synonyme \`a `centerline' peut \^etre utilis\'e.
\enddescription

\syntax
      |centreline [OPTIONS]
          date <date>
          team <personne> [<roles>]
          explo-date <date>
          explo-team <personne>
          instrument <quantity list> <description>
          calibrate <quantity list> <zero error> [<scale>]
          units <quantity list> [<facteur>] <unites>
          sd <quantity list> <valeur> <unites>
          grade <grade list>
          declination <valeur> <unites>
          grid-angle <valeur> <unites>
          infer <quoi> <on/off>
          mark <type>
          flags <shot flags>
          station <station> <commentaire> [<flags>]
          cs <systeme de coordonnees>
          fix <station> [<x> <y> <z> [<std x> <std y> <std z>]]
          equate <liste de stations>
          data <style> <ordre de lecture>
          break
          group
          endgroup
          walls <auto/on/off>
          vthreshold <numbre> <unites>
          extend <spec> [<station> [<station>]]
          station-names <prefix> <suffix>
          ...
          [SURVEY DaTa]
          ...
        endcentreline|
\endsyntax

\context
  aucun, survey
\endcontext

\options
  * |id <ext_keyword>| = id de l'object
  * |author <date> <personne>| = auteur des donn\'ees et date de cr\'eation
  * |copyright <date> <string>| = date du copyright date et d\'esignation du copyright
  * |title <string>| = description de l'object
\endoptions


\comopt
  * |date <date>| = Date de la topographie (des mesures). 
                              Si plusieurs dates sont sp\'ecifi\'ees, un intervalle de temps est cr\'e\'e
  * |explo-date <date>| = Date d'exploration. Si plusieurs dates sont sp\'ecifi\'ees,
                                        un intervalle de temps est cr\'e\'e.
  * |team <personne> [<roles>]| = un membre de l'\'equipe de topographie. 
                                                     Le premier argument est son nom, les autres d\'ecrivent les r�les de la personne dans l'\'equipe 
                                                     (facultatif - non utilis\'e actuellement). 
                                                     Les mots-cl\'es de r�le pris en charge sont les suivants : 
                                                     station, length, tape, [back]compass, [back]bearing, [back]clino,
                                                     [back]gradient, counter, depth, station, position, notes, pictures, pics, instruments (insts), assistant (dog).
  * |explo-team <personne>| = un membre de l'\'equipe d'exploration. 
  * |instrument <quantity list> <description>| = description de l'instrument utilis\'e obtenir les donn\'ees du relev\'e topographique (m\^emes mots cl\'es que pour le r�le de l'\'equipier)
  * |infer <quoi> <on/off>| = `|infer plumbs on|' indique au programme d'interpr\'eter les gradients $\pm90\,^\circ$ comme HAUT / BAS 
                                             (cela signifie qu'aucune correction n'est appliqu\'ee au clinom\`etre). `|infer equates on|' indique au  programme 
                                             d'interpr\'eter au cas par cas les vis\'ees avec une longueur \'egale \`a 0 en tant que commandes \'equivalentes 
                                             (ce qui signifie qu'aucune correction n'est appliqu\'ee au d\'ecam\`etre) 
  * |declination <valeur> <units>| = d\'efinit la d\'eclinaison magn\'etique pour les vis\'ees suivantes $$true\ bearing = measured\ bearing + declination.$$
                                                       La d\'eclinaison est positive lorsque le nord magn\'etique est \`a l'est du nord vrai. 
                                                       Si aucune d\'eclinaison n'est sp\'ecifi\'ee ou si la d\'eclinaison est r\'einitialis\'ee (|-|), 
                                                       une valeur de d\'eclinaison valide est recherch\'ee dans toutes les topographies dans lesquelles 
                                                       se trouve les donn\'ees de cet objet. Voir les options de d\'eclinaison magn\'etique de la commande |survey|.
  * |grid-angle <valeur> <units>| = sp\'ecifie l'angle de la grille magn\'etique (d\'eclinaison par rapport au nord de la grille).
  * |sd <quantity list> <valeur> <units>| = (sd mis pour � d\'eviation standard �) d\'efinit l'\'ecart-type pour les mesures donn\'ees. 
                                                                 La liste quantit\'e peut contenir les mots-cl\'es suivants : 
                                                                 length, tape, bearing, compass,  gradient, clino, counter, depth, x, y, z, position, easting, dx,
                                                                 northing, dy, altitude, dz.
                                                                 
                                                                 Pour utiliser correctement cette commande, vous devez comprendre ce qu'est un \'ecart-type (ou � d\'eviation standard � en anglais). 
                                                                 Il attribue  une valeur \`a la "propagation" des erreurs dans une mesure. 
                                                                 En supposant que celles-ci soient normalement distribu\'ees (courbe de Gauss), nous pouvons affirmer que 95,44 \% 
                                                                 des longueurs r\'eelles se situeront dans les limites de deux \'ecarts-types (2 $\sigma$) de la longueur mesur\'ee. 
                                                                 Prenons l'exemple des mesures de longueur : un \'ecart-type de 0,25 m signifie que la longueur r\'eelle de ces mesures 
                                                                 se situe dans la limite de $\pm$ 0,5 m de leur valeur enregistr\'ee par l'instrument dans 95,44 \% des cas. 
                                                                 Donc, si la mesure est de 7,34 m, la longueur r\'eelle sera tr\`es probablement comprise entre 6,84 m et 7,84 m. 
                                                                 NB : cet exemple correspond \`a la ``classe 3'' de la BCRA (British Cave Research Association).
                                                         
  * |grade <grade list>| = d\'efinit les \'ecarts-types en fonction de la classification sp\'ecifique des topographies \'etablie par la BCRA 
                                       (British Cave Research Association ; voir d\'etails plus loin pour la commande grade). 
                                       Tous les \'ecarts-type (sd) ou degr\'es (grade) sp\'ecifi\'es pr\'ec\'edemment sont perdus. 
                                       Si vous souhaitez changer de SD, utilisez l'option |sd| apr\`es cette commande. 
                                       Si plusieurs degr\'es de pr\'ecision diff\'erents sont sp\'ecifi\'ees, seul le dernier s'applique.
                                       Vous pouvez sp\'ecifier des \'ecarts-types pour toute la topographie ou uniquement pour une position donn\'ee. 
                                       Si vous souhaitez combiner les deux, vous devez les utiliser dans une seule ligne grade.
  * |units <quantity list> [<facteur>] <units>| = d\'efinir les unit\'es utilis\'ees pour les mesures donn\'ees (m\^eme contenu de liste quantit\'e que pour sd).
  * |calibrate <quantity list> <zero error> [<echelle>]| = calibrate est utilis\'e pour sp\'ecifier les corrections (\'etalonnage) d'instruments de mesure, 
                                                                                       via une erreur de z\'ero et un facteur d'\'echelle. 
                                                                                       Par d\'efaut, l'erreur z\'ero est de 0.0 et le facteur d'\'echelle de 1.0 pour toutes les quantit\'es mesur\'ees. 
                                                                                       La valeur mesur\'ee sera ensuite calcul\'ee \`a l'aide de la formule suivante : 
                                                                                       $valeur\ mesuree = (valeur\ lue - zero\ error) \times scale$. 
                                                                                       M\^eme contenu de liste quantit\'e que pour sd (voir ci-dessus).
  * |break| = peut \^etre utilis\'e avec des donn\'ees entrelac\'ees pour s\'eparer deux cheminements.
  * |mark [<station list>] <type>| = d\'efinit le type de stations nomm\'ees.  |<type>| > est l'un des \'el\'ements suivants :  
                                                      fixed (fixe), painted (marquage \`a la peinture)  et temporary (temporaire ; valeur par default). 
                                                      S'il n'y a pas de liste de stations, toutes les stations suivantes sont marqu\'ees avec le type choisi.
  * |flags <shot flags>| = d\'efinir des marqueurs pour les vis\'ees suivantes. Les marqueurs accept\'es sont : 
                                      |surface| (pour les mesures de surface), 
                                      |duplicate| (pour des vis\'ees dupliqu\'ees, 
                                      |splay| (pour les vis\'ees lat\'erales d'habillage qui sont masqu\'ees sur  les dessins et les mod\`eles par d\'efaut). 
                                      Ceux-ci sont exclus des calculs de longueur. 
                                      
                                      Toutes les vis\'ees dont l'une des stations est marqu\'ee soit `|.|' soit `|-|' sont d\'efinies 
                                      comme des vis\'ees lat\'erales d'habillage par d\'efaut (voir aussi la commande |data|)
                                      
                                      Si le marqueur est d\'efini sur |approx[imate]|, la vis\'ee est incluse dans les calculs de longueur totale, 
                                      mais est \'egalement affich\'ee s\'epar\'ement dans les statistiques de la topographie. 
                                      Ce marqueur doit \^etre utilis\'e pour les vis\'ees incorrectes ou impr\'ecises et qui doivent \^etre r\'eexamin\'ees.
                                      
                                      De plus, ``|not|'' (non) est autoris\'e avant un marqueur.
    
  * |station <station> <commentaire> [<flags>]| = ajouter le commentaire de la station et ses marqueurs. 
                                                                             Si |""| est indiqu\'e comme commentaire, il est ignor\'e.
                                                                              
                                                                              Les marqueurs accept\'es sont : 
                                                                              |entrance| (entr\'ee), |continuation|, |air-draught|
                                                                              |[:winter|/|summer]| (courant d'air [:hiver/\'et\'e], 
                                                                              |sink| (perte), |spring| (arriv\'ee d'eau), |doline|, |dig| (surcreusement, d\'esobstruction), 
                                                                              \NEW{5.3}|arch| (vo\^ute), |overhang| (surplomb). 
                                                                              De plus, |not| (non) est autoris\'e avant un marqueur, pour supprimer le flag ajout\'e pr\'ec\'edemment. 
                                                                              
                                                                              Vous pouvez \'egalement sp\'ecifier des attributs personnalis\'es pour une station \`a l'aide du flag |attr|, 
                                                                              suivi du nom et de la valeur de l'attribut. 
                                                                              Exemple:\hfil\break |station 4 "puits \'a explorer" continuation attr code "V"|
                                                                              
                                                                              S'il existe un secteur qui a \'et\'e explor\'e, mais pas encore topographi\'e, 
                                                                              la longueur explor\'ee estim\'ee de ce secteur peut \^etre ajout\'ee \`a une station avec le flag |continuation|
                                                                              Ajoutez simplement  |explored <explored-length>| aux marqueurs de la station. 
                                                                              Les longueurs explor\'ees font partie des statistiques topographiques de la cavit\'e et sont affich\'ees s\'epar\'ement.
                                                                              Exemple:\hfil\break |station 40 "ugly crollway" continuation explor\'ee sur 100 m|
    
  * |cs <coordinate system>| = syst\`eme de coordonn\'ees pour les stations ayant des coordonn\'ees fixes.
  * |fix <station> [<x> <y> <z> [<std x> <std y> <std z>]]| = fixe les coordonn\'ees de la station topo (avec des erreurs sp\'ecifiques). 
                                                                                             Seule la conversion d'unit\'es, et non l'\'etalonnage, pourra leur \^etre appliqu\'ee
  * |equate <station list>| = d\'efinit des points topos \'equivalents.
  * |data <style> <readings order>| = d\'efinit le style des donn\'ees 
                                                          (normal, topofil, diving, cartesian, cylpolar---coordonn\'ees polaires cylindriques : 
                                                          dans ce cas les mesures de longueurs sont horizontales et non en suivant la pente---, dimensions, nosurvey---pas de topographie---) 
                                                          et l'ordre de lecture. La zone de l'ordre de lecture utilise l'un des mots-cl\'es suivants : 
                                                          station, from, to, tape/length, [back]compass/[back]bearing, [back]clino/\penalty0[back]gradient, 
                                                          depth, fromdepth, todepth, depthchange, counter, 
                                                          fromcount, tocount, northing, easting, altitude,
                                                          up/ceiling\[La dimension peut \^etre sp\'ecifi\'ee comme une paire 
                                                           |[<from> <to>\char"5D| 
                                                          c'est \`a dire la dimension au d\'ebut et \`a la fin de la vis\'ee.], 
                                                          down/floor, left, right, ignore, ignoreall.
                                                          
                                                          C'est-\`a-dire : station, depuis, vers, d\'ecam\`etre/longueur, compas/azimuth et mesure inverse, 
                                                                               clinom\`etre/pente et mesure inverse, profondeur, profondeur depuis, profondeur vers, 
                                                                               changement de profondeur, compteur (topofil), compteur depuis, compteur vers, 
                                                                               vers le nord, vers l'est, altitude, haut/vo\^ute, bas/sol, gauche, droite, ignorer, tout ignorer.
                                                          
                                                          Voir le manuel du logiciel Survex pour plus de d\'etails.
                                                          
                                                          Pour les donn\'ees entrelac\'ees, les mots-cl\'es de nouvelle ligne et de direction sont pris en charge. 
                                                          Si des mesures avant-arri\`ere (normale / inverse) de compas ou de clinom\`etre sont enregistr\'ees, 
                                                          la moyenne des deux sera calcul\'ee.
                                                          
                                                          \NEW{5.3} Si l'une des stations de vis\'ee porte le nom `|.|' ou `|-|', le marqueur splay (vis\'ee lat\'erale d'habillage) lui est attribu\'e. 
                                                          Le point `|.|' devrait \^etre utilis\'e pour les vis\'ees se terminant sur des \'el\'ements situ\'es \`a l'int\'erieur de la galerie topographi\'ee, 
                                                          le tiret `|-|' pour les vis\'ees se terminant sur les parois, au sol ou au plafond. 
                                                          Bien que Therion ne fasse pas encore de distinction entre eux, il devrait \^etre utilis\'e pour am\'eliorer la mod\'elisation 3D \`a l'avenir.
    
  * |group| 
  * |endgroup| = Cette paire |group/endgroup| permet \`a l'utilisateur d'apporter des modifications temporaires dans presque tous les param\`etres (calibrate, units, sd, data, flags...).
  * |walls <auto/on/off>| = Active / d\'esactive la cr\'eation de la forme de la galerie (parois automatiques) \`a partir des donn\'ees LRUD pour les vis\'ees suivantes. 
                                       Si cette option est d\'efinie sur |auto|, le secteur n'est cr\'e\'e que s'il n'y a pas de scrap r\'ef\'eren\c{c}ant la ligne de cheminement (centerline).
  * |vthreshold <nombre> <units>| = seuil pour interpr\'eter les lectures LRUD comme des lectures gauche-droite-avant-arri\`ere perpendiculaires \`a la vis\'ee (habillage).
                                                        
                                                        Si le secteur est horizontal (|inclination < vthreshold|), LR est perpendiculaire \`a la vis\'ee et UD est vertical.
                                                        
                                                        Si le secteur est plus ou moins vertical (|inclination > vthreshold|), m\^eme UD devient perpendiculaire au plan - sinon, le rendu ne serait pas correct. 
                                                        Dans le cas de vis\'ees verticales, la valeur UD est interpr\'et\'ee comme une dimension nord-sud de la station 
                                                        afin de permettre la mod\'elisation tubulaire des verticales (visualisation des puits)
    
  * |extend <spec> [<station> [<station>]]| = contr�le la direction de la ligne de cheminement. |<spec>| est l'un des mots-cl\'es suivants :
                                                                      
                                                                      |normal/reverse| = \'etendre les stations donn\'ees et suivantes dans la m\^eme direction / ou la direction inverse par rapport \`a la station pr\'ec\'edente. 
                                                                                                    Si deux stations sont indiqu\'ees, la direction est appliqu\'ee uniquement \`a la vis\'ee donn\'ee.
                                                                      
                                                                       |left/right| = comme ci-dessus, mais la direction est sp\'ecifi\'ee explicitement.
                                                                       
                                                                       |vertical| = ne d\'eplacez pas la station (vis\'ee/shot) dans la direction $X$, , utilisez uniquement la composante $Z$ (verticale) de la vis\'ee.
                                                                       
                                                                       |start| = sp\'ecifie la station de d\'epart (vis\'ee).
                                                                       
                                                                       |ignore| = ignore la sation sp\'ecifi\'ee (vis\'ee), continuer et prolonger le cheminement avec une autre station (vis\'ee) si c'est possible.
                                                                       
                                                                       |hide| = masquer (ne pas afficher) la station sp\'ecifi\'ee (vis\'ee) en \'el\'evation prolong\'ee. 
                                                                                   Si aucune station n'est sp\'ecifi\'ee, |<spec>| est valide pour les vis\'ees sp\'ecifi\'ees suivantes.

  * |station-names <prefix> <suffix>| = ajoute un pr\'efixe / suffixe donn\'e \`a toutes les stations topo de la ligne de cheminement en cours (centerline). Enregistre bon nombre de signes typographiques.
    
\endcomopt



%%%%%%%%%%%%%%%%%%%%%%%%%%%%%%%%%%%%%%%%


\subsubchapter `scrap'.

\description
  Un scrap est un morceau de topographie 2D qui ne contient pas de parties qui se superposent 
  (tous les passages peuvent \^etre dessin\'es sur un papier sans se chevaucher). 
  Pour les grottes courtes et simples, la cavit\'e enti\`ere peut appartenir \`a un seul scrap. 
  Dans les syst\`emes complexes, un scrap repr\'esente g\'en\'eralement une salle ou une galerie. 
  Id\'ealement, un scrap contient environ 100 m de cavit\'e\[Si n\'ecessaire, 
  les scraps peuvent \^etre beaucoup plus petits---pour afficher juste quelques m\`etres de la cavit\'e. 
  Lorsque vous d\'ecidez de la taille du scrap, veuillez prendre en compte les \'el\'ements suivants : 
  L'utilisation de petits scraps peut prendre plus de temps au cartographe pour optimiser les jonctions de scraps. 
  D'un autre c�t\'e, les algorithmes de d\'eformation de topographie d\'eforment probablement moins les scraps les plus petits. 
  L'utilisation de scraps trop volumineux peut \'epuiser la m\'emoire de \MP si des remplissages de passages sont fr\'equemment utilis\'es, 
  de plus l'\'editeur de topos de XTherion est beaucoup moins r\'eactif lors de l'\'edition de gros morceaux.]. 
  Chaque scrap est trait\'e s\'epar\'ement par \MP ; 
  les scraps trop volumineux peuvent d\'epasser la m\'emoire de \MP et provoquer des erreurs.
  
  Un scrap de topo se compose de symboles, de points, de lignes et de zones. 
  Voir le chapitre {\it Comment le dessin est-il construit ?} pour une explication sur comment et dans quel ordre ces \'el\'ements  sont affich\'es.
    
  La bordure d'un scrap est constitu\'ee de lignes avec les options |-outline out| ou |-outline in| 
  (les parois de galerie ont |-outline in| par d\'efaut). 
  Ces lignes ne doivent pas se croiser---sinon Therion (\MP) ne peut pas d\'eterminer l'int\'erieur du scrap
  et \MP renvoit un message d'avertissement (Warning) ``|scrap outline intersects itself|'' (Le contour du scrap se coupe lui-m\^eme).
  
  Chaque bloc a son propre syst\`eme de coordonn\'ees cart\'esiennes local, qui correspond g\'en\'eralement au quadrillage sur papier millim\'etr\'e 
  (si vous mesurez manuellement les coordonn\'ees des symboles de la topo) ou aux pixels de l'image num\'eris\'ee 
  (si vous utilisez XTherion). Therion effectue la transformation de ce syst\`eme de coordonn\'ees local en coordonn\'ees r\'eelles 
  \`a l'aide des positions des stations de relev\'e, qui sont sp\'ecifi\'ees \`a la fois dans le scrap en tant que symboles de points topo 
  et dans les donn\'ees de la ligne de cheminement (centerline). Si le scrap ne contient pas au moins deux stations de lev\'e avec la r\'ef\'erence 
  |-name|, vous devez utiliser l'option |-scale| pour calibrer le scrap (ceci est courant pour les sections transverses).

  
 La transformation comprend les \'etapes suivantes :
    \list  
    * Transformation lin\'eaire (d\'ecalage, mise \`a l'\'echelle et rotation) qui fait correspondre `au mieux' 
       les stations dessin\'ees du scrap aux stations r\'eelles. 
       `Au mieux' signifie ici que la somme des distances au carr\'e entre les stations correspondantes avant et apr\`es la transformation est minimale. 
       Le r\'esultat est affich\'e en rouge si l'option |debug| (d\'ebogage) de la commande |layout| (pr\'esentation / mise en page) est activ\'ee (|on|).
    * Transformation non lin\'eaire du scrap qui (1) d\'eplace les stations topographiques \`a la position correcte, (2) est continue. R\'esultat affich\'e en bleu en mode |debug| (d\'ebogage).
    * Transformation non lin\'eaire du scrap qui (1) d\'eplace les points joints, (2) ne d\'eplace pas les stations topographiques, (3) est continue. 
       Enfin, la position des points de contr�le des courbes est ajust\'ee pour pr\'eserver la douceur du rendu. Le r\'esultat produit la topographie finale.
    \endlist
    
\enddescription

\syntax |scrap <id> [OPTIONS]
       ... commandes point, line (ligne) et area (aire) ...
       endscrap [<id>]|
\endsyntax

\context
  aucun, survey
\endcontext

\arguments
  *|<id>| = identifiant de scrap
\endarguments

\options
  * |projection <specification>| = sp\'ecifie la projection du dessin. Chaque projection est identifi\'ee par un type 
                                                   et \'eventuellement par un index sous la forme |type[:index]|. 
                                                   L'index peut \^etre n'importe quel mot-cl\'e. Les types de projection suivants valides sont les suivants :
  

    1. |none| = Aucune projection, utilis\'ee pour les sections transverses ou les topographies ind\'ependantes des donn\'ees du relev\'e 
                      (par exemple, num\'erisation d'anciennes topos o\`u aucune donn\'ee originelle n'est disponible). 
                      Aucun index n'est autoris\'e pour cette projection.

    2. |plan| = Projection du plan de base (par d\'efaut).

    3. |elevation| = Projection orthogonale (pour une coupe projet\'ee), 
                            qui prend \'eventuel-lement la direction de la vue comme argument (par exemple |[elevation 10]| ou |[elevation 10 deg]|).

    4. |extended| = Etendue (pour une coupe d\'evelopp\'ee).

  * |scale <specification>| = Est utilis\'e pour une pr\'e-mise \`a l'\'echelle (conversion des coordonn\'ees de pixels en m\`etres) des donn\'ees du scrap. 
                                            Si la projection du scrap est nulle, il s'agit de la seule transformation effectu\'ee avec des coordonn\'ees. La |<specification>| peut prendre quatre formes :

    1. |<nombre>| = |<nombre>| m\`etres par unit\'e de dessin.

    2. |[<nombre> <unites de longueur>]| = |<nombre> <unites de longueur>| par uni-t\'e de dessin.

    3. |[<num1> <num2> <unites de longueur>]| = |<num1>| unit\'es de dessin correspond en r\'ealit\'e \`a  |<num2> <unites de longueur>|.

    4. |[<num1> ... <num8> [<unites de longueur>]]| = il s'agit du format le plus g\'en\'eral, dans lequel vous sp\'ecifiez, dans l'ordre, 
                                                                         les coordonn\'ees $x$ et $y$ de deux points du scrap et de deux points de la r\'ealit\'e. 
                                                                         Vous pouvez \'egalement sp\'ecifier des unit\'es pour les coordonn\'ees des `points dans la r\'ealit\'e'. 
                                                                         Ce formulaire vous permet d'appliquer \`a la fois la mise \`a l'\'echelle et la rotation.
  
  * |cs <systeme de coordonnees>| = suppose que les coordonn\'ees locales (calibr\'ees) des scraps sont donn\'ees dans le syst\`eme de coordonn\'ees sp\'ecifi\'e. 
                                                           C'est utile pour le placement absolu des esquisses import\'ees lorsqu'aucune station topo n'est sp\'ecifi\'ee.\[S'il y a 
                                                           des stations topo dans le scrap, la sp\'ecification cs est donc ignor\'ee.]
  * |stations <liste de noms de stations>| = stations que vous souhaitez tracer sur le scrap, mais qui ne sont pas utilis\'ees pour la transformation de celui-ci. 
                                                                    Vous n'avez pas \`a les sp\'ecifier (dessiner) avec la commande |point station|.
  * |sketch <nom_de_fichier> <x> <y>| = sp\'ecification de rep\'erage bitmap d'une esquisse sous-jacente (coordonn\'ees de son coin inf\'erieur gauche).
  * |walls <on/off/auto>| = sp\'ecifie si le scrap doit \^etre utilis\'ee dans la reconstruction de mod\`ele 3D.
  * |flip (none)/horizontal/vertical| = (re)tourne le scrap apr\`es transformation.
  * |station-names <prefix> <suffix>| = ajoute un pr\'efixe / suffixe donn\'e \`a toutes les stations topo du scrap actuel. 
                                                            Enregistre bon nombre de signes typographiques.

  * |author <date> <personne>| = auteur des donn\'ees (scrap) et date de cr\'eation.
  * |copyright <date> <string>| = date et nom du copyright
  * |title <string>| = description de l'objet.
\endoptions




\subsubchapter `point'.

\description
Point est une commande permettant de dessiner un symbole de point sur la topographie.
\enddescription

\syntax
  |point <x> <y> <type> [OPTIONS]|
\endsyntax

\context
  scrap
\endcontext

\arguments
  * |<x>| et |<y>| sont les coordonn\'ees d'un objet.
  * |<type>| determine le type d'objet. Les termes suivants sont accept\'es :

    {\it Objets sp\'eciaux :} 
    |station|\[Station topo. Pour chaque scrap (\`a l'exception des scraps sans projection - avec option `none'), 
                   au moins une station ayant une r\'ef\'erence de station (option |-name|) doit \^etre sp\'ecifi\'ee.],  
    |section|\[|section| est une ancre permettant de placer une section transverse \`a cet endroit.
                   Ce symbole n'a aucune repr\'esentation visuelle. La section transverse doit \^etre dans un scrap s\'epar\'e 
                   avec la projection `none' sp\'ecifi\'ee. Vous pouvez le sp\'ecifier via l'option |-scrap|.],
     |dimensions|\[Utilisez l'option |-value| pour sp\'ecifier les dimensions de la galerie au-dessus / au-dessous du plan 
                          de la ligne m\'ediane utilis\'e lors de la cr\'eation d'un mod\`ele 3D.] ;
                   (idem en fran\c{c}ais)

    {\it Etiquettes (labels) :} |label|, |remark|, |altitude|\[Etiquette d'altitude g\'en\'erale. 
                                         Toutes les altitudes sont export\'ees sous la forme d'une diff\'erence par rapport \`a l'origine de la grille $Z$ (0 par d\'efaut). 
                                         Pour afficher l'altitude sur la paroi de la galerie, utilisez l'option altitude pour n'importe quel point constituant  la ligne de cette paroi.], 
    |height|\[Hauteur des formations \`a l'int\'erieur du secteur topographi\'e (comme un puits, un c�ne d'absorption, etc.); voir plus loin pour plus de d\'etails.], 
    |passage-height|\[Hauteur de la galerie ou de la salle; voir plus loin pour plus de d\'etails.], 
    |sta- tion-name|\[Si aucun texte n'est sp\'ecifi\'e, le nom de la station la plus proche est utilis\'e.], \penalty-100
    |date| ; (etiquette, remarque, altitude, hauteur, hauteur de galerie, nom de station, date) ;

    {\it Symboles de remplissages des passages :}\[Contrairement aux autres symboles ponctuels, ceux-ci sont d\'ecoup\'es par la bordure. Voir le chapitre {\it Comment la topographie est-elle construite ?} ]
    |bedrock|, |sand|, |raft|, |clay|, |pebbles|, |debris|, |blocks|, |water|, |ice|, |guano|, |snow| ; 
    (roche m\`ere ou en place ou substrat rocheux, sable, gour, argile, cailloux, d\'ebris, blocs, eau, glace, guano, neige) ;

    {\it speleoth\`emes:} |flowstone|, |moonmilk|, |stalactite|, |stalagmite|,
    |pillar|, |curtain|, |helictite|, |soda-straw|, |crystal|, |wall-calcite|,
    |popcorn|, |disk|, |gypsum|, |gypsum-flower|, |aragonite|, |cave-pearl|,
    |rimstone-pool|, |rimstone-dam|, |anastomosis|, |karren|, |scallop|,
    |flute|, |raft-cone|, \NEW{5.4}|clay-tree| ; 
    (coul\'ee de calcite, lait de lune, stalactite, stalagmite, pilier, draperie, excentrique, fistuleuse, cristal, paroi concr\'etionn\'ee, 
    choux-fleurs, disque, gypse, fleur de gypse, aragonite, pisolithe, bassin de gour, bordure de gour, anastomose, karren/karst (rainures de dissolution karstique), 
    coups de gouge, fl\^ute, c�ne de gour, argile) ;

    {\it Equipement:} |anchor|, |rope|, |fixed-ladder|, |rope-ladder|, |steps|,
    |bridge|, |traverse|, |camp|, |no-equipment| ;
    (amarrage, corde, \'echelle fixe, \'echelle de corde, marches, pont, traverse, campement, pas d'\'equipement) ;

    {\it Fin de galerie :} |continuation|, |narrow-end|, |low-end|, 
    |flowstone-choke|, |breakdown-choke|, \NEW{5.4}|clay-choke|, |entrance| ;
    (continuation, extr\'emit\'e \'etroite, extr\'emit\'e basse, bouchon de calcite, \'etroiture infranchissable, bouchon d'argile, entr\'ee)

    {\it Autres:} |dig|, |archeo-material|, |paleo-material|, |vegetable-debris|, 
    |root|, |water-flow|, 
    |spring|\[Utilisez toujours les symboles |spring| (arriv\'ee d'eau) et |sink| (perte) avec une fl\`eche water-flow pour le d\'ebit d'eau.], 
    |sink|, \NEW{5.4}|ice-stalactite|, |ice-stalagmite|, |ice-pillar|,
    |gradient|, |air-draught|\[Le nombre de barbules est d\'efini en fonction de l'option |-scale|.],
    |map-connection|\[Point virtuel, utilis\'e pour indiquer la connexion entre des topos d\'eplac\'ees (coupe d\'evelopp\'ee, d\'ecalage de topographie).], 
    |extra|\[Ajout de points au morphing.], 
    |u|\[Pour d\'efinir des points de symboles par utilisateur.] ; 
    (creusement, mat\'eriel arch\'eologique, mat\'eriel pal\'eotonlogique, d\'ebris v\'eg\'etaux, racines, \'ecoulement d'eau, source, perte, 
    stalactite de glace, stalagmite de glace, pilier de glace, gradient, courant d'air, connexion topo, extra, utilisateur).

\endarguments


\options
  * |subtype <mot-clef>| = d\'etermine le sous-type de l'objet. Pour chaque type donn\'e, voici les sous-types possibles :
    
    {\it station:}\[Si le sous-type de station n'est pas sp\'ecifi\'e, Therion le lit \`a partir de la ligne de cheminement, 
                        s'il est sp\'ecifi\'e |water-flow| (arriv\'ee d'eau) le sous-type sera : |permanent| (par d\'efaut), |intermittent| ou |paleo|.]
                        (c'est le sous-type d\'ecrivant une station topo) 
    |temporary| (temporaire ; par d\'efaut), |painted| (marquage \`a la peinture), |natural| (\'el\'ement naturel), |fixed| (fixe) ;

    {\it air-draught (courant d'air) :} |winter| (hiver), |summer| (\'et\'e), |undefined| (ind\'efini ; par d\'efaut);

    {\it water-flow :} |permanent| (par defaut), |intermittent|, |paleo|.

    Le sous-type peut \'egalement \^etre indiqu\'e directement dans la sp\'ecification |<type>| en utilisant les deux points superpos\'es `|:|' comme s\'eparateur.\[Par exemple |station:fixed|] 
    
    Toute sp\'ecification de sous-type peut \^etre utilis\'ee avec un type d\'efini par l'utilisateur (|u|). 
    Dans ce cas, vous devrez \'egalement d\'efinir le symbole \MP correspondant (voir le chapitre {\it Nouveaux symboles topographiques}).
    
  * |orientation/orient <nombre>| = definit l'orientation du symbole Si rien n'est sp\'ecifi\'e, il est orient\'e vers le nord. 0 $\le$ |number| $<$ 360.
 
  * |align| = alignement du symbole ou du texte. Les valeurs suivantes sont accept\'ees : center, c, top (haut), t, bottom (bas), b, left (gauche), l, right (droite), r, top-left, tl, top-right, tr, bottom-left, bl, bottom-right, br (aller \`a la ligne).
  * |scale| = \'echelle (taille) du symbole, qui peut \^etre : tiny (xs), small (s), normal (m), large (l), huge (xl) ou une valeur num\'erique. 
                   `normal' est la valeur par d\'efaut. Les tailles augmentent d'un facteur $\sqrt 2$ donc 
                   si  $xs \equiv 0.5$, $s \equiv 0.707$, $m \equiv 1.0$, $l \equiv 1.414$ et $xl \equiv 2.0$.
 
  * |place <bottom/default/top>| = change l'ordre d'affichage sur la topographie finale.
  * |clip <on/off>| = sp\'ecifie si un symbole est coup\'e par la bordure du scrap. Vous ne pouvez pas sp\'ecifier cette option pour les symboles suivants : 
                              station, station-name, label, remark, date, altitude, height, passage-height.
  * |dist <distance>| = valide pour des points suppl\'ementaires, sp\'ecifie la distance jusqu'\`a la station la plus proche 
                                   (ou \`a la station sp\'ecifi\'ee \`a l'aide de l'option |-from|). Si non sp\'ecifi\'e, la valeur appropri\'ee des donn\'ees LRUD est utilis\'ee.
  * |from <station>| =  valable pour des points suppl\'ementaires, sp\'ecifie la station de r\'ef\'erence.
  * |visibility <on/off>| = montre/cache le symbole.
  * |context <point/line/area> <symbol-type>| = (doit \^etre utilis\'e avec les options |sym-bol-hide| et |symbol-show| de la mise en page (layout)) 
                                                                           le symbole sera cach\'e/affich\'e conform\'ement aux r\`egles sp\'ecifi\'ees par |<symbol-type>|.\[Exemple : 
                                                                           si vous sp\'ecifiez |-context point air-draught| sur une \'etiquette indiquant la date d'observation, 
                                                                           la commande |symbol-hide point air-draught| masquera \`a la fois la fl\`eche du cou-rant d'air et l'\'etiquette correspondante.]
  * |id <ext_keyword>| = identifiant de symbole.

    {\it Options sp\'ecifiques de type de point :}\Nobreak

  * |name <reference>| = si le type de point est station, cette option donne la r\'ef\'erence \`a la station topo r\'eelle.
  * |extend [prev[ious] <station>]| = si le type de point est |station| et que la projection du scrap est en coupe d\'evelopp\'ee, 
                                                       vous pouvez ajuster l'extension de l'axe de la ligne de cheminement \`a l'aide de cette option.
  * |scrap <reference>| = si le type de point est |section|, il s'agit d'une r\'ef\'erence \`a une section transverse du scrap.
  * |explored <length>| = Si le type de point est |continuation|, vous pouvez sp\'ecifier la longueur des zones explor\'ees mais non encore topographi\'ees. 
                                       Cette valeur est ensuite affich\'ee dans les statistiques survey/cave.
  * |text| = texte d'une \'etiquette, d'une remarque ou d'une suite de galerie. Il peut contenir les mots-cl\'es suivants pour le formatage :\[Pour un export en SVG, 
                seuls les mots-cl\'es| <br>|, |<thsp>|, |<it>|, |<bf>|, |<rm>| et |<lang:XX>|  sont pris en compte; tous les autres sont ignor\'es.]
    
    |<br>| = saut de ligne (break).
    
    |<center>|/|<centre>|, |<left>|, |<right>| = alignement des lignes pour les \'etiquettes multilignes. Ignor\'e s'il n'y a pas de balise |<br>|.
    
    |<thsp>| = petit espace.
    
    |<rm>|, |<it>|, |<bf>|, |<ss>|, |<si>| = commutateurs de police de caract\`eres (normal, italique, gras, normal simple, simple italique).
        
\NEW{5.3}    |<rtl>| and |</rtl>| = marque le d\'ebut et la fin d'un texte \'ecrit de droite \`a gauche.
    
\NEW{5.3}|<lang:XX>| = cr\'ee une \'etiquette multilingue (voir le type string (cha\^ine) pour une description d\'etaill\'ee).
    
  * |value| =  valeur de la hauteur, hauteur d'une zone, \'etiquette d'altitude ou dimensions \`a un certain point.

      {\it Hauteur :} selon le signe de cette valeur (positif, n\'egatif ou sans signe), ce type de symbole repr\'esente en g\'en\'eral la hauteur d'une chemin\'ee, 
                            la profondeur d'un trou ou la hauteur des ressauts. La valeur num\'erique peut \'eventuellement \^etre suivie de `|?|' si elle est estim\'ee. 
                            Des unit\'es peuvent \^etre ajout\'ees (par exemple, |-value [40? ft]|).
              
      {\it hauteur du passage :} les quatre formes de valeur suivantes sont accept\'es : |+<nombre>| (la hauteur du plafond), |-<nombre>| 
                                               (la profondeur du sol vers le bas ou de la profondeur de l'eau), |<nombre>| (la distance entre le plancher et la vo\^ute) 
                                               et |[+<nombre> -<number>]| (la distance au plafond et la distance au sol).

      {\it Dimensions:} |-value [<above> <below> [<units>]]| sp\'ecifie les dimensions de la zone au-dessus et au-dessous du plan d'axe utilis\'e dans le mod\`ele 3D.
\endoptions


\subsubchapter `line'.

\description
Line est une commande permettant de dessiner une ligne sur la topographie. Chaque ligne est orient\'ee et son aspect final peut d\'ependre de son orientation 
(par exemple, le sens des coches d'une bordure de pente). La r\`egle g\'en\'erale est que l'espace libre (le vide) est \`a gauche et la roche \`a droite. 
Exemples : le bas d'une pente, le haut d'une chemin\'ee et l'int\'erieur d'une galerie se trouvent respectivement \`a gauche des symboles de pente, de chemin\'ee ou de paroi.
\enddescription

\syntax
  |line <type> [OPTIONS]
         [OPTIONS]
         ...
         [LINE DaTa]
         ...
         [OPTIONS]
         ...
         [LINE DaTa]
         ...
       endline|
\endsyntax

\context
  scrap
\endcontext

\arguments
   * |<type>| est un mot-cl\'e qui d\'etermine le type de ligne. Les types suivants sont accept\'es :
     
     {\it passages/galeries :} |wall|, |contour|, 
     |slope|\[La ligne de pente marque le bord sup\'erieur de la zone en pente. Il est n\'ecessaire de sp\'ecifier |l-size| sur au moins un point. 
                 La longueur et l'orientation des lignes de pente sont une moyenne des tailles |l-size| sp\'ecifi\'ees et des orientations aux points les plus proches. 
                 S'il n'y a pas de sp\'ecification d'orientation, les marques de pente sont perpendiculaires \`a la ligne de pente.], 
     |floor-step|, |pit|, 
     |ceiling-step|, |chimney|, |overhang|, |ceiling-meander|, 
     |floor-meander| ;
     (paroi, contour, pente, ressaut, puits, d\'ecrochement de vo\^ute, chemin\'ee, surplomb, m\'eandre de plafond, m\'eandre au sol) ;

     {\it Remplissage des passages :} |flowstone|, |moonmilk|,  
     |rock-border|\[Contour ext\'erieur de gros rochers. Si la ligne est ferm\'ee, elle est remplie avec la couleur d'arri\`ere-plan.], 
     |rock-edge|\[Ar\^etes int\'erieures de gros rochers.], 
     |water
     -flow| ; (plancher stalagmitique, lait de lune, bordure rocheuse, ar\^ete rocheuse, \'ecou-lement d'eau) ;
     
     {\it Labels / \'etiquettes:} |label| ; (etiquette) ;

     {\it Sp\'ecial:} |border|, |arrow|, 
     |section|\[Ligne indiquant la position de la section transverse. \NEW{5.3}\ Si les deux points de contr�le (points rouges) d'une courbe de B\'ezier 
                    (ligne grise) sont indiqu\'es, la ligne de section (bleue) est dessin\'ee vers la projection perpendiculaire (en pointill\'e) du premier point de contr�le et \`a partir de la projection (en pointill\'e) du second point de contr�le de cette section. Aucune courbe de section n'est affich\'ee.\hfil\break\MPpic{xsect.1}], 
     |survey|\[La ligne de cheminement du lev\'e topographique est automatiquement dessin\'ee par Therion.], 
     |map-connection|\[Utilis\'e pour indiquer une connexion entre des topographies (en d\'ecalage, ou les m\^emes points en coupe d\'evelopp\'ee).], 
     |u|\[Utilis\'e pour les symboles de lignes d\'efinis par l'utilisateur.].
\endarguments


\comopt
       * |subtype <keyword>| = d\'etermine le sous-type de ligne. Les sous-types suivants sont valides pour les types donn\'es :

         {\it wall:} |invisible|, |bedrock| (par defaut), |sand|, |clay|, 
         |pebbles|, |debris|, |blocks|, |ice|, |underlying|, \NEW{5.4}|overlying|, |unsurveyed|, 
         |presumed|, |pit|\[Habituellement ouvert \`a la surface.], |flowstone|, |moonmilk| ;
         (invisible, substrat rocheux, sable, argile, cailloux, d\'ebris, blocs, glace, sous-jacente, sus-jacente, non topographi\'ee, pr\'esum\'ee, puits, 
         plancher stalagmitique, lait de lune ; 

         {\it border :} |visible| (default), |invisible|, |temporary|, |presumed| ; (visible (par d\'efaut), invisible, temporaire, pr\'esum\'ee) ;
         
         {\it water-flow :} |permanent| (par defaut), |conjectural|, |intermittent|;
	 
	 {\it survey :} |cave| (par defaut), |surface| (par d\'efaut si la ligne de cheminement poss\`ede un marqueur de surface)

    Le sous-type peut \'egalement \^etre mentionn\'e directement dans la sp\'ecification |<type>| en utilisant `|:|' comme s\'eparateur.\[E.g.~|border:invisible|] 
    
    Toute indication de sous-type peut \^etre utilis\'ee avec un type d\'efini par l'utilisateur (|u|). 
    Dans ce cas, vous devrez \'egalement d\'efinir le symbole \MP correspondant (voir le chapitre {\it Nouveaux symboles topographiques}).
             
       * |[LINE DaTa]| sp\'ecifie soit les coordonn\'ees d'un segment de ligne |<x> <y>|,  ou les coordonn\'ees d'une courbe de B\'ezier |<c1x> <c1y> <c2x> <c2y> <x> <y>|, 
                               o\`u |c| indique le point de contr\^ole.
      * |close <on/off/auto>| = d\'etermine si une ligne est ferm\'ee ou non.
       * |mark <mot_clef>| = est utilis\'e pour marquer le point sur la ligne (voir la commande |join|).
       * |orientation/orient <nombre>| = orientation des symboles sur la ligne. S'il n'est pas sp\'ecifi\'e, 
                                                            il est perpendiculaire \`a la droite et orient\'e du c�t\'e gauche. 0 $\le$ |number| $<$ 360.
       * |outline <in/out/none>| = d\'etermine si la ligne sert de limite \`a un scrap. La valeur par d\'efaut est `|out|' pour les parois, `|none|' pour les autres lignes. 
                                                 Utiliser |-outline in| pour de gros pilers etc.
       * |reverse <on/off>| = si les points sont indiqu\'es dans l'ordre inverse.
       * |size <nombre>| = largeur de trait (les tailles gauche et droite prendront la moiti\'e de cette valeur).
       * |r-size <nombre>| = taille de la ligne \`a droite.
       * |l-size <nombre>| = m\^eme chose mais \`a gauche. Utilis\'e pour le type |slope| (pente).
       * |smooth <on/off/auto>| = si la ligne doit \^etre liss\'ee au point donn\'e. |Auto| est la valeur par d\'efaut.
  * |adjust <horizontal/vertical>| = d\'ecale le point de la ligne \`a aligner horizontalement / verticalement avec le point pr\'ec\'edent 
                                                     (ou le point suivant s'il n'y a pas de point pr\'ec\'edent). Le r\'esultat est un segment de ligne horizontale / verticale). 
                                                     Si tous les points de la ligne ont cette option, ils sont align\'es respectivement sur la coordonn\'ee moyenne $y$ ou $x$. 
                                                     Cette option n'est pas autoris\'ee dans la projection |plan|.
  * |place <bottom/default/top>| = changements de la disposition lors de l'affichage sur la topographie finale.
  * |clip <on/off>| = sp\'ecifie si un symbole est coup\'e par la bordure d'un scrap.
  * |visibility <on/off>| = affiche / cache le symbole.
  * |context <point/line/area> <symbol-type>| = (\`a utiliser avec les options de mise ne page (layout) |symbol-hide| et |symbol-show|). 
                                                                           Le symbole sera masqu\'e/affich\'e conform\'ement aux r\`egles du |<symbol-type>| sp\'ecifi\'e.

    {\it Options sp\'ecifiques :}\Nobreak

       * |altitude <value>| = ne peut \^etre sp\'ecifi\'e qu'avec le type paroi (|wall|). Cette option cr\'ee une \'etiquette d'altitude sur la paroi. 
                                         Toutes les altitudes sont export\'ees sous la forme d'une diff\'erence par rapport \`a l'origine de la grille $Z$ (0 par d\'efaut). 
                                         Si la valeur est sp\'ecifi\'ee, elle donne la diff\'erence d'altitude du point sur la paroi par rapport \`a la station la plus proche. 
                                         La valeur peut \^etre pr\'efix\'ee par le mot-cl\'e ``|fix|'', alors aucune station proche ne sera prise en compte ; 
                                         la valeur absolue donn\'ee \'etant utilis\'ee \`a la place. Les unit\'es peuvent suivre la valeur. 
                                         Exemples : |+4|, |[+4 m]|, |[fix 1510 m]|.
       * |border <on/off>| = cette option ne peut \^etre sp\'ecifi\'ee qu'avec le type de symbole `slope' (pente). Il active / d\'esactive la ligne de limite de pente.
       * |direction <begin/end/both/none/point>| = doit \^etre utilis\'e uniquement avec le type `section'. Il indique o\`u placer une fl\`eche de direction sur la ligne de section transverse. 
                                                                           `none' est la valeur par d\'efaut.
       * |gradient <none/center/point>| = Ne peut \^etre utilis\'e qu'avec le type |contour| ; indique o\`u ajouter une marque de gradient (pente) sur la ligne |contour|
                                                             S'il n'y a pas de sp\'ecification |gradient|, le comportement est d\'ependant de symbol-set (i.e. pas de tick avec l'UIS, tick au milieu avec SKBB).
       * |head <begin/end/both/none>| = peut \^etre utilis\'e uniquement avec le type `|arrow|' (fl\`eche) et indique o\`u placer la pointe de la fl\`eche. `|end|' est la valeur par d\'efaut.
       * |text <string>| = valable uniquement pour les lignes d'\'etiquettes (label).
       * \NEW{5.4}|height <value>| = hauteur d'un puits ou d'une paroi ; disponible comme variable num\'erique \MP |aTTR__height|.
\endcomopt

\options
  * |id <ext_keyword>| = Identifiant du symbol.
\endoptions


\subsubchapter `area'.

\description
Une zone est sp\'ecifi\'ee par les lignes de bordures environnantes. 
Elles peuvent \^etre de n'importe quel type, mais doivent \^etre r\'epertori\'ees dans l'ordre et chaque paire de lignes cons\'ecutives doit se croiser. 
Pour vous assurer que les lignes se coupent m\^eme apr\`es la transformation, vous pouvez, par exemple, continuer une bordure de lac 1\,cm 
derri\`ere une paroi de la galerie---ces chevauchements seront automatiquement d\'ecoup\'es par une des bordures du scrap. 
Pour y parvenir, vous pouvez utiliser une bordure invisible \`a l'int\'erieur de la galerie.
\enddescription

\syntax
  |area <type>
         place <bottom/default/top>
         clip <on/off>
         visibility <on/off>
       ... border line references ...
       endarea|
\endsyntax

\context
  scrap
\endcontext

\arguments
  * |<type>| est un des termes suivants : |water|, |sump|, |sand|, |debris|,
    |blocks|, |flowstone|, |moonmilk|, |snow|, |ice|, |clay|, |pebbles|,
    |bedrock|\[une aire vide qui peut \^etre utilis\'ee pour nettoyer l'arri\`ere-plan.], 
    |u|\[Pour les symboles d\'efinis par l'utilisateur, ils peuvent \^etre suivi d'un sous-type.] ; 
    (eau, siphon, sable, d\'ebris, blocs, coul\'ee de calcite, moonmilk, neige, argile, galets, roche m\`ere).
\endarguments

\comopt
  * La donn\'ees des lignes consistent en leur r\'ef\'erences de lignes de bord (IDs)
  * |place <bottom/default/top>| = changements de la disposition lors de l'affichage sur la topo finale.
  * |clip <on/off>| = sp\'ecifie si un symbole est coup\'e par la bordure du scrap.
  * |visibility <on/off>| = affiche / cache le symbole.
  * |context <point/line/area> <symbol-type>| = (\`a utiliser avec les options de mise ne page (layout) |symbol-hide| et |symbol-show|). 
                                                                           Le symbole sera masqu\'e/affich\'e conform\'ement aux r\`egles du |<symbol-type>| sp\'ecifi\'e.
\endcomopt

\options
  * |id <ext_keyword>| = Identifiant du symbole.
\endoptions


\subsubchapter `join'.

\description
  La jonction fonctionne selon deux modes : elle peut relier deux scraps, mais aussi deux ou plusieurs points ou lignes d'une m\^eme topographie.
  
  Lorsque vous joignez plus de deux points ou lignes, utilisez une seule commande de jonction pour chacun d'eux, 
  et non une s\'equence de commandes de jonction pour des paires de points.\[par exemple : 
  utiliser |join a b c|, et non |join a b| suivi par |join b c|.]
  
  Lorsque vous joignez des scraps, seules les parois sont jointes. 
  Il est pr\'ef\'erable de placer une jonction aussi simple que possible dans la galerie, 
  sinon vous devrez sp\'ecifier une jonction pour chaque paire d'objets \`a joindre.\[Si 
  vous voulez qu'un objet coup\'e par une limite de scrap continue jusqu'\`a un scrap voisin, utilisez l'option |-clip off| pour cet objet.]
  
  Lorsque vous voulez joindre plus de deux scraps \`a la m\^eme limite de scrap, 
  une jonction manuelle doit \^etre effectu\'ee, les points de connexion devant \^etre entr\'es dans une d\'eclaration de jonction.

  When joining more than two scraps at the same scrap border, a manual
  join must be performed where the connection points must be entered
  in one join statement. \[Comme par ex. 
  |join origScrapLineWest:end upperScrapLineWest:0 lowerScrapLineWest:0| % line was too long
  et une autre ligne de commande similaire pour les trois lignes du mur c�t\'e Est.]

\enddescription

\syntax
  |join <point1> <point2> ... <pointN> [OPTIONS]|
\endsyntax

\context
  aucun, scrap, survey
\endcontext

\arguments
   * |<pointX>| peut \^etre un ID d'un symbole point ou ligne, optionellement suivi par un point de ligne sp\'ecifique |<id>:<mark>| 
     (e.g.~|podangl_l31@podangl:mark1|).
     |<mark>| peut aussi \^etre `|end|' (fin de ligne) ou un point d'index de ligne (o\`u 0 est le premier point). 
     
     Il y a un cas sp\'ecial lorsque |<point1>| et |<point2>| sont des ID de scraps---en ce cas, les extr\'emit\'es des scraps les plus proches sont jointes. 
\endarguments

\options
  * |smooth <on/off>| indique si deux lignes doivent \^etre connect\'ees en arrondissant l'angle.
  * |count <N>| (avec l'utilisation de scraps) = Therion va essayer de joindre les scraps en connectant |N| passages/galeries.
\endoptions


\subsubchapter `equate'.

\description
  D\'efinit l'\'equivalence des stations topographiques.
\enddescription

\syntax
  |equate <liste de stations>|
\endsyntax

\context
aucun, survey
\endcontext


\subsubchapter `map'.

\description
  Une `map' (topographie) est un assemblage de scraps ou de plusieurs topographies pr\'esentant le m\^eme type de projection. 
  Il est possible d'inclure un relev\'e topographique \`a la topographie : cela affichera la ligne de cheminement sur le dessin. 
  L'objet `map' simplifie la gestion des donn\'ees lors de la s\'election des donn\'ees \`a imprimer. 
  Voir le chapitre {\it Comment la topographie est-elle construite ?} pour une explication plus d\'etaill\'ee.
  
\enddescription

\syntax
  |map <id> [OPTIONS]
        ... scrap, survey or other map references ...
        break
        ... next level scrap, survey or other map references ...
        preview <above/below> <other map id>
      endmap|
\endsyntax

\context
  aucun, survey
\endcontext

\arguments
  *|<id>| = identifiant de scraps
\endarguments

\comopt
  * les lignes de donn\'ees sont constitu\'ees de r\'ef\'erences de scraps ou de topographies. Notez que vous ne pouvez pas les m\'elanger.
  * Si vous vous r\'ef\'erez \`a la topographie, vous pouvez sp\'ecifier le d\'ecalage auquel cette sous-topographie sera affich\'ee, 
    ainsi que le type de pr\'e-visualisation de sa position d'origine. 
    La syntaxe est la suivante :\hfil\break
    |<map reference> [<offset X> <offset Y> <units>] <above/below/none>|
  * les scraps situ\'es apr\`es la pause (|break|) seront plac\'es \`a un autre niveau.
  * |preview <above/below> <other map id>| mettra le contour de l'autre topographie en tant qu'aper\c{c}u par rapport \`a la topographie actuelle.
    
    L'aper\c{c}u ne s'affiche que si la topo est au niveau (|map-level|) correspondant \`a celui sp\'ecifi\'e par la commande de s\'election |select|.
    
    Utilisez la commande |revise| si vous souhaitez ajouter des topographies de niveaux sup\'erieurs \`a l'aper\c{c}u.

  * |colo[u]r <couleur>| = d\'efinit la couleur de la topographie ; cette option annule le choix automatique lorsque la mise en page (|layout|) sp\'ecifie |colour map-fg [map]|.
\endcomopt

\options
  * |projection/proj <plan/elevation/extended/none>| = est requis si la topographie contient le relev\'e.
  * |title <string>| = description de l'objet.
  * |survey <id>| =\NEW{5.4} associe un relev\'e \`a une topographie (par exemple, toutes les statistiques de ce relev\'e seront utilis\'ees lorsque cette topographie sera s\'electionn\'ee en sortie).
\endoptions

\subsubchapter `surface'.

\description
Sp\'ecification de surface (zone ou espace de terrain ou de sol). 
Il est possible de l'afficher de deux mani\`eres : sous forme de topographie num\'eris\'ee (possible en topo 2D et aussi en mod\`ele 3D)\[Vous devrez 
saisir des donn\'ees d'altitude pour afficher la carte topographique dans un mod\`ele 3D. Actuellement, seules les cartes JPEG sont prises en charge en 3D.]) 
ou sous la forme d'une grille de surface de type mod\`ele num\'erique d'\'el\'evation (MNE/MNT/DEM) (en 3D uniquement).
\enddescription

\syntax
|surface [<name>]
   cs <systeme de coordonnees>
   bitmap <nom_de_fichier> <calibration>
   grid-units <unites>
   grid <origin x> <origin y> <x spacing> <y spacing> <x count> <y count>
   grid-flip (none)/vertical/horizontal
   [grid data]
endsurface|
\endsyntax

\context
  aucun, survey
\endcontext

\comopt
* |cs <systeme de coordonnees>| = syst\`eme de coordonn\'ees pour le calibrage bitmap et la sp\'ecification d'origine de la grille.
* |bitmap <filename> <calibration>| = topographie num\'eris\'ee.

  |calibration| peut prendre deux formes :
  
  1. |[X1 Y1 x1 y1 X2 Y2 x2 y2 [units]]|, o\`u les variables majuscules $X$ / $Y$ sont les coordonn\'ees sur l'image 
                                                             (pixels; le coin inf\'erieur gauche valant |0 0|), les variables minuscules $x$ / $y$ sont les coordonn\'ees r\'eelles. 
                                                             Les unit\'es facultatives s'appliquent aux coordonn\'ees r\'eelles (m\`etres par d\'efaut).
  
  2. |[X1 Y1 station1 X2 Y2 station2]|, o\`u les variables $X$ / $Y$ en majuscules sont les coordonn\'ees de l'image et les |stations1| et |station2| sont les noms des stations de topographie.
  
* |grid-units <units>| = unit\'es dans lesquelles la grille est sp\'ecifi\'ee. M\`etres par d\'efaut.

* |grid <origin x> <origin y> <x spacing> <y spacing> <x count> <y count>| 
  
  |<origin x> <origin y>| = sp\'ecifie les coordonn\'ees du coin en bas \`a gauche (S-W) de la grille
  
  |<x spacing> <y spacing>| = distance entre les n{\oe}uds (intersections) de la grille dans les directions E-W et N-S.
  
  |<x count> <y count>| = nombre de noeuds (pixels) dans la ligne et nombre de lignes qui forment la grille (voir ci-dessous).
  
* |[grid data]| = un flux de nombres donnant l'altitude au niveau de la mer des n{\oe}uds de la grille. 
                        Il commence \`a l'origine de la grille et remplit la grille en lignes (depuis la ligne W \`a E ;  puis des lignes S \`a N).

 * |grid-flip (none)/vertical/horizontal| = utile si votre grille (export\'ee depuis un autre programme) doit \^etre retourn\'ee.

\endcomopt


\subsubchapter `import'.

\description
  Lecture les donn\'ees du relev\'e topographique dans diff\'erents formats (pour le moment la ligne de cheminement est trait\'ee aux formats * .3d, * .plt, * .xyz). 
  Les stations topographiques peuvent \^etre r\'ef\'erenc\'ees dans des scraps, etc. 
  Lors de l'importation d'un fichier Survex 3D, les stations sont ins\'er\'ees dans la hi\'erarchie de la topographie s'il existe une hi\'erarchie identique \`a celle du fichier 3D.
  \enddescription

\syntax
  |import <nom_de_fichier> [OPTIONS]|
\endsyntax

\context
survey / all\[Uniquement avec les fichiers .3D, o\`u la structure du relev\'e est sp\'ecifi\'ee.]
\endcontext

\options
  * |filter <prefix>| = si sp\'ecifi\'e, seules les stations avec un pr\'efixe donn\'e et des vis\'ees entre elles seront import\'ees. Le pr\'efixe sera supprim\'e des noms de station.
  * |surveys (create)/use/ignore| = sp\'ecifie comment importer une structure de relev\'e topographique (fonctionne uniquement avec les fichiers .3d).

     |create| = diviser les stations en sous-niveaux. S'il n'y en a pas, les cr\'eer. 

     |use| = r\'epartir les stations en sous-niveaux.

     |ignore| = ne pas r\'epartir les stations en sous-niveaux.
  * |cs <systeme de coordonnees system>| = syst\`eme de coordonn\'ees pour les stations avec des coordonn\'ees fixes.
  * |calibrate [<x> <y> <z> <X> <Y> <Z>]| = les coordonn\'ees du fichier import\'e sont transf\'er\'ees des coordonn\'ees minuscules aux coordonn\'ees majuscules.
\endoptions


\subsubchapter `grade'.

\description
   Cette commande est utilis\'ee pour stocker des pr\'ecisions sur les donn\'ees pr\'ed\'efinies pour la ligne de cheminement.
   Les grades ou degr\'es int\'egr\'es sont : BCRA\[voir \www{http://bcra.org.uk/surveying/} ;
     la syntaxe est : |BCRa{\it n}|, o\`u |{\it n}| peut \^etre |3| ou |5|. Pour info, voici le classement BCRA des topographies suivant leur pr\'ecision :
     
     Degr\'e |1| $\gg$ Esquisse de faible pr\'ecision o\`u aucune mesure n'a \'et\'e faite
     
     Degr\'e |2| $\gg$ Peut \^etre utilis\'ee, si n\'ecessaire, pour d\'ecrire un croquis dont la pr\'ecision est interm\'ediaire entre les niveaux |1| et |3| (\`a utiliser uniquement si n\'ecessaire).
     
     Degr\'e |3| $\gg$ Relev\'e magn\'etique approximatif. Angles horizontaux et verticaux mesur\'es \`a $\pm$2,5$^{\circ}$; distances mesur\'ees \`a $\pm$50 cm ; erreur de position de la station inf\'erieure \`a 50 cm.
     
     Degr\'e |4| $\gg$ Peut \^etre utilis\'ee, si n\'ecessaire, pour d\'ecrire une topographie qui ne r\'epond pas \`a toutes les exigences de la classe |5| mais est plus pr\'ecise qu'une topo de classe |3|. (\`a utiliser uniquement si n\'ecessaire).
     
     Degr\'e |5| $\gg$ Relev\'e magn\'etique. Angles horizontaux et verticaux mesur\'es \`a $\pm$1$^{\circ}$ ; les distances doivent \^etre observ\'ees et enregistr\'ees au centim\`etre pr\`es et les positions des stations identifi\'ees \`a moins de 10 cm.
     
     Degr\'e |6| $\gg$ Relev\'e magn\'etique plus pr\'ecis que la classe |5|. C'est-\`a-dire avec une pr\'ecision angulaire de $\pm$0,5$^{\circ}$ ; les lectures du clinom\`etre doivent avoir la m\^eme pr\'ecision. L'erreur de position de la station doit \^etre inf\'erieure \`a $\pm$2,5 cm, ce qui n\'ecessitera l'utilisation de tr\'epieds dans toutes les stations ou d'autres rep\`eres de stations fixes.
     
     Degr\'e |X| $\gg$ Topographie bas\'ee principalement sur l'utilisation d'un th\'eodolite / station totale (tach\'eo-m\`etre).] 
     et UISv1\[voir \www{http://www.uisic.uis-speleo.org/UISmappingGrades.pdf} ;
     la syntaxe est : |UISv1\_{\it n}|\NEW{5.4}, o\`u |{\it n}| est de |-1| \`a |6| ou |X| ; alors que 
     |-1| to |2| ne sont 2 sont uniquement l\`a \`a titre d'information, |X| n\'ecessite des donn\'ees sd dans la |centerline|.
     Les degr\'es |2| et |4| ne doivent \^etre utilis\'ees que lorsque des conditions mat\'erielles ont emp\^ech\'e la topographie de satisfaire \`a toutes les exigences
     requises pour le degr\'e sup\'erieur et qu'il est difficile de recommencer.].

   Voir la description de l'option |sd| pour la commande |centreline| afin de d\'efinir vos propres degr\'es de pr\'ecision.
\enddescription

\syntax:
  |grade <id>
        ...
        [<quantity list> <value> <units>]
        ...
        endgrade|
\endsyntax

\context
  all
\endcontext



\subsubchapter `revise'.

\description
 Cette commande est utilis\'ee pour d\'efinir ou modifier les propri\'et\'es d'un objet d\'ej\`a existant.
\enddescription

\syntax
  Pour les objets cr\'e\'es avec des commandes ``single line'' la syntaxe est la suivante
  
  |revise id [-option1 value1 -option2 value2 ...]|
  
  Pour les objets cr\'e\'es avec des commandes ``multi-lignes'', la syntaxe est la suivante :

|revise id [-option1 value1 -option2 value2 ...]
  ...
  optionX valueX
  data
  ...
endrevise|
\endsyntax

\context
  all
\endcontext

\arguments
  L'identifiant signifie ici identifiant de l'objet (car l'identifiant d'un objet que vous voulez r\'eviser doit toujours \^etre sp\'ecifi\'e).
\endarguments



\subchapter Attributs personnalis\'es.

Les objets  {\it survey}, {\it centreline}, {\it scrap}, {\it point}, {\it 
line}, {\it area}, {\it map} et {\it surface} peuvent contenir des attributs d\'efinis par l'utilisateur 
sous la forme |-attr <nom> <valeur>|. |<nom>| peut contenir des caract\`eres alphanum\'eriques, |<valeur>| est une cha\^ine de charact\`eres (string).

Les attributs personnalis\'es sont utilis\'es dans l'exportation de la topographie en fonction du format de sortie:
\list
* lors de l'exportation de {\it shapefiles} (fichiers de donn\'ees vectorielles pour syst\`eme d'infor-mation g\'eographique---SIG), elles sont \'ecrites directement dans le fichier dbf associ\'e. 
* dans les topographies (`maps') g\'en\'er\'ees \`a l'aide de \MP (PDF, SVG), les attributs sont \'ecrits dans le fichier source \MP 
   sous forme de cha\^ines (nomm\'ees ainsi: |aTTR_<nom>|)) et peuvent \^etre \'evalu\'es et utilis\'es par l'utilisateur dans des macros de d\'efinition de symboles.
  
  Vous pouvez tester la pr\'esence d'une telle variable avec |if known aTTR_<name>: ... fi|.
\endlist


\subchapter XTherion.

XTherion est une GUI (interface utilisateur graphique) pour Therion. 
Il aide beaucoup \`a la cr\'eation de fichiers de donn\'ees d'entr\'ee. 
Actuellement, il fonctionne dans trois modes principaux : \'editeur de texte, \'editeur de dessin et compilateur \[Ici, 
nous sommes concern\'es par la cr\'eation de donn\'ees, ce qui explique que cette section ne d\'ecrit que les deux premiers modes. 
Pour tout ce qui concerne les actions de compilation, voir le chapitre d\'edi\'e {\it Compilation des donn\'ees}.]%{\it Processing data}.]

XTherion n'est pas n\'ecessaire pour Therion lui-m\^eme.
Vous pouvez modifier les fichiers d'entr\'ee dans votre \'editeur de texte favori et ex\'ecuter Therion \`a partir de la ligne de commande. 
XTherion n'est donc pas la seule interface graphique pouvant \^etre utilis\'ee avec Therion. 
Il serait possible d'en \'ecrire une meilleure, ce qui serait plus convivial, plus WYSIWYG, plus rapide, plus robuste et plus facile \`a utiliser. Des volontaires ?

Ce manuel ne d\'ecrit pas des \'el\'ements familiers tels que `si vous souhaitez enregistrer un fichier, allez au menu Fichier et s\'electionnez Enregistrer ou appuyez sur Ctrl-s'. 
Parcourez le menu du haut pendant une minute pour vous familiariser avec XTherion.

Pour chaque mode de fonctionnement, il existe un menu suppl\'ementaire \`a droite ou \`a gauche. 
Les sous-menus peuvent \^etre repli\'es ; vous pouvez les d\'erouler en cliquant sur le bouton du menu. 
Pour la plupart des menus et des boutons, il y a une courte description (traduite) dans la ligne d'\'etat, 
il n'est donc pas difficile de deviner la signification de chacun des sous-menus. 
L'ordre des sous-menus sur le c�t\'e peut \^etre personnalis\'e par l'utilisateur. 
Cliquez avec le bouton droit sur le bouton de menu et s\'electionnez dans le menu celui des autres menus avec lequel il doit \^etre \'echang\'e.

\subsubchapter XTherion---text editor.

L'\'editeur de texte de XTherion' offre des outils int\'eressant qui peuvent aider \`a la cr\'eation des fichiers texte d'entr\'ee : 
support de l'encodage Unicode et capacit\'e \`a ouvrir plusieurs fichiers en parall\`ele\[L'encodage des fichiers est sp\'ecifi\'e sur la premi\`ere ligne de chaque fichier. 
Cette ligne est cach\'ee par XTherion mais peut \^etre accessible indirectement via le menu de droite.]

Pour faciliter la saisie des donn\'ees, l'\'editeur prend en charge le formatage des donn\'ees de la ligne de cheminement sous forme de tableau. 
Il existe un menu Table de donn\'ees ({\it Data table}) pour la saisie des donn\'ees. 
Il peut \^etre personnalis\'e en fonction des  donn\'ees de commandes utilisateur en appuyant sur une touche 
`Num\'eriser le format des donn\'ees' ({\it Scan data format}) lorsque le curseur se trouve sous la sp\'ecification de la donn\'ee de commande 
(option `data' dans la commande `centreline').

\subsubchapter XTherion---map editor.

L'\'editeur de topographies/cartes vous permet de dessiner et d'\'editer des topos de mani\`ere totalement interactive. 
Mais n'en attendez pas trop. XTherion n'est pas un \'editeur vraiment WYSIWYG. 
Il affiche uniquement la position et non la forme r\'eelle des symboles de points ou de lignes dessin\'es. 
Visuellement, il n'y a pas de diff\'erence entre une excentrique et une \'etiquette de texte : 
les deux sont rendus sous forme de points simples. 
Le type et les autres attributs de tout objet sont sp\'ecifi\'es uniquement dans les menus {\it Contr�le de point} et {\it Contr�le de ligne}.

\ifx\pdfoutput\undefined\else
\leavevmode\llap{\smash{\raise10pt\hbox{\pdfannot width 6cm height 0cm depth  4cm 
{/Subtype /Text 
 /Name /Help
 /Contents (Aide: 1. Que fait le calcul de bouclages ?
            2. Pourquoi utilise-t-on MetaPost ?)
 %/Contents (Hints: 1. What does loop closure do?
 %           2. Why do we use MetaPost?)
}}} \qquad}\fi
{\it Exercice :} Trouvez deux raisons importantes pour lesquelles la topographie dessin\'ee dans XTherion ne peut pas \^etre identique \`a la sortie Therion. 
(Si vous r\'epondez \`a cette question, vous saurez pourquoi XTherion ne sera jamais un v\'eritable \'editeur WYSIWYG. 
La paresse des auteurs n'est pas la bonne r\'eponse.)

Commen\c{c}ons par d\'ecrire l'utilisation typique de l'\'editeur de dessins. 
Tout d'abord, vous devez choisir la partie de la cavit\'e \`a dessiner.\[Il est impossible
de dessiner plus d'un scrap dans un fichier, sinon dans ce cas, tous les scraps inactifs seront affich\'es en jaune.]

Apr\`es avoir cr\'e\'e un nouveau fichier dans l'\'editeur de dessin, vous pouvez charger une ou plusieurs {\bf images} 
(esquisses du relev\'e topo num\'eris\'ees de la cavit\'e\[XTherion ne peut pas mettre \`a l'\'echelle ni imposer une rotation \`a des images ; 
En cons\'equences, utilisez la m\^eme orientation, \'echelle et r\'esolution (DPI) pour toutes les images utilis\'ees dans le m\^eme scrap.]) 
en tant qu'arri\`ere-plan du dessin. 
Cliquez sur le bouton {\it Ins\'erer} dans le menu {\it Images d'arri\`ere-plan}. 
Malheureusement, en tant que limitation du langage Tcl/Tk, seules les images GIF, PNM et PPM (plus PNG et JPEG si vous avez install\'e l'extension tkImg) 
sont prises en charge. De plus, XTherion prend en charge le format XVI (image vectorielle XTherion), qui affiche les informations de la ligne de cheminement 
et les LRUD sur l'arri\`ere-plan, ainsi que les donn\'ees PocketTopo export\'ees au format Therion (voir ci-dessous). 
Toutes les images ouvertes sont plac\'ees dans le coin sup\'erieur gauche de la zone de travail. D\'eplacez-les en double-cliquant 
et en les faisant glisser avec le bouton droit de la souris ou en utilisant le menu. 
Pour obtenir de meilleures performances sur des ordinateurs plus lents, il est possible de ne pas charger temporairement une image inutilis\'ee 
de la m\'emoire en d\'ecochant la case \`a cocher {\it Visibilit\'e}. Il est possible d'ouvrir un fichier existant sans charger d'images d'arri\`ere-plan \`a l'aide
du menu {\it Ouvrir XP}. \[{\it Note :} Therion n'utilise pas d'images de fond tant que vous ne les assignez pas \`a un scrap sp\'ecifique 
avec l'option |-sketch|.]

Le r\'eglage de la taille et du zoom de la {\bf zone de dessin} s'effectue dans le menu correspondant. 
Le r\'eglage automatique ({\it auto adjust}) calcule la taille optimale de la zone de travail en fonction 
de la taille et de la position des images d'arri\`ere-plan charg\'ees.

Apr\`es ces \'etapes de pr\'eparation, vous \^etes pr\^et pour dessiner ou, plus pr\'ecis\'ement, pour la {\bf cr\'eation d'un fichier de donn\'ees cartographiques (map)}. 
Il est important de noter que vous cr\'eez en fait un fichier texte conforme \`a la syntaxe d\'ecrite dans le chapitre 
{\it Format des donn\'ees}. En r\'ealit\'e, seul un sous-ensemble des commandes Therion est utilis\'e dans l'\'editeur de dessins : 
op\'erations de suppression multilignes |scrap ... endscrap| pouvant contenir des commandes |point|, |ligne| et |aires|. 
(Cf. le chapitre {\it format de donn\'ees}). Cela correspond \`a une section de la topographie dessin\'ee \`a la main, 
constitu\'ee de points, de lignes et de zones remplies.

La premi\`ere \'etape consiste donc \`a d\'efinir le {\bf scrap} par une commande multiligne |scrap ... endscrap|. 
Dans le menu {\it Commandes de fichier}, cliquez sur le sous-menu {\it Action} et s\'electionnez {\it Ins\'erer un scrap}. 
Cela change le bouton {\it Action} en {\it Insert scrap} s'il avait une autre valeur. 
Apr\`es avoir appuy\'e sur ce bouton, un nouveau scrap sera ins\'er\'ee au d\'ebut du fichier. 
Vous devriez voir ces nouvelles lignes dans la fen\^etre preview au dessus du bouton {\it Insert scrap} :

|scrap - scrap1
endscrap
end of file|

Cette fen\^etre est une description simplifi\'ee de la structure du fichier texte tel qu'il sera enregistr\'e par XTherion.
Uniquement les commandes |scrap|, |point|, |line|, |text|---voir la raison ci-dessous-- et leurs types (pour |point| et |line|) ou ID (pour |scrap|) sont affich\'es. 

L'int\'egralit\'e des commandes est affich\'ee dans le menu {\it Command preview}.

Pour modifier des commandes cr\'e\'ees pr\'ec\'edemment, il existe des menus suppl\'ementaires, comme 
par exemple, {\it Contr�le du scrap} pour la commande du |scrap|. Ici, vous pouvez modifier l'ID (tr\`es important!) et d'autres options. 
Pour plus de d\'etails, voir le chapitre {\it Format des donn\'ees}.

Il est maintenant possible d'ins\'erer des {\bf symboles de points}. 
Comme pour l'insertion d'un scrap, acc\'edez au menu des {\it commandes de fichiers}, 
cliquez sur le sous-menu {\it Action} et s\'electionnez {\it Ins\'erer un point}. 
Appuyez sur le bouton {\it Ins\'erer un point} r\'ecemment renomm\'e. Un raccourci pour tout cela est Ctrl-p. 
Cliquez ensuite sur l'emplacement souhait\'e dans la zone de travail et vous verrez un point bleu repr\'esentant un symbole de point. 
Ses attributs peuvent \^etre ajust\'es dans le menu de {\it contr�le de point}. 
Vous resterez en mode `insertion' : chaque clic sur la zone de travail ajoute un nouveau symbole de point. 
Veillez \`a ne pas cliquer deux fois au m\^eme endroit---vous ins\'ereriez deux symboles de point au m\^eme endroit ! 
Pour passer du mode `ins\'erer' au mode `s\'electionner', appuyez sur la touche {\it Echap/Esc} du clavier 
ou sur le bouton de {\it s\'election} du menu {\it Commandes de fichiers}.

Quel sera l'ordre des commandes dans le fichier de sortie ? 
Exactement le m\^eme que dans le plan du menu {\it Commandes Fichier}. 
Les objets point, ligne et texte nouvellement cr\'e\'es sont ajout\'es avant la ligne marqu\'ee dans la description de la structure. 
Il est possible de changer l'ordre en s\'electionnant une ligne et en appuyant sur les boutons 
{\it D\'eplacer vers le bas}, {\it D\'eplacer vers le haut} ou {\it D\'eplacer vers} dans le menu {\it Commandes de fichier}. 
De cette fa\c{c}on, vous pouvez \'egalement d\'eplacer des objets entre des scraps.

{\bf Dessiner des lignes} est similaire \`a dessiner dans d'autres programmes de dessin vectoriel qui fonctionnent avec les courbes de B\'ezier. 
(Devinez comment entrer dans le mode d'insertion de ligne, autre que d'utiliser le raccourci Ctrl-l.) 
Cliquez \`a l'endroit o\`u le premier point doit se situer, puis faites glisser la souris en maintenant le bouton gauche enfonc\'e 
et rel\^achez-le \`a l'emplacement o\`u devrait se trouver le premier point de contr�le. 
Puis cliquez ailleurs (ce point sera le deuxi\`eme point de la courbe) et faites glisser la souris (en ajustant simultan\'ement le deuxi\`eme point de contr�le 
de l'arc pr\'ec\'edent et le premier point de contr�le du suivant). 
Si cette explication semble trop obscure, on peut s'exercer \`a travailler avec certains programmes 
de dessin vectoriel standard avec une documentation compl\`ete. 
La ligne sera termin\'ee apr\`es avoir quitt\'e le mode d'insertion. 
Le d\'ebut et l'orientation de la ligne sont marqu\'es par une petite coche orange \`a gauche au premier point.

Pour les symboles de lignes, il existe deux menus de contr�le : {\it Contr�le de ligne} et {\it Contr�le de point de ligne}. 
Le premier d\'efinit les attributs pour toute la courbe, comme le type ou le nom. 
La case \`a cocher {\it Inverser} est importante : Therion requiert des courbes orient\'ees et il n'est pas rare que vous commenciez \`a dessiner du mauvais c�t\'e. 
Le menu de {\it contr�le de point de Ligne} vous permet d'ajuster les attributs de n'importe quel point s\'electionn\'e sur la ligne, 
par exemple une courbe lisse \`a cet endroit (activ\'ee par d\'efaut) ou la pr\'esence de points de contr�le voisins (cases \`a cocher |<<| et |>>|).

Les {\bf Aires} sont sp\'ecifi\'ees par leurs lignes environnantes. 
Cliquez sur {\it Ins\'erer une zone}, puis sur les lignes entourant la zone souhait\'ee. 
Elles sont automatiquement ins\'er\'ees dans le champ {\it Aire} et nomm\'es (s'ils ne le sont pas d\'ej\`a). 
Une autre m\'ethode consiste \`a les ins\'erer en tant que commande |text|, dont le contenu 
(entr\'e dans le menu de l'\'editeur de texte de l'\'editeur de carte) est la commande multilignes habituelle |area ... endarea| (voir le chapitre {\it Format des donn\'ees}).\[ATTENTION ! 
La commende |text| n'est pas une commande Therion, mais uniquement un hack pour un bloc de texte arbitraire dans XTherion. 
Dans le fichier enregistr\'e par XTherion, il n'y aura que ce que vous entrerez dans l'{\it Editeur de Texte} 
ou ce que vous verrez dans la {\it Commande preview}.
Cela pourra \^etre une d\'efinition d'aire, mais aussi tout ce que vous voulez, comme par exemple un commentaire commen\c{c}ant avec le caract\`ere `|\#|'.]

Si vous tracez des scraps sans projection, il est n\'ecessaire de {\bf calibrer} la zone de dessin. 
L'\'echelle ne peut \^etre d\'efinie que d'une seule mani\`ere dans XTherion, \`a l'aide des coordonn\'ees de deux points 
(sp\'ecifi\'ees \`a la fois dans le syst\`eme de coordonn\'ees image et dans le syst\`eme de coordonn\'ees `r\'eel').

Apr\`es avoir s\'electionn\'e un scrap (cliquez sur son en-t\^ete dans le menu {\it Commandes Fichier}), 
deux petits carr\'es rouges reli\'es par une fl\`eche rouge apparaissent (par d\'efaut, ils se trouvent dans les coins inf\'erieurs de la zone de dessin). 
Vous devez les faire glisser vers des points dont les coordonn\'ees sont connues, 
g\'en\'eralement des intersections de lignes de grille en mm sur le dessin num\'eris\'e. Si vous ne pouvez pas voir ces points, vous pouvez soit :
\list
* Appuyer sur le bouton {\it Scale} dans le menu {\it Scraps} et cliquez \`a deux endroits diff\'erents 
  de l'image o\`u les extr\'emit\'es de la fl\`eche de calibration doivent se trouver, ou
* d\'eplacer le pointeur de la souris sur la position souhait\'ee, lire les coordonn\'ees du pointeur dans la barre d'\'etat 
  et entrer ces coordonn\'ees dans les zones de {\it points de l'\'echelle de l'image} du contr�le {\it Scraps}. 
  Apr\`es avoir rempli les paires de coordonn\'ees $X1$, $Y1$ et $X2$, $Y2$, la fl\`eche de calibrage sera d\'eplac\'ee en cons\'equence. 
\endlist
Ensuite, vous devez entrer les coordonn\'ees r\'eelles de ces points $(X1, Y1, X2, Y2)$.

En {\bf mode s\'election}, vous pouvez s\'electionner des objets de lignes ou de points existants 
et d\'efinir leurs attributs dans les menus correspondants, les d\'eplacer ou les supprimer 
(Ctrl-d ou bouton {\it Action} dans le menu {\it Commandes de fichiers} apr\`es avoir d\'efini {\it Action} sur {\it Supprimer}).


% [adding or deleting a point on the curve]

Il existe un menu {\it Rechercher et s\'electionner} qui permet de basculer facilement d'un objet \`a l'autre et de visualiser 
ce que vous ne pouvez pas voir au premier regard sur l'image. 
Par exemple, si vous entrez l'expression `station' et que vous appuyez sur {\it Afficher tout}, 
toutes les stations de l'image deviennent rouges.

XTherion ne v\'erifie pas la syntaxe. 
Il n'\'ecrit que les objets dessin\'es avec leurs attributions dans un fichier texte. 
Toutes les erreurs sont d\'etect\'ees uniquement lorsque vous traitez ces fichiers avec Therion (compilation).

CONSEIL : la saisie simultan\'ee de symboles du m\^eme type vous permet de gagner beaucoup de temps, 
                   car vous n'avez pas besoin de changer le type de symbole ni les options de remplissage pour chaque nouveau symbole. 
                   La {\it bo�te \`a options} conserve l'ancienne valeur saisie et il suffit donc de changer quelques caract\`eres.\[Dans le cas des stations topographiques, 
                   XTherion incr\'emente automatiquement le num\'ero de la station \`a chaque nouveau symbole ins\'er\'e.] 
                   Il est pr\'ef\'erable de commencer par dessiner toutes les stations topo 
                   (n'oubliez pas de leur donner des noms en fonction des noms r\'eels dans la commande de ligne de cheminement), 
                   afin que toutes les parois de galeries soient suivies par tous les autres symboles ponctuels, lignes et zones. 
                   Enfin, dessinez des coupes transversales.

\subsubchapter Outils additionnels.

\NEW{5.3}{\bf Help/Calibrate bitmap} g\'en\`ere un fichier MaP compatible `OziExplorer' \`a partir des donn\'ees de g\'eor\'ef\'erencement 
contenues dans un fichier topographique PDF.\[Les informations de calibration pour neuf points distincts sont pr\'esentes 
si la ligne de cheminement contient des stations fix\'ees \`a l'aide d'un syst\`eme de coordonn\'ees g\'eod\'esiques.]

Si la topographie au format PDF a \'et\'e convertie en raster \`a l'aide d'un programme externe, 
Converter utilise une image raster {\it et} une topographie PDF portant le m\^eme nom de base 
situ\'es dans le m\^eme r\'epertoire pour calculer les donn\'ees d'\'etalonnage.

Si le fichier PDF est utilis\'e directement, vous devez d\'efinir le format de DPI et de sortie avant la conversion 
automatique en format raster.\[|ghostscript| et |convert| doivent \^etre install\'es sur votre syst\`eme. 
Notez, que l'installation Therion pour Windows n'inclue pas |ghostscript|].

\NEW{5.3}Les {\bf donn\'ees PocketTopo} export\'ees au format Therion\[C'est un format de texte sp\'ecial 
qui doit \^etre import\'e \`a l'aide de XTherion et ne peut pas \^etre trait\'e directement par Therion.] 
\`a partir de l'application PocketTopo peuvent \^etre import\'ees aussi bien dans un \'editeur de texte que dans un \'editeur de dessin 
({\it Fichier $\to$ Importer $\to$ Exportation Therion PocketTopo} et {\it Images d'arri\`ere-plan $\to$ Ins\'erer $\to$ Exportation Therion PocketTopo}). 
Le m\^eme fichier est utilis\'e pour les deux importations. L'importation d'une esquisse ne cr\'ee pas directement de donn\'ees de scraps. 
Le dessin est simplement affich\'e sur le fond d'\'ecran comme des bitmaps num\'eris\'es et doit \^etre num\'eris\'e manuellement.


\subsubchapter Raccourcis clavier et souris dans l'\'editeur de dessins.
{\it G\'en\'eral :}
\list
 * Ctrl + Z = d\'efaire
 * Ctrl + Y = refaire
 * F9 = compiler le projet en cours
 * Pour s\'electionner un objet dans la liste \`a l'aide du clavier : 
       utilisez la touche `Tab' dans la liste souhait\'ee ; 
       d\'eplacez le curseur soulign\'e sur l'objet souhait\'e ; 
       appuyer sur la barre `espace'.
 * PageUp/PageDown = faire d\'efiler vers le haut / bas dans le panneau lat\'eral.
 * Shift + PageUp/PageDown = aire d\'efiler vers le haut / bas la fen\^etre de commande du fichier.
\endlist

{\it Zone de dessin et images de fond :}
\list
 * Clic droit = d\'efilement de zone de dessin 
 * Double Clic droit sur une image = d\'eplacer l'image
\endlist

{\it Ins\'erer un scrap :}
\list
 * Ctrl+R = ins\'erer un scrap
\endlist

{\it Ins\'erer une ligne :}
\list
 * Crtl + L = ins\'erer une nouvelle ligne et entrer dans le mode `ins\'erer un point de ligne'
 * Clic gauche = ins\'erer un point de ligne (sans points de contr�le)
 * Ctrl + Clic gauche = ins\'erer un point de ligne tr\`es proche du point existant (normalement ins\'er\'e juste au-dessus du point existant le plus proche)
 * Clic gauche + glisser = ins\'erer un point de ligne (avec des points de contr�le)
 * Ctrl enfonc\'ee tout en faisant glisser  = fixer la distance du point de contr�le pr\'ec\'edent
 * Clic gauche + glisser sur le point de contr�le = d\'eplacer sa position
 * Clic droit sur un des points pr\'ec\'edents = s\'electionner le point pr\'ec\'edent en mode insertion 
                                                                    (utile si vous souhaitez \'egalement modifier la direction du point de contr�le pr\'ec\'edent).
 * Esc (Echap) ou Clic gauche sur le dernier point = terminer l'insertion de ligne
 * Clic gauche sur le premier point de la ligne = fermer la ligne et ins\'erer la derni\`ere ligne
\endlist

{\it Edition d'une ligne :}
\list
 * Clic gauche + glisser = d\'eplacer un point de ligne
 * Ctrl + Clic gauche sur le point de contr�le + glisser = d\'eplacer le point de la ligne pr\`es du point existant 
                  (normalement, il est d\'eplac\'e juste au-dessus du point existant le plus proche)
 * Clic gauche sur le point de contr�le + glisser = d\'eplace le point de contr�le
\endlist

{\it Ajouter un point de ligne :}
\list
  * s\'electionnez le point avant lequel vous souhaitez ins\'erer des points ; 
     ins\'erer les points requis; appuyez sur Echap/Esc ou faites un Clic gauche sur le point que vous avez s\'electionn\'e au d\'ebut.
\endlist
  
{\it Effacer un point de ligne :}
\list
  * s\'electionnez le point que vous souhaitez supprimer ; 
    appuyez sur {\it Modifier ligne} $\to$ {\it Supprimer un point} dans le {\it panneau de configuration Ligne}.
\endlist

{\it S\'eparation/fractionnement d'une ligne (Splitting) :}
\list
 * s\'electionnez le point o\`u vous souhaitez fractionner la ligne ; 
   appuyez sur {\it Modifier ligne} $\to$ {\it S\'eparer ligne} dans le {\it panneau de commande Ligne}.
\endlist


{\it Insertion d'un point :}
\list
 * Ctrl+P = permet de passer en mode `insertion de point'.
 * Clic gauche = ins\'erer un point \`a une position donn\'ee.
 * Ctrl + Clic gauche = ins\'erer un point tr\`es proche du point existant (normalement, il sera ins\'er\'e juste au-dessus du point le plus proche).
 * Esc/Echap = sortir du mode `insertion de point'.
\endlist

{\it Edition d'un point :}
\list
 * LeftClick + glisser = d\'eplacer le point
 * Ctrl + Clic gauche + glisser = d\'eplacer le point pr\`es du point existant 
                    (normalement, il est d\'eplac\'e juste au-dessus du point existant le plus proche)
 * Clic gauche + glisser les fl\`eches de point = modifier l'orientation ou la taille des points (en fonction des commutateurs donn\'es dans le panneau de configuration Point).
\endlist

{\it Insertion d'une aire :}
\list
 * Appuyez sur les touches Ctrl + A ou les {\it commandes de fichier} $\to$ {\it Ins\'erer} $\to$ {\it aire} pour passer au mode `Ins\'erer Aire'.
 * Faites un clic droit sur les lignes qui entourent l'aire souhait\'ee
 * Esc/Echap pour terminer l'insertion de lignes de bordure
\endlist

{\it Edition d'une aire :}
\list
 * s\'electionner l'aire que vous souhaitez modifier
 * pressez `Ins\'erer' dans le `contr�le des aires' pour ins\'erer d'autres lignes \`a la position actuelle du curseur
 * pressez `Insert ID' pour ins\'erer une bordure avec un ID donn\'e \`a la position actuelle du curseur
 * appuyez sur `Supprimer' pour supprimer la limite s\'electionn\'ee
\endlist


{\it S\'election d'un objet existant :}
\list
 * Clic gauche = s\'election d'un objet en haut
 * Clic droit = s\'electionnez un objet juste en dessous de l'objet sup\'erieur (utile lorsque plusieurs points se superposent)
\endlist



\subchapter Penser Therion.

Bien que tout (ou presque tout) concernant les fichiers d'entr\'ee de Therion ait \'et\'e expliqu\'e, ce chapitre propose quelques astuces et conseils suppl\'ementaires.

\subsubchapter Comment entrer une ligne de cheminement (centreline) ?.

La commande de base permettant la d\'eclaration d'un bloc de lignes de cheminement est la commande |centreline|.
Si la cavit\'e est plus grande que quelques m\`etres, il peut \^etre judicieux de diviser les donn\'ees en plusieurs fichiers distincts et de s\'eparer les donn\'ees 
 centreline des donn\'ees de la topographie (map) 

Nous utilisons classiquement un seul fichier |*.th| qui contient un bloc |centreline| par s\'eance topographique.
Il est facile de d\'emarrer avec un fichier template comme celui montr\'e ci-dessous, 
o\`u les points seront remplac\'es par le texte appropri\'e.

|encoding ISO8859-1
survey ... -title "..."
  centreline
    team "..."
    team "..."
    date ...
    units clino compass grad
    data normal from to compass clino length
      ... ... ... ... ...
  endcentreline
endsurvey|

Pour cr\'eer un espace de nom unique, la commande |centreline| est incluse dans la commande |survey| ... |endsurvey|. 
C'est utile lorsque le relev\'e porte le m\^eme nom que le fichier qui le contient.\[E.g.~|survey entrance| dans le fichier |entrance.th|.] 
Les points seront ensuite r\'ef\'erenc\'es avec le caract\`ere |@|---voir la description de la commande |survey|.

Pour les tr\`es grandes cavit\'es, il est possible de construire une structure hi\'erarchique des r\'epertoires. 
Dans un tel cas, nous cr\'eons un fichier sp\'ecial appel\'e |INDEX.th|, qui contient tous les autres fichiers |*.th| 
d'un r\'epertoire donn\'e et contient aussi des commandes |equate| permettant de d\'efinir les connexions entre les topographies.


\subsubchapter Comment dessiner des topographies ?.

Le plus important est de concevoir la division de la cavit\'e en plusieurs scraps. 
Le |Scrap| est la pierre angulaire de la topographie. C'est g\'en\'eralement une {\it mauvaise\/} id\'ee d'essayer d'ajuster chaque scrap au fichier |*.th| correspondant. 
La raison en est que les connexions entre les scraps doivent \^etre aussi simples que possible. 
En r\`egle g\'en\'erale, les scraps sont ind\'ependants de la hi\'erarchie de la ligne de cheminement. 
Essayez donc d'oublier la hi\'erarchie des relev\'es lorsque vous tracez des topographies et choisissez les meilleures jonctions.

Nous ins\'erons g\'en\'eralement des topographies dans l'avant-dernier niveau de la hi\'erarchie du relev\'e.\[N'oubliez pas que les relev\'es cr\'eent des espaces de noms. 
Vous ne pouvez donc r\'ef\'erencer que les objets du relev\'e et de tous les sous-niveaux.]
Chaque scrap peut contenir une partie arbitraire de n'importe quelle topographie du dernier niveau hi\'erarchique. 
Par exemple, une topographie principale contient les topographies |a|, |b|, |c| et |d|. 
Les relev\'es |a| -- |d| contiennent les donn\'ees de la ligne de cheminement de quatre secteurs du relev\'e 
et chacun d'eux se trouve dans un fichier s\'epar\'e. Il existe une topographie |main_map| 
qui contient les scraps |s1| et |s2|. 
Si la topo principale (|main_map|) est situ\'ee dans le relev\'e principal (|main|), 
le scrap |s1| peut couvrir une partie de la ligne de cheminement de la topographie |a|, 
compl\'eter le relev\'e |b| et une partie de |c| ; |s2| couvrira une partie des topographies |a| et |c| 
ainsi qu'une topographie compl\`ete |d|. 
Les noms des stations du relev\'e seront r\'ef\'erenc\'es \`a l'aide du symbole |@| (par exemple |1@a|) dans les scraps.\[Si 
vous incluez des topographies dans le relev\'e de niveau sup\'erieur, vous pouvez r\'ef\'erencer n'importe quelle station dans n'importe quelle feuille, 
ce qui est tr\`es flexible. Mais en revanche, vous devrez utiliser des noms plus longs dans les r\'ef\'erences de stations, comme |3@dno.katakomby.jmn.dumbier|]

Les scraps sont g\'en\'eralement stock\'es dans des fichiers |*.th2|. Chaque fichier peut contenir plusieurs scraps. 
Pour garder les donn\'ees bien organis\'ees, il existe quelques conventions de d\'enomination : 
dans le fichier |foo.th2|, tous les scraps sont nomm\'es |foo_si|, o\`u |i| est \'egal \`a |1|, |2|, etc. 
Les sections transverses sont nomm\'ees |foo_ci|, les lignes |foo_li|, etc. 
Cela aide beaucoup avec les grands syst\`emes karstiques : si un scrap est r\'ef\'erenc\'e, vous savez imm\'ediatement dans quel fichier il a \'et\'e d\'efini.

Comme pour les fichiers |*.th|, il peut y avoir un fichier |INDEX.th2| par r\'epertoire qui inclut tous les fichiers |*.th2|, 
d\'efinit les jointures de scraps et les topographiess finales.

Lorsque vous tracez des scraps, v\'erifiez si le contour est correctement d\'efini : 
toutes les lignes cr\'eant la bordure ext\'erieure doivent avoir l'option |-outline out| ;
toutes les lignes entourant les piliers int\'erieurs l'option |-outline in|. 
Les contours du scrap ne peuvent pas se croiser, sinon la face int\'erieure du scrap ne peut pas \^etre d\'etermin\'ee. 
Il y a deux tests simples pour savoir si le contour du scrap est correct :
\list
* il n'ya aucun message d'avertissement (warning) \MP\  ``|scrap outline intersects itself|'' (le contour du scrap se croise)
* lorsque vous d\'efinissez le remplissage des passages/galeries sur n'importe quelle couleur (option |color map-fg <nombre>| dans le |layout|), 
vous pouvez voir ce que Therion consid\`ere comme se trouvant \`a l'int\'erieur du scrap.
\endlist

\subsubchapter Comment construire les mod\`eles ?.

Un mod\`ele est cr\'e\'e \`a partir des contours des scraps. La hauteur et la profondeur des galeries sont calcul\'ees \`a partir des symboles de points |passage-height| et |dimensions| (dans les scraps).

\subchapter Therion en profondeur.

\subsubchapter Comment le dessin est-il construit ?.

Ce chapitre explique comment les options |-clip|, |-place|, |-visibility| et |-context| des commandes |point|, |line| et |area| fonctionnent exactement. 
Il explique \'egalement les options |color|, |transparency|, |symbol-hide| et |symbol-show| 
(respectivement couleur, transparence, masque et affichage de symboles) de la commande de pr\'esentation |layout|.

Lors de l'exportation de la topographie, Therion doit d\'eterminer trois attributs pour chaque symbole de point, de ligne ou de zone : visibilit\'e (visibility), coupure (clip) et ordre (place ; orderring).

(1) Le symbole est visible si tout ce qui suit est vrai :

\list
* l'option |-visibility on| est activ\'ee (par d\'efaut pour tous les symboles)
* il n'a pas \'et\'e masqu\'e par l'option |-symbol-hide| dans la pr\'esentation (|layout|), 
* si son option |-context| est d\'efinie, le symbole correspondant n'a pas \'et\'e masqu\'e par l'option |-symbol-hide| de la pr\'esentation (|layout|).
\endlist

Seuls les symboles visibles sont export\'es.

(2) Certains symboles sont coup\'es par le contour du scrap. Ce sont par d\'efaut tous les \'el\'ements suivants :
\list
* {\it symboles points :} symboles de remplissages de galeries/passages (substrat rocheux, guano, etc.),
* {\it symboles lignes :} tous les symboles de ligne qui n'ont pas l'option |-outline| mentionn\'ee, \`a l'exception de |section|, |arrow|, |label|, |gradient| et 
  |water-flow|
* {\it symboles d'aires :} tous.
\endlist

Le param\`etre par d\'efaut peut \^etre modifi\'e \`a l'aide de l'option |-clip|, si cela est autoris\'e pour un symbole particulier. 
Tous les autres symboles ne sont pas coup\'es par la limite du scrap.


(3) Ordre : Chaque symbole appartient \`a l'un des groupes suivants, qui sont dessin\'es les uns apr\`es les autres :

\list
* bottom (bas) = tous les symboles avec l'option |-place bottom| 
* default-bottom = tous les symboles area par d\'efaut
* default = symboles qui n'appartiennent \`a aucun autre groupe
* default-top = |ceiling-step| (marche de plafond) et |chimney| (chemin\'ee) par defaut
* top = tous les symbols avec l'option |-place top|
\endlist

L'ordre des symboles \`a l'int\'erieur de chaque groupe suit l'ordre des commandes dans le fichier d'entr\'ee\[Ou menu {\it File commands} dans XTherion] : 
les symboles qui viennent en premier sont dessin\'es en dernier (c'est-\`a-dire qu'ils sont affich\'es en haut de chaque groupe).

Nous sommes maintenant pr\^ets \`a d\'ecrire comment les topographies (ou les chapitres de l'atlas) sont construites :

\list\obeyspaces\obeylines
*chaque aire de la topographe finale est remplie gr\^ace \`a color |color map-bg|
*les surfaces bitmap sont affich\'ees si |surface| est configur\'ee avec |bottom|
*FOR (pour) chaque scrap : le contour est rempli de blanc
*la grille est affich\'ee si |grid| est configur\'ee avec |bottom|
*preview below (aper\c{c}u ci-dessous)\[comme sp\'ecifi\'e en utilisant l'option |preview| dans la commande |map|.]%
  est rempli gr\^ace \`a |color preview-below|
*FOR (pour) chaque niveau\[Level est une collection de scraps non separ\'es par une commande |break|%
 dans la commande |map|] :
  BEGIN (d\'ebut) de la transparence
    FOR (pour) chaque scrap : le contour est rempli gr\^ace \`a  |color map-fg|
    FOR (pour) chaque scrap : les symboles d'aires sont remplis et soud\'es (clip) \`a la limite du scrap
  END (fin) de la transparence
  BEGIN (d\'ebut) de la soudure des \'etiquettes de texte (pour toutes les \'etiquettes dans ce niveau et les niveaux sup\'erieurs)
    FOR chaque scrap : 
      dessiner tous les symboles \`a d\'ecouper (\`a l'exception de |line survey|) 
        mettre en ordre de bas en haut
      tracer les symboles de |line survey| qui sont coup\'es en bordure de scrap
      d\'ecouper \`a la bordure du scrap
    FOR chaque scrap:
      dessiner tous les symboles \`a ne pas couper (\`a l'exception des points de stations |point station| et de toutes les \'etiquettes)
         mettre en ordre de bas en haut
      dessiner les symboles |point station|
  END (fin) du d\'ecoupage par les \'etiquettes de texte
  FOR (pour) chaque scrap : dessiner toutes les \'etiquettes (point et ligne) (y compris |wall-altitude|)
*l'aper\c{c}u ci-dessus est dessin\'e avec |color preview-above|
*les bitmaps de surface sont affich\'es si |surface| est positionn\'e sur |top|
*la grille est affich\'ee si |grid| est positionn\'e sur |top|
\endlist




\endinput

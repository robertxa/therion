\eject
\vbox{\pic[93.2mm]{thom.jpg}}
\vskip-3cm %-2.35cm
{\tenitbx\baselineskip15pt\leftskip2mm\rightskip6.8cm\parindent4.6cm
Jesus a dit, `Laisse celui qui chercher continuer \`a chercher jusqu'\`a ce qu'il trouve.
Quand il aura trouv\'e, il deviendra troubl\'e. 
Quand il sera devenue troubl\'e, il sera \'etonn\'e, et il prendra tout en main.'
\hfill\eightrm -- Gospel attribu\'e \`a Thomas, {\eightmit 2}$^{e}$ s.\par%i\`ecle\par
}

\chapter Modifier le layout, Mise en page des dessins en PDF.

Ce chapitre est extr\`emement utile si vous n'\^ets pas satisfait de la mise en page (layout) fournie pour les symboles et les topographies,
et que vous d\'esirez les adapter \`a vos besoins.
Cependant, vous devrez \^etre capables de comprendre et d'\'ecrire des macros \TeX\ et \MP\ pour arriver \`a vos fins.


\subchapter Mise en page dans le mode altas.

La commande |layout| permet une mise en page basique dans le mode atlas. 
Ceci est effectu\'e \`a travers des options comme |page-setup| ou |overlap|. 
Mais il n'y a pas d'option qui vont sp\'ecifier la position de la topographie (map), de la fen\^etre de navigation 
et des autres \'el\'ements dans l'espace d\'efini par les dimensions |page-width| et |page-height| ; 
par exemple, pourquoi la fen\^etre de navigation est sous la planche topographique, et non \`a sa droite o\`u \`a sa gauche ?

Il existe diff\'erentes possibilit\'es de mise en page d'une page. 
Plut�t que d'offrir encore plus d'options, pour la commande |layout|, Therion utilise le langage \TeX 
pour d\'ecrire la mise en page de l'atlas.
%(Voir la section `layout' where to put \TeX\ commands.) 
Cette approche poss\`ede l'avantage que l'utilisateur a un acc\`es direct \`a une mise en page avanc\'ee sans rendre le langage de Therion trop complexe.

Therion utilise pdf\TeX\ avec le format {\it plain} de mise en page. En cons\'equences, vous devriez \^etre habitu\'e au langage \TeX\ 
si vous d\'efinir nouveau Layouts.

La r\'ef\'erence ultime pour le langage \TeX\ est :
\list
  Knuth, D. E.: {\it The \TeX book}, 
                Reading, Massachusetts, Addison-Wesley $^1$1984 %[...]
\endlist

Pour les extensions pdf\TeX, il existe un manuel court :
\list
  Th\`anh, H. T.---Rahtz, S.---Hagen, H.: {\it The pdf\TeX\ user manual}, 
                disponible sur  \hfil\break 
		\www{http://www.pdftex.org}
\endlist

Les macros \TeX\ sont utilis\'ees \`a l'int\'erieur d'un bloc |code tex-atlas| dans la commande |layout| 
(Voir le chapitre {\it Traitement des donn\'ees} pour plus de d\'etails). 
La macro basique pr\'ed\'efinie par THerion est :

|\dopage|

L'id\'ee est simple : pour chaque page, Therion d\'efinit des variables \TeX\ 
(count, token, et box registers) qui contiennent les \'el\'ements de la page (map (topographie), 
navigator (fen\^etre de navigation), page name (nom de la page) etc.). 
A la fin de chaque page, la macro |\dopage| est invoqu\'ee. 
Cela d\'efini la position de chaque \'el\'ement sur la page. Red\'efinir cette macro va vous permettre d'obtenir la mise en page d\'esir\'ee.
Sans cette red\'efinition, vous n'aurez que la mise en page standard.

Voila la liste des variables d\'efinies pour chaque page :

\list
  {\it Boxes:}

  * |\mapbox| = Boite contenant la topographie. Sa largeur (hauteur) est d\'efinie en fonction des options |size| et |overlap| de la commande |layout| : 

    |size_width + 2*overlap| ou
    
    |size_height + 2*overlap|, respectivement

  * |\navbox| = Boite contenant la fen\^etre de navigation, avec les dimensions :

    |size_width * (2*nav_size_x+1) / nav_factor| ou

    |size_height * (2*nav_size_y+1) / nav_factor|, respectivement

    Les deux boites |\mapbox| et |\navbox| contiennent des liens hypertextes.

  {\it Count registers:}

  * |\pointerE|, |\pointerW|, |\pointerN|, |\pointerS| contiennent les num\'eros de pages des pages voisines respectivement dans les directions E, W, N et S. 
     Si une telle page n'existe pas, son num\'ero est 0.

  * |\pagenum| num\'ero de page courant

  {\it Token registers:}

  * |\pointerU|, |\pointerD| contiennent respectivement l'information sur les pages au des-sus et en dessous de la page courante.
    Elle consiste en un ou plusieurs enregistrements concat\'en\'e. 
    Chaque enregistrement poss\`ede un format sp\'ecial :
    
    {\tt|page-name|\char124|page-number|\char124|destination|\char124\char124}
    
    S'il n'existe pas de telles pages, la valeur est |notdef|.

    Voir la description de la macro |\processpointeritem| ci-dessous pour savoir comment extraire et utiliser cette information.

  * |\pagename| = nom de la topographie courante en fonction des options de la commande |map|.

  * |\pagelabel| = l'\'etiquette de page comme sp\'ecifi\'ee par les options |origin| et |origin-label| de la commande |layout|.
\endlist


Les variables suivantes sont d\'efinies au d\'ebut du document :

\list
  * |\hsize|, |\vsize| = dimensions des pages \TeX\, d\'efinies en fonction des param\`etres  |page-width|
    et |page-height| de l'option |page-setup| de la commande |layout|. 
    Elles d\'eterminent l'aire utile lorsque la mise en page est effectu\'ee via la macro |\dopage|.

  * |\ifpagenumbering| = Ce boolean (conditionnel) est vrai ou faux en fonction de l'option |page-numbers| de la commande |layout|.
\endlist

Il y a aussi des macros pr\'ed\'efinies qui aident au traitement des variables |\pointer*| :

\list
  * |\showpointer| avec un des arguments |\pointerE|, |\pointerW|, |\pointerN| ou
    |\pointerS| affiche la valeur de l'argument. Si cette valeur est 0, il n'affiche rien.
    C'est en effet utile parce que la valeur z\'ero (pas de page voisine) ne doit pas \^etre affich\'ee.
    
  * |\showpointerlist| avec l'un des arguments |\pointerU| ou |\pointerD| pr\'esente le contenu de cet argument. 
    (qui contient |\pointerU| ou |\pointerD|, voir au dessus.) Pour chaque enregistrement, il appelle la macro
    |\processpointeritem|, qui est responsable du formatage des donn\'ees. 
    
    La macro |\showpointerlist| doit \^etre utilis\'ee sans la red\'efinir \`a la place o\`u vous voulez afficher le contenu de son argument ;
    pour un formatage de donn\'ees personnalis\'e, il faut red\'efinir la macro |\processpointeritem|.
         
  * |\processpointeritem| poss\`ede trois arguements (page-name, page-number, 
    destination) et permet de visualiser ces donn\'ees. Les arguments sont d\'elimit\'es comme ci-dessous :
    
    {\tt|\def\processpointeritem#1|\char124|#2|\char124|#3\endarg{...}|}

    Un exemple de d\'efinition pourrait \^etre :

    {\tt|\def\processpointeritem#1|\char124|#2|\char124}|#3\endarg{%
  \hbox{\pdfstartlink attr {/Border [0 0 0]}% 
        goto name {#3} #2 (#1)\pdfendlink}%
}|

   (notez comment utiliser l'argument {\it destination}), 
   ou plus simplement (si nous ne voulons pas de liens hypertextes):

    {\tt|\def\processpointeritem#1|\char124|#2|\char124}|#3\endarg{%
  \hbox{#2 (#1)}%
}|
\endlist

Pour g\'erer les polices, il existe ces macros :

\list
* |\size[#1]| pour changer la taille de la police,

* |\color[#1 #2 #3]| pour changer la couleur (valeurs RGB dans l'intervalle 0--100), et

* |\rm|, |\it|, |\bf|, |\ss|, |\si| pour changer le type de caract\`eres (normal, italique, gras,...).
\endlist

%If you use own texts in \TeX\ macros (like a label for the scale bar), these 
%font switches work only in restricted manner: it's supposed that all characters 
%are present in the first encoding specified in the initialization file. 

Voir en dessous pour une liste de textes pr\'ed\'efinis qui peuvent \^etre utilis\'es avec l'atlas.

Il y a aussi un macro |\framed| qui prend une boite comme argument et qui affiche le contour de la boite
Le style du contour peut \^etre personnalis\'e en red\'efinissant la macro |\linestyle| 
qui vaut par d\'efaut |1 J 1 j 1.5 w|.

Maintenant, nous sommes pr\^ets pour re d\'efinir la macro |\dopage|.
Vous devez choisir quel(s) \'el\'ement(s) pr\'ed\'efini(s) utiliser. Un exemple tr\`es simple pourrait \^etre :

|layout my_layout
  scale 1 200
  page-setup 29.7 21 27.7 19 1 1 cm
  size 26.7 18 cm
  overlap 0.5 cm
  code tex-atlas
    \def\dopage{\box\mapbox}
    \insertmaps 
endlayout|

qui d\'efini une mise en page sur une feuille A4 en mode paysage sans fen\^etre de navigation ni texte. Il n'y a que la topographie sur la page.

Notez la macro |\insertmaps|. Le pages de topographies sont ins\'er\'ees \`a leur posision.
Ce n'et pas fiat automatiquement parce que vous pouvez choisir d'ins\'erer d'autres pages avant la premi\`ere page de topographie. 

La d\'efinition par d\'efaut de la macro |\dopage| est plus avanc\'ee :

|\def\dopage{%
 \vbox{\centerline{\framed{\mapbox}}
  \bigskip
  \line{%
    \vbox to \ht\navbox{
      \hbox{\size[20]\the\pagelabel
        \ifpagenumbering\space(\the\pagenum)\fi
        \space\size[16]\the\pagename}
      \ifpagenumbering
        \medskip
        \hbox{\qquad\qquad
          \vtop{%
            \hbox to 0pt{\hss\showpointer\pointerN\hss}
            \hbox to 0pt{\llap{\showpointer\pointerW\hskip0.7em}%
              \raise1pt\hbox to 0pt{\hss$\updownarrow$\hss}%
              \raise1pt\hbox to 0pt{\hss$\leftrightarrow$\hss}%
              \rlap{\hskip0.7em\showpointer\pointerE}}
              \hbox to 0pt{\hss\showpointer\pointerS\hss}
          }\qquad\qquad
          \vtop{
            \def\arr{$\uparrow$}
            \showpointerlist\pointerU
            \def\arr{$\downarrow$}
            \showpointerlist\pointerD
          }
        }
      \fi
      \vss
      \scalebar
    }\hss
    \box\navbox
  }
 }
}|

Il est possible d'utiliser d'autres macros en langage \TeX\ ou des primitives \TeX\ pour ajouter de nouvelles caract\'eristiques, 
par exemple une mise en page diff\'erente entre les pages paires et impaires ; les cartouches de haut de page et de bas de page ;
ou ajouter un logo \`a chaque page. 

En plus des topographies, les pages de l'atlas peuvent contenir des objets additionnels comme :
le titre de la page, des informations de base sur la cavit\'e, la l\'egende avec les symboles utilis\'es etc.

Therion g\'en\`ere automatiquement une liste des symboles utilis\'es ainsi qu'une liste des personnes qui ont explor\'e, topographi\'e et dessin\'e 
certaines parties de la cavit\'e.
Les token registers suivant peuvent \^etre utilis\'es (en accord avec les demandes de l'utilisateur 
avant ou apr\`es la macro |\insertmaps|):

\list
* |\explotitle|, |\topotitle|, |\cartotitle| = titres traduits
* |\exploteam|, |\topoteam|, |\cartoteam| = membres qui ont particip\'e (en r\'ef\'erence au options |team|, |explo-team| de la |centreline| et de l'option |author| des scraps)
* |\explodate|, |\topodate|, |\cartodate| = dates correspondantes
* |\comment| = d\'efini par l'option |map-comment| de la commande |layout|
* |\copyrights| = d\'efini par l'option |copyright| des |surveys|, |scrap| et autres objets
* |\cavename| = nom de la topographie export\'ee ; d\'efini \`a partir de l'option |-title| de la topographie (map) export\'ee
* |\cavelength|, |\cavedepth| = longueur et profondeur approximative de la topographie affich\'ee
* |\cavelengthtitle|, |\cavedepthtitle| = \'etiquettes traduites
* \NEW{5.4}|\cavemaxz|, |\caveminz| = valeurs max/min de l'altitude
* \NEW{5.4}|\thversion| = version actuelle de Therion
* \NEW{5.4}|\currentdate| = date actuelle
* \NEW{5.4}|\outcscode|, |\outcsname| = code et nom du syst\`eme de coordonn\'ees d'exportation
* \NEW{5.4}|\northdir| = nord `true' ou `grid'
* \NEW{5.4}|\magdecl| = d\'eclinaison magn\'etique en degr\'es
* \NEW{5.4}|\gridconv| = convergence des m\'eridiens de la grille en degr\'es
\endlist

Il y a aussi une macro |\atlastitlepages| qui combine la plupart des token registers d\'ecrits ci-dessous
pour obtenir une page d'introduction format\'ee simplement.

Pour afficher la l\'egende, l y a :

\list
* |\iflegend| = bool\'een/conditionnel ; `vrai' (true) si l'option |legend| de la commande |layout| est sur |on| ou |all|
* |\legendtitle| = token register contenant le titre traduit
* |\insertlegend| = macro pour ins\'erer l'image des symboles de l\'egende avec la description traduite
                            dans le nombre de colonnes sp\'ecifi\'e par l'option |legend-columns| de la commande |layout|)
* |\formattedlegend| = combine les trois commandes ci-dessus pour obtenir une l\'egende pr\'eformat\'ee avec un cartouche 
                                    et une composition des symboles en deux colonnes\[Par D\'efaut ; 
                                    ajuster le nombre de colonnes de la l\'egende avec l'option |legend-columns| de la commande |layout|]
                                    si l'option |legend| est sur |on|.
\endlist

La fl\`eche du Nord et la barre d'\'echelle peuvent \^etre affich\'ees en utilisant :

\list
* |\ifnortharrow| = bool\'een/conditionnel ; `vrai' (true) si la projection de la topographie (map) est `plan' 
                             est que le symbole de la fl\`eche du nord n'est pas cach\'e dans le |layout|
* |\ifscalebar| = bool\'een/conditionnel ; `vrai' (true) si la barre d'\'echelle n'est pas cach\'ee dans le |layout|
* |\northarrow| = forme PDF de la fl\`eche du nord
* |\scalebar| = forme PDF de la barre d'\'echelle
\endlist

Il ya une macro \`a usage g\'en\'eral pour la composition en colonnes mutltiples\[Ne pas utiliser avec
la l\'egende de la topographie, o\`u le nombre de colonnes est ajust\'e par l'option
|legend-columns| de la commande |layout|.] :
\list
* |\begmulti <i>|, |\endmulti| = texte entre ces macros format\'e en |<i>| colonnes
\endlist

Voici un exemple sur comment cr\'eer un atlas avec la liste des topographes etc.\ suivi par les pages de topographies puis avec la l\'egende \`a la fin :

|code tex-atlas
  \atlastitlepages
  \insertmaps
  \formattedlegend|

\subchapter Mise en page dans le mode map (topographie).

Dans le mode map (topographie), il est possible d'utiliser des variables pr\'ed\'efinies qui sont d\'ecrites dans le chapitre pr\'ec\'edent : 

{\rightskip0cm plus 5cm
|\cavename|, |\comment|, |\copyrights|, 
|\explotitle|, |\topotitle|, |\cartotitle|, 
|\exploteam|, |\topoteam|, |\cartoteam|, 
|\explodate|, |\topodate|, |\cartodate|, 
|\cavelength|, |\cavedepth|, |\cavelengthtitle|, |\cavedepthtitle|,
|\cavemaxz|, |\caveminz|, |\thversion|, |\currentdate|,
|\outcscode|, |\outcsname|, |\northdir|, |\magdecl|, |\gridconv|,
|\ifnortharrow|, |\ifscalebar|, |\northarrow|, |\scalebar|, 
|\iflegend|, |\legendtitle|, |\insertlegend|, |\begmulti <i>|, |\endmulti|, 
|\formattedlegend|, |\legendcolumns|.
\par}

Pour les placer quelque part sur la topographie, vous devez d\'efinir une macro |\maplayout| dans le bloc de code |code tex-map| de la comande |layout|. 
Il doit contenir un ou plusieurs invocations |\legendbox|.
La macro |\legendbox| poss\`ede quatre param\`etres :
les coordonn\'ees dans l'intervalle 0--100, la sp\'ecification de l'alignement
(N, E, S, W, NE, SE, SW, NW ou C) et le contenu \`a afficher.

Voici un exemple simple :

|\def\maplayout{
  \legendbox{0}{100}{NW}{\northarrow}
}|

qui affiche la fl\`eche du nord dans le coin sup\'erieur-gauche de la topographie

Pour faciliter l'utilisation pour l'utilisateur, il y a un token register |\legendcontent|. 
Il contient un pr\'eformatage du nom de la cavit\'e, de la fl\`eche du nord, de la barre d'\'echelle, des \'equipes d'exploration/topographie/dessins,
du commentaire, des copyrights et de la l\'egende.
(La macro |\legendcontent| est aussi utilis\'ee dans la d\'efinition de la mise en page par d\'efaut de la topographie :
|\def\maplayout{\legendbox{0}{100}{NW} {\the\legendcontent}}|).

La largeur du texte ci-dessus peut \^etre ajust\'e avec l'option |\legendwidth|
(sa valeur par d\'efaut est d\'efinie par l'option |legend-width| de la commande |layout|). 
La couleur et la taille du texte de la l\'egende pr\'eformat\'ee peut \^etre facilement modifi\'ees en utilisant les token registers 
|\legendtextcolor|, |\legendtextsize|, |\legendtextsectionsize| et
|\legendtextheadersize|, 
par exemple pour un texte grand et bleu :

|code tex-map
  \legendwidth=20cm
  \legendtextcolor={\color[0 0 100]}
  \legendtextsize={\size[20]}
  \legendtextheadersize={\size[60]}|


Il est possible d'afficher le contour de toute la topographie en renseignant le dimen register |\framethickness| 
avec une valeur positive, comme par exemple. |0.5mm|.


\subchapter Personnalisation des \'etiquettes de texte.

A partir de la version Therion 5.4.1 vous pouvez utiliser l'option |fonts-setup| de la commande |layout| \`a la place de la macro \MP\ |fonts_setup()|.

\subchapter Nouveau symboles topographiques.

La commande |layout| de Therion facilite la modification des banques de symboles pr\'ed\'efi-nies. 
Si le symbole que vous chercher ou si la banque de symbole que vous voulez n'existe pas, il est possible de les cr\'eer. 

Mais pour cela, vous devrez poss\'eder une connaissance du langage \MP, qui est utilis\'e la visualisation des topographies. 
Il est d\'ecrit dans :

\list
  Hobby, J. D.: {\it A User's Manual for MetaPost}, disponible \`a \hfil\break
     \www{http://cm.bell-labs.com/cm/cs/cstr/162.ps.gz}
\endlist

L'utilisateur peut aussi tirer b\'en\'efice de la r\'ef\'erence sur le langage {\manfnt METAFONT}, qui est assez similaire \`a \MP :

\list
  Knuth, D. E.: {\it The METAFONTbook}, Reading, Massachusetts, Addison-Wesley
    $^1$1986
\endlist

Les nouveaux symboles peuvent \^etre d\'efinis dans un bloc |code metapost| de la commande |layout| 
Il devient alors facile de rajouter des nouveaux symboles pour une compilation sp\'ecifique.
Il est aussi possible de les rajouter de fa\c{c}on permanente en les compilant dans l'ex\'ecutable Therion
(voir les {\it Annexes} pour les instructions de compilation).

Chaque symbole poss\`ede un nom unique, qui consiste en les termes suivants :

\list 
* une des lettres `p', `l', `a', `s' pour les symboles point, ligne, aire ou sp\'ecial, respectivement ;
* un caract\`ere underscore (soulign\'e ; `|_|' ;
* un type de symbole comme list\'e dans le chapitre {\it Format des donn\'ees\/} avec tous les tirets supprim\'es ;
* si le symbole poss\`ede un sous-type, ajouter un caract\`ere underscore (`|_|') et le sous-type ;
* un caract\`ere underscore (`|_|') ;
* un identifiant de symbole en casse majuscules.
\endlist

Par exemple : le nom standard du symbole de point `water-flow'avec un sous-type `permanent' 
dans le `MY' est d\'efini ainsi : |p_waterflow_permanent_MY|. Le nom standard pour les d\'efinitions de l'utilisateur
des types de symboles ne doit pas inclure d'identifiant de type de symbole, par exemple ~|p_u_bat|.

Chaque nouveau symbole doit \^etre enregistr\'e dans une macro appel :

|initsymbol("<nom_standard>");|

sauf si c'est compil\'e dans l'ex\'ecutable Therion. 

Il existe quatre pens (stylo) pr\'ed\'efinis : {\it PenA} (le plus \'epais) \dots\ {\it PenD} 
(le plus fin), qui peuvent \^etre utilis\'es pour tous les dessins. 
Pour dessiner et remplir, utiliser les commandes |thdraw| et |thfill| \`a la place des commandes \MP |draw| et |fill|.

\NEW{5.4}Les variables suivantes sont aussi disponibles :

\list
* bool\'een |ATTR__shotflag_splay|, |ATTR__shotflag_duplicate|,\hfil\break
  |ATTR__shotflag_approx| = pour la ligne survey
* bool\'een |ATTR__stationflag_splay| = est true (vrai) pour les stations finales des vis\'ees d'habillage (splay shots)
* bool\'een |ATTR__scrap_centerline| = est true(vrai) pour les scraps cr\'e\'es \`a partir de la centreline
* bool\'een |ATTR__elevation| = true(vrai) pour les coupes projet\'ees ou \'etendues (extended), false (faux) pour la projection en plant
* numeric |ATTR__height| = hauteur d'un puit ou d'un wall:pit
* string |ATTR__id| = ID de l'objet actuel
* string |ATTR__survey| = contient le nom de la topographie actuelle
* string |ATTR__scrap| = contient le nom du scrap actuel
* picture |ATTR__text| = contient le texte, par exemple pour le point continuation
* string |NorthDir| = `true' ou `grid'
* numeric |MagDecl| = d\'eclinaison magn\'etique en degr\'es
* numeric |GridConv| = convergences des m\'eridiens de la grille en degr\'es
\endlist

\subsubchapter Symboles de points.

Les symboles de points sont d\'efinis comme des macro en utilisant les commandes |def ... enddef;|.
La majorit\'e des d\'efinitions de symboles de points poss\`ede quatre arguments :
la position (paire), la rotation (numeric), d'\'echelle (scale ; numeric) et l'alignement (alignment ; paire). 
les exceptions sont pour le point {\it section} qui n'a pas de repr\'esentation visuelle, tous les points 
{\it labels}, qui ne demandent pas de traitement sp\'ecial comme d\'ecrit dans le chapitre pr\'ec\'edent, 
et le point {\it station} qui ne prend qu'un seul argument : la position (pair).

Tous les symboles sont dessin\'es en coordonn\'ees locales avec une unit\'e de longueur {\it u}. 
Les intervalles de longueur recommand\'es sont $\left<-0.5u,0.5u\right>$ pour les deux axes. 
Le symbole doit \^etre centr\'e sur l'origine des coordonn\'ees.
Pour la topographie finale, tous les dessins sont transform\'es dans la variable de transformation sp\'ec\'efique {\it T\/}, 
il est donc n\'ec\'essaire de d\'efinir cette variable avec de dessiner. 

Ceci est habituellement fait en deux \'etapes (en faisant l'hypoth\`ese que les quatre arguments sont {\it P}, {\it R}, {\it S} et {\it A}) : 
  
\list
* d\'efinir la variable paire {\it U} du symbole avec $\left({width\over2},{height\over2}\right)$ 
  pour un alignement correcte. L'argument d'alignement {\it A} est une paire repr\'esentant les rapports
  $\left(shift_x\over U_x\right)$ et $\left(shift_y\over U_y\right)$. 
  
  (Ainsi, |aligned A| signifie |shifted (xpart A * xpart U, ypart A * ypart U)|.)
* d\'efini la variable de transformation {\it T} : 

  {\catcode`\=12|T:=identity aligned A rotated R scaled S shifted P;|}
\endlist

Pour dessiner et remplir, utiliser les commandes |thdraw| et |thfill| \`a la place des commandes \MP |draw| et |fill|.
Elles prennent automatiquement en compte la transformation {\it T}.

Un exemple de d\'efinition peut \^etre :

|def p_entrance_UIS (expr P,R,S,A)=
  U:=(.2u,.5u);
  T:=identity aligned A rotated R scaled S shifted P;
  thfill (-.2u,-.5u)--(0,.5u)--(.2u,-.5u)--cycle;
enddef;
initsymbol("p_entrance_UIS");|

\subsubchapter Symboles de lignes.

Les symboles de lignes diff\`erent des symboles de points parce qu'ils ne sont pas dans un syst\`eme de coordonn\'ees local. 
Chaque ligne poss\`ede un chemin ({\it path}) en coordonn\'ees absolues comme premier argument. 
Il est donc n\'ecessaire de d\'eclarer la variable {\it T} comme |identity| avant tout dessins.

Les symboles suivant prennent des arguments optionnels : 
\list
* arrow (fl\`eche) = numeric : 0 signifie par de fl\`eche, 1 signifie une fl\`eche \`a la fin, 2 au d\'ebut, et 3 aux deux extr\'emit\'es
* contour = texte : liste de points qui auront une coche (tick) ou l'un des arguments
  $-1$, $-2$ or $-3$ pour indiquer un tick ind\'efini, au milieu ou pas de tick, respectivement 
* section = texte : liste de points qui poss\`edent une fl\`eche d'orientation ou $-1$ pour indiquer l'absence d'une telle fl\`eche
* slope = numeric : 0 pas de bord, 1 un bord ; texte : liste de triplets (point, direction, length [longueur]) 
\endlist

Voici un exemple d'utilisation :

|def l_wall_bedrock_UIS (expr P) = 
  T:=identity;
  pickup PenA;
  thdraw P;
enddef;
initsymbol("l_wall_bedrock_UIS");|

\subsubchapter Symboles d'aires.

Les symboles d'aires sont similaires aux symboles de lignes : ils ne prennent qu'un seul argument -- {\it path} en coordonn\'ees absolues. 

Vous pourrez les remplir de trois mani\`eres :

\list
 * remplir une grille al\'eatoire ou uniforme dans une mage temporaire (poss\'edant les dimensions |bbox path|) 
    avec des symboles de points ; les bord sont coup\'es en fonction du |path| et ajouter \`a |currentpicture|
 * remplir le |path| avec une couleur unie
 * remplir le |path| avec un motif pr\'ed\'efini en utilisant le mot-clef |withpattern|.
\endlist

Les motifs sont d\'efinis en utilisant la m\^eme interface utilisateur (sans la macro
|patterncolor|) comme d\'ecrit dans l'article

\list
  Bolek, P.: ``\MP\ and patterns,'' {\it TUGboat}, 3, XIX (1998), pp.~276--283,
  disponi-ble en ligne \`a
  \www{https://www.tug.org/TUGboat/Articles/tb19-3/tb60bolek.pdf}
\endlist

Vous devrez utiliser la macro \MP\ standard |draw| et des macros similaire sans d\'efinir la variable {\it T} dans la definition des motifs.

Voici un exemple sur comment d\'efinir et utiliser les motifs :

|beginpattern(pattern_water_UIS);
  draw origin--10up withpen pensquare scaled (0.02u);
  patternxstep(.18u);
  patterntransform(identity rotated 45);
endpattern;

def a_water_UIS (expr p) =
  T:=identity;
  thclean p;
  thfill p withpattern pattern_water_UIS;
enddef;
initsymbol("a_water_UIS");|

\subsubchapter Symboles sp\'eciaux.

Actuellement, il n'y a que deux symboles sp\'eciaux : la barre d\'echelle et la fl\`eche du nord. 
Ils sont tous les deux exp\'erimentaux et sujets \`a de modifications dans le futur.

\endinput

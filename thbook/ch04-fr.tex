\eject
\chapter Qu'obtenons-nous ?.

\subchapter Fichiers d'informations.

\subsubchapter Fichier Log.

A c�t\'e des messages issus de Therion et des autres programmes associ\'es, le fichier Log (|therion.log|)
contient des informations sur les valeurs calcul\'ees de la d\'eclinaison magn\'etique et de la convergence des m\'eridiens,
des erreurs de bouclages et des distortions des scraps.

Les erreurs absolues de bouclages sont $\sqrt{\Delta x^2+\Delta y^2+\Delta z^2}$, o\`u
$\Delta x$ est la diff\'erence entre les points (identiques) de d\'epart et de fin de la boucle
avant la distribution de l'erreur le long de l'axe $x$, et idem pour $y$ et $z$. 
Le pourcentage d'erreur de bouclage est calcul\'e comme l'{\it erreur absolue / longueur totale de la boucle}. 
L'erreur moyenne est la moyenne arithm\'etique simple de toutes les erreurs de bouclage.

La distortion du scrap est calcul\'ee en utilisant une mesure de la distortion d\'efinie par
toutes les paires de points (point symbols, points and control points of line symbols) dans le scrap
La mesure est calcul\'ee comme $\left\vert d_a-d_b\right\vert\over d_b$, o\`u $d_b$ est
la distance aux points avant la distorsion, et $d_a$ la distance aux points apr\`es la distortion. 
Les distortions du scrap maximale et moyenne sont calcul\'ee respectivement comme 
le maximum ou la moyenne de ces mesures appliqu\'ees \`a toutes les paires de points.


\subsubchapter XTherion.

Therion donne des statistiques basiques a propos de chaque topographie (survey) si l'option |-x| est donn\'ee :
longueur, intervalles vertical, N--S, et E--W, nombre de vis\'ees et nombre de stations 
Dans XTherion, cette information est fournie dans la fen\^etre de {\it Compilation} et dans 
le menu {\it Info de la topo} ({\it Survey info}) lorsque certaines topographies (surveys) du menu 
 {\it Structure de la topographie} ({\it Survey structure}) sont s\'electionn\'ees.


\subsubchapter Export SQL.

L'export SQL permet d'obtenir facilement toutes les informations de la ligne de cheminement (centerline)
de fa\c{c}on d\'etaill\'ee et subtile.
C'est un fichier texte commen\c{c}ant avec des d\'eclarations de tableau 
(o\`u `?' est dans le listing ci-dessous pour une valeur maximale requise par la donn\'ee de la colonne en question) :

|create table SURVEY (ID integer, PARENT_ID integer, 
  NAME varchar(?), FULL_NAME varchar(?), TITLE varchar(?));
create table CENTRELINE (ID integer, SURVEY_ID integer, 
  TITLE varchar(?), TOPO_DATE date, EXPLO_DATE date, 
  LENGTH real, SURFACE_LENGTH real, DUPLICATE_LENGTH real);
create table PERSON (ID integer, NAME varchar(?), SURNAME varchar(?));
create table EXPLO (PERSON_ID integer, CENTRELINE_ID integer);
create table TOPO (PERSON_ID integer, CENTRELINE_ID integer);
create table STATION (ID integer, NAME varchar(?), 
  SURVEY_ID integer, X real, Y real, Z real);
create table STATION_FLAG (STATION_ID integer, FLAG char(3));
create table SHOT (ID integer, FROM_ID integer, TO_ID integer, 
  CENTRELINE_ID integer, LENGTH real, BEARING real, GRADIENT real, 
  ADJ_LENGTH real, ADJ_BEARING real, ADJ_GRADIENT real, 
  ERR_LENGTH real, ERR_BEARING real, ERR_GRADIENT real);
create table SHOT_FLAG (SHOT_ID integer, FLAG char(3));|

Ceci est suivi par une grande quantit\'e de commandes d'insertion SQL. 
Ce fichier SQL peut \^etre charg\'e (`load') par n'importe quel base de donn\'ees SQL
(apr\`es initialisation base-d\'epentante, qui peuvent inclure de lancer un serveur SQL et de s'y connecter, cr\'eer une base de donn\'ees et de s'y connecter.
Une bonne m\'ethode est de d\'ebuter une transaction avant de charger ce fichier SQL si la base de donn\'ee n'initialise pas une transaction automatiquement).
Il est important de bien faire correspondre l'encodage de la base de donn\'ee \`a l'encodage utilis\'e dans la commande Therion |export database|.


\midinsert
    \centerline{\pic{database.pdf}}%
\endinsert

Les noms de tableau et colonnes s'expliquent par eux-m\^eme ; 
pour les valeurs ind\'efinies ou non-existantes, la valeur |NULL| est utillis\'ee ; 
|FLAG| dans le tableau |SHOT_FLAG| est |dpl| ou |srf| pour les vis\'ees dupliqu\'ees ou de surface ;
dans le tableau |STATION_FLAG|, |ent|, |con|, |fix|, 
|spr|, |sin|, |dol|, |dig|, |air|, |ove|, |arc| sont pour les stations
avec les attributs entrance (entr\'ee), continuation, fixed (fixe), spring (source, arriv\'ee d'eau), sink (perte), doline, dig (d\'esobstruction), air-draught (courant d'air), 
overhang (surplomb) or arch (arche), respectivement.

Voici quelques exemples simple d'interrogation de la base de donn\'ees SQL de Therion:

\penalty-100{\it Liste des membres des \'equipes de topographie avec information de la longueur topographi\'ee par chacun d'entre-eux :}

|select sum(LENGTH), sum(SURFACE_LENGTH), NAME, SURNAME 
  from CENTRELINE, TOPO, PERSON 
  where CENTRELINE.ID = TOPO.CENTRELINE_ID and PERSON.ID = PERSON_ID 
  group by NAME, SURNAME order by 1 desc, 4 asc;|

{\it Quelle parties de la cavit\'e a \'et\'e topographi\'ee durant l'ann\'ee 1998 ?}

|select TITLE from SURVEY where ID in 
  (select SURVEY_ID from CENTRELINE 
  where TOPO_DATE between '1998-01-01' and '1998-12-31');|

{\it Quelle longueur de galerie y-a-t-il entre les altitudes 1500 et 1550 m a.s.l. ?}

|select sum(LENGTH) from SHOT, STATION S1, STATION S2 
  where (S1.Z+S2.Z)/2 between 1500 and 1550 and 
  SHOT.FROM_ID = S1.ID and SHOT.TO_ID = S2.ID;|

\subsubchapter Listes---caves (cavit\'es), surveys (topographies), continuations (points d'interrogations).

Utiliser l'|export continuation-list| nous permet d'obtenir un \'etat des lieu de tous les points de la ligne de cheminement (centerline)
et des scraps qui poss\`edent l'information d'une continuation possible\[en utilisant l'attribut |station| pour un point de la centreline
et |point continuation| dans les scraps].

|export cave-list| produit sous la forme d'un tableau l'information sur les diff\'erentes cavit\'es topographi\'ees (surveys)
(pour cela, vous devez sp\'ecifier le flag |entrance| dans vos donn\'ees), ce qui inclut la longueur, la profondeur et la localisation des entr\'ees.

L'information d\'etaill\'ee de chaque sous-topographie (sub-survey) est donn\'ee par la commande |export survey-list|.
La longueur inclut les vis\'ees avec le flag |approximate|, mais pas celles avec les flags |explored| (explor\'e), 
|duplicate| (dupliqu\'e) ou |surface|.

\vfill\eject

% Pour forcer le saut de page :
\subchapter Topographies en 2D (maps).

\subsubchapter Topographies \`a imprimer.

Les topographies sont produites aux formats PDF et SVG, qui peuvent \^etre visualis\'es ou imprim\'es via un bon nombre de logiciels.
Pensez bien \`a d\'ecocher la case {\it Fit page to paper} (adapt\'e \`a la taille du papier) ou option similaire si vous voulez imprimer \`a l'\'echelle exacte que vous avez d\'efinie.

Dans le mode Atlas, des informations suppl\'ementaires sont ajout\'ees \`a chaque page :
num\'ero de page, nom de la topographie et \'etiquette de page.

Les num\'eros des pages voisines en N, S, E et W, tout comme les niveaux sup\'erieurs et inf\'erieurs
sont extr\^emement utiles. Il ya aussi des liens `hypertextes' sur les c�t\'es de la topographie si la cavit\'e continue
\`a la page suivante o\`u \`a la page appropri\'ee correspondante \`a la suite de la cavit\'e dans cette direction.

Les fichiers PDF sont tr\`es optimis\'es---les scraps sont stock\'es dans des formes de types XObject
uniquement une fois, et r\'ef\'erenc\'es sur les pages appropri\'ees.
Therion utilise les options les plus avanc\'ees des fichiers PDF comme la gestion de la transparence et des calques.

Cr\'eer des fichiers PDF peut \^etre optionnellement effectu\'e en post-processing par des applications comme
pdf\TeX\ ou Adobe Acrobat---Il est possible d'extraire ou modifier certaines pages, d'ajouter des commentaires ou du cryptage, etc.

\NEW{5.3}Si la topographie a \'et\'e produite en utilisant des donn\'ees g\'eor\'ef\'erenc\'ee, alors elle contient les informations de g\'eor\'ef\'erencement.
Ceci peut \^etre extrait par XTherion pour produire des images raster g\'eor\'ef\'erenc\'ees (voir {\it XTherion/Additional tools} 
pour plus de details).


\subsubchapter Topographies pour SIG (Syst\`emes d'Informations G\'eographiques).

Les topographies peuvent \^etre export\'ees aux formats DXF, ESRI (shapefiles 2D et 3D) ou KML
afin d'\^etre utilis\'ees dans les logiciels appropri\'es.
Ces topographies ne contiennent pas le graphisme des symboles.


\subsubchapter Topographies pour applications sp\'ecifiques.

Les topographies au format XVI contiennent les informations de la ligne de cheminement (centerline) avec les LRUD
(et optionellement les dessins scann\'es et transform\'es)
Ce format peut \^etre import\'e dans XTherion pour servir de background (arri\`ere-plan) \`a la digitalisation de la topographie.

Les topographie dans le format Survex sont pr\'evues pour obtenir une visualisation rapide dans le logiciel Aven.


\subchapter Mod\`eles 3D.

Therion put exporter des mod\`eles 3D dans diff\'erents formats autre que le format natif |.lox|.
Ils peuvent \^etre charg\'es dans des programmes appropri\'es pour la visualisation, l'\'edition ou trac\'e 
pour \^etre imprim\'es ou modifi\'es/analys\'es plus en profondeur.
Si le format ne supporte pas la forme des galeries arbitraire, seule la ligne de cheminement (centerline) est incluse.

\subsubchapter Loch.

Loch est un visualisateur de mod\`eles 3D (fichiers |.lox|) inclus dans la distribution Therion. 
Il supporte par exemple le rendu de haute r\'esolution d'un fichier et la visualisation 3D avec des lunettes 3D.

\endinput

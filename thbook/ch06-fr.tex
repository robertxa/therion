\eject
\vbox{\hsize=93.2mm\baselineskip=12pt\leftskip3.5mm\parskip0pt
\eightit
\leavevmode\llap{1. }Lorsque un scientifique distingu\'e mais \^ag\'e d\'eclare que quelque chose est possible, il est certainement dans le vrai. S'il annonce que quelque chose est impossible, il est tr\`es probblement dans le faux.

\leavevmode\llap{2. }Le seul moyen de d\'ecouvrir les limites du possible est de s'aventurer d'un pas vers l'impossible.

\leavevmode\llap{3. }Aucune technologie suffisamment avanc\'ee ne peut \^etre attribu\'ee \`a la magie.
\vskip 6pt
\rightline{\eightrm ---Arthur C. Clarke, {\eightmit 1973}}
}

\chapter Annexes.

\subchapter Compilation.

Si vous voulez compiler Therion \`a partir du code source et l'utiliser, vous devez 
(les trois premiers points sont requis uniquement pour la compilation) :
\list
* Compiler GNU C/C++
* GNU make
* Perl
* Python 2.7 or 3
* Tcl/Tk 8.4.3 ou plus r\'ecent (\www{https://www.tcl.tk}) avec l'ensemble de widgets
  {\it BWidget}  %\hfil\break
  (\www{https://sourceforge.net/projects/tcllib/})
  et optionnellement l'extension {\it tkImg} \hfil\break
  (\www{https://sourceforge.net/projects/tkimg/}).
%  {\it Tom} OpenGL extension (improved version is included in Therion source 
%  distribution)
%  and 
* Distribution \TeX\ avec au minimum \TeX\ avec le format Plain, 
  un pdf\TeX r\'ecent, et \MP\ (\www{https://www.tug.org}).
* Le package LCDF Typetools (\www{https://www.lcdf.org/type/})
* La distribution ImageMagick avec les outils {\it convert} et {\it identify}, 
  si vous voulez pouvoir utiliser les facilit\'es de distortion et de morphing des dessins bitmap.
* {\it ghostscript} si vous voulez cr\'eer des images calibr\'ees \`a partir de topographies PDF g\'eor\'ef\'erenc\'ees.
\endlist

Pour compiler Loch, vous avez besoin de :

\list
* freetype 2 ou plus r\'ecent ; freetype-config doit fonctionner
* wxWidgets 2.6 ou plus r\'ecent ; wx-config doit fonctionner
* VTK 5.0 ou plus r\'ecent
* libjpeg, libpng, zlib
\endlist

Tous les programmes (\`a l'exception des packages BWidget et tkImg) sont habituellement inclus avec 
les distributions Linux, Unix ou MacOS\,X.
Pour Windows, il faut consid\'erer utiliser MinGW et MSYS (\www{http://www.mingw.org}).
C'est une distribution d'outils GNU avec GNU make et GCC.
(By The Way, pourquoi ne pas utiliser la version pr\'e-compil\'ee pour Windows ?)

\subsubchapter D\'emarrage rapide.

\list
* D\'ecompresser la distribution source |therion-5.*.tar.gz|
* |cd therion|
* |make config-macosx| ou |make config-win32|, si vous \^etes sous MacOS~X ou Windows, respectivement
* |make|
* |sudo make install|
\endlist

%Installing Tom:
%
%\list
%* if you use Windows, download a Tcl/Tk source distribution, |make| and
%  |make install| it under MSYS
%* |cd therion/thtom/linux| or |cd therion/thtom/win|
%* |make|
%* copy |Tom0.2| directory (which should contain |pkgIndex.tcl| and one of 
%  |libtom.so| or |libtom.dll|) 
%  to the |lib| subdirectory of your 
%  Tcl/Tk distribution.
%\endlist

\subsubchapter Le guide du Hacker.

{\it Param\`etres de Make (Makefile)}

Le {\it Makefile} de Therion peut prendre diff\'erents param\`etres optionnels.

\list
* |config-linux|, |config-macosx|, |config-win32| = configuration Therion pour une pla-teforme (OS) sp\'ecifique. Linux est la configuration par d\'efaut.
* |config-release|, |config-oxygen|, |config-ozone| = d\'efini le niveau d'optimization pour le compiler C++ (none, |-O2| and |-O3|)  
* |config-debug| = utile avant de d\'ebogguer le program  
* |install| = installer Therion
* |clean| = effaced les fichiers temporaires
\endlist

{\it Cross-compilation pour Windows}\NEW{5.4}

Therion supporte la compilation d'ex\'ecutables Win32 sur Linux gr\^ace au cross
compiler MXE (\www{http://mxe.cc}). 

\list
* installer les packages static/win32 suivant (i686-w64-mingw32.static-\char42) dans le r\'eper-toire |/usr/lib/mxe/| : binutils, bzip2, expat, freetype-bootstrap,
gcc, gettext, glib, harfbuzz, jpeg, libiconv, libpng, tiff, vtk, wxwidgets,
xz, zlib.
* modifier le PATH: {\catcode`\=12|export PATH=/usr/lib/mxe/usr/bin:$PATH|}
* |cd therion|
* |make config-win32cross|
* |make|
\endlist

{\it Ajouter de nouvelles traductions}

Therion supporte la traduction des \'etiquettes de topographies. 
Supposons que vous voulez rajouter un nouveau langage |xx|. 

\list
* lancer `|perl process.pl export xx|' dans le sous-r\'epertoire Therion `thlang'. 
  Cela cr\'ee un fichier |texts_xx.txt|. Ce fichier est encod\'e en UTF-8.
* \'editer le texte du ficher |texts_xx.txt|. Rajouter votre traduction aux lignes commen\c{c}ant par `|xx:|'.
* lancer |make update|
* (re)compiler Therion
\endlist


{\it Ajouter de nouveau encodages/encodings}

Bien que l'encodage Unicode UTF-8 couvre tous les caract\`eres que Therion est capable de processer,
il peut \^etre probl\'ematique de l'utiliser.
Dans ce cas, il est possible de rajouter le support de n'importe quel encodage 8-bit pour les fichiers textes d'entr\'ee.
Copier un fichier de traduction dans le r\'epertoire |thchencdata| ; 
Ajouter son nom au hash `ifiles' hash au d\'ebut du script Perl script |generate.pl| ; 
le lancer et recompiler Therion.

Le fichier de traduction doit contenir deux valeurs hexadecimal d'un caract\`ere
(le premier dans l'encodage 8-bit, le second en unicode) sur chaque ligne. 
Il est possible de faire suivre un commentaire avec le caract\`ere `|#|'. 

{\it Ajouter de nouveaux encodages \TeX}

Il est facile de rajouter de nouveaux encodages pour les exports des topographies en 2D.%
\[\NEW{5.3}Cette section d'applique aux styles anciens de polices en utilisant la commande |tex-fonts|
dans le fichier d'initialisation et est obsol\`ete si nous utilisons la commande |pdf-fonts| .]
Copier l'encodage appropri\'e du fichier d'encodage avec une extension |*.enc| dans le dossier |texenc/encodings|,
lancer le script Perl |mktexenc.pl| situ\'e dans le r\'epertoire |texenc| et (re)compiler Therion.

Therion utilise les m\^emes fichier d'encodage que le programme |afm2tfm| de la distribution \TeX, 
qui poss\`ede le m\^eme format qu'un vecteur d'encodage dans une police PostScript. 
Vous trouverez de plus amples d\'etails dans le chapitre {\it 6.3.1.5 Format d'encodage des fichiers} 
dans la documentation du programme Dvips.


{\it G\'en\'erer de nouveau cartouches (headers) \TeX\ et \MP\ }

Therion utilise \TeX\ et \MP\ pour la composition et la visualisation des topographies 2D. 
Des macros pr\'ed\'efinies sont compil\'ees avec l'ex\'ecutable Therion et sont copi\'ees dans le r\'epertoire de travail
juste avant d'ex\'ecuter \MP\ et \TeX\ (sauf si l'option |--use-extern-libs| est utilis\'ee). 
La commande de Layout rend possible de modifier certaines macros dans le fichier de configuration lors de la compilation. 

Pourtant, il est possible d'effectuer des modifications permanentes aux fichiers macros. 
Apr\`es avoir modifi\'e les fichiers dans les r\'epertoires |mpost| et |tex|, il est n\'ecessaire d'ex\'ecu-ter les scripts
Perl |genmpost.pl| et |gentex.pl|, qui g\'en\`erent des fichiers header C++ header,
et de compiler l'exc\'ecutable Therion de nouveau .

\subchapter Variables d'environment.

Therion lit les variables d'environnement suivantes :

\list
* |THERION| = [non requis] recherche du path (chemin) pour le(s) fichier(s) (x)therion.ini
* |HOME| (|HOMEDRIVE| + |HOMEPATH| on WinXP) = 
  [non requis, mais g\'en\'eralement pr\'esent sur votre syst\`eme] chemin/path de recherche pour le(s) fichier(s) (x)therion.ini
* |TEMP|, |TMP| = r\'epertoire temporaire du syst\`eme, o\`u Therion stocke les fichiers temporaires (dans un r\'epertoire nomm\'e |th$PID$|, o\`u |$PID$| est l'ID du processus),
                              sauf si l'option |tmp-path| est sp\'ecifi\'e dans le fichier d'initialisation.
\endlist

Consulter la documentation de votre OS pour savoir comme le d\'efinir.

\subchapter Fichiers d'nitialisation.

Les caract\'eristiques Therion et XTherion d\'ependantes du syst\`eme sont sp\'ecifi\'ees dans les fichiers therion.ini ou xtherion.ini, respectivement.
Ils sont cherch\'es dans les r\'epertoires suivant :

\list
* sur UNIX: 
  |.|, |$THERION|, |$HOME/.therion|, |/etc|, |/usr/etc|, |/usr/local/etc|
* sur Windows:
  |.|, |$THERION|, |$HOME\.therion|, |<Therion-installation-directory>|, 
  |C:\WINDOWS|, |C:\WINNT|, |C:\Program Files\Therion|
\endlist
 
\subsubchapter Therion.

Si aucun fichier n'est trouv\'e, Therion utilise la configuration par d\'efaut. 
Si vous voulez une liste de cette configuration par d\'efaut, 
utilisez l'option |--print-init-file|. 
Le fichier d'initialisation est lu comme n'importe quel autre fichier Therion.
(Les lignes vides ou commen\c{c}ant pas un `|#|' sont ignor\'ees ; 
les lignes finissant avec backslash `|\|' continuent sur la ligne suivante on next line.) 
Actuellement, les commandes d'initialisation sont :

\list
* |loop-closure <therion/survex>|

  Par d\'efaut, survex est utilis\'e si le programme est pr\'esent dans le syst\`eme, sinon, therion est utilis\'e.

* |encoding-default <encoding-name>|

  D\'efini l'encodage pas d\'efaut (actuellement non utilis\'e).

* |encoding-sql <encoding-name>|

  D\'efini l'encodage par d\'efaut des export SQL.
        
* |language <xx[_YY]>|  

  Langage d'export par d\'efaut. Voir la page de licence / copyright pour obtenir la liste des langages disponibles.

* |units <metric/imperial>| 

  D\'efini les unit\'es par d\'efaut.

* |mpost-path <file-path>|

  D\'efini le path (chemin) complet vers l'ex\'ecutable \MP\ si Therion est incapable de le trouver
  (``|mpost|'' est la valeur par d\'efaut).

* |mpost-options <string>|

  D\'efini les options \MP.

* |pdftex-path <file-path>|

  D\'efini le path (chemin) complet vers l'ex\'ecutable pdf\TeX\ si Therion est incapable de le trouver
  (``|pdfetex|'' est la valeur par d\'efaut).

* |identify-path <file-path>|

  D\'efini le path (chemin) complet vers l'ex\'ecutable identify d'ImageMagick si Therion est incapable de le trouver
  (``|identify|'' est la valeur par d\'efaut).
  
* |convert-path <file-path>|

  D\'efini le path (chemin) complet vers l'ex\'ecutable convert d'ImageMagick si Therion est incapable de le trouver
  (``|convert|'' est la valeur par d\'efaut).

* |source-path <directory>| 

  Path (chemin) vers les donn\'ees et les fichiers de configuration. Utilis\'e principalement pour des d\'efinitions de layout et de niveau \`a l'\'echelle du syst\`eme.

* |tmp-path <directory>| 

  Path (chemin) o\`u le r\'epertoire temporaire doit \^etre cr\'e\'e.
  
* |tmp-remove <OS command>| 

  Commande syst\`eme pour effacer les fichiers des r\'epertoires temporaires.

* |tex-env <on/off>| 

  [Ne fonctionne que pour Windows.]
  Lorsque l'option est sur |off| (valeur par d\'efaut), Therion efface temporairement toutes les variables d'environnement reli\'ees \`a 
  \TeX. Utile si d'autre distributions \TeX sont install\'ee sur votre syst\`eme et qui ont d\'ej\`a d\'efinit des variables d'environnement,
  ce qui peut rendre confus les programmes \TeX\ et \MP\ fournis par Therion dans la distribution Windows. 
  
  Si l'option est sur |on|, vous utiliserez une autre distribution \TeX\ pour l'export des topographies.

* |text <language ID> <therion text> <my text>|

  Utiliser cette option permet de modifier toute traduction de texte par d\'efaut dans Therion pour l'export.
  Pour la liste des textes Therion et des traductions disponibles, voir le fichier |thlang/texts.txt|.

* |cs-def <id> <proj4def>|\NEW{5.4}  %[other options]

  D\'efini un nouvel |<id>| de syst\`eme de coordonn\'ees en utilisant la syntaxe Proj4.

* |pdf-fonts <rm> <it> <bf> <ss> <si>|\NEW{5.3}

  D\'efini les polices \`a utiliser pour les topographies export\'ees en PDF. 
  La commande doit \^etre suivit par les paths (chemins) sp\'ecifiant o\`u les polices 
  regular (normal), italic (italique), bold (gras), sans-serif et sans-serif oblique sont stock\'ees sur le syst\`eme.
  
  Les polices TrueType et OpenType sont support\'ees. 
  
  Therion n\'ecessite l'installation de LCDF Typetools sur le syst\`eme pour pouvoir utiliser cette commande. Exemple :
  
  |pdf-fonts  "/usr/share/fonts/Serif.ttf" \
           "/usr/share/fonts/Serif-Italic.ttf" \
           "/usr/share/fonts/Serif-Bold.ttf" \
           "/usr/share/fonts/Sans.ttf" \
           "/usr/share/fonts/Sans-Oblique.ttf"|

* |otf2pfb <on/off>|\NEW{5.3}

  Si l'option est sur |on| (default), les polices OpenType utilis\'ees dans |pdf-fonts| sont converties en polices
  PFB, si elles sont bas\'ees sur PostScript. 
  De l'information est perdue dans le format PFB, mais l'avantage est que pdf\TeX\ peut encapsuler le sous-ensemble des polices PFB 
  (au contraire des polices OpenType qui doivent \^etre encapsul\'ees en totalit\'e).

* |tex-fonts <encoding> <rm> <it> <bf> <ss> <si>|
        
  Mani\`ere originale et plus complexe de d\'efinir les polices pour les exports pdf des topographies. 
  Vous avez besoin de sp\'ecifier explicitement l'encodage
  (maximum 256 caract\`eres de la polices seront utilis\'es). 
  La liste des encodages support\'es est donn\'ee pas l'option de la commande en ligne |--print-tex-encodings|.   
  Le m\^eme encodage doit \^etre utilis\'e pour g\'en\'erer la m\'etrique \TeX\ (fichiers |*.tfm|) 
  pour ces polices (par exemple avec le programme afm2tfm) et cet encodage doit \^etre donn\'e explicitement dans le fichier map de pdf\TeX. 
  La seule exception est syst\`eme de base des polices des ordinateurs r\'ecents (Computer Modern) qui utilisent un encodage `raw'. 
  Cet encodage n'a pas besoin d'\^etre sp\'ecifi\'e dans le fichier map de pdf\TeX.
  
  L'encodage doit \^etre suivi par cinq sp\'ecifications de styles de polices : regular (normal), italic (italique), bold (gras), sans-serif et sans-serif oblique.
  Les param\`etres par d\'efaut sont |tex-fonts raw cmr10 cmti10 cmbx10 cmss10 cmssi10|
  
  Un exemple sur comment utiliser d'autre polices (par exemple TrueType Palatino dans encodage xl2 (d\'eriv\'e de ISO8859-2)), lancer :
  
  |ttf2afm -e xl2.enc -o palatino.afm palatino.ttf|
  
  |afm2tfm palatino.afm -u -v vpalatino -T xl2.enc|

  |vptovf vpalatino.vpl vpalatino.vf vpalatino.tfm|
  
  Vous produirez les fichiers |vpalatino.vf|, |vpalatino.tfm| et |palatino.tfm|. Ajouter la ligne :
  
  |palatino <xl2.enc <palatino.ttf|
  
  au fichier map de pdf\TeX. La m\^eme chose doit \^etre effectu\'e pour les styles italic (italique), bold (gras)
  et les styles correspondant sans-serif et sans-serif-oblique. Si vous \^etes paresseux voir feignant, essayez :
  
  |tex-fonts xl2 palatino palatino palatino palatino palatino|
  
  (Nous ne devrions utiliser que la police virtuelle |vpalatino| au lieu de |palatino|,
   qui contient ni cr\'enage, ni ligatures, mais pdf\TeX\ ne supporte pas la commande |\pdfincludechars| pour les polices virtuelles.
   Cela doit \^etre am\'eliorer.)
    
  Si vous voulez ajouter des encodages non support\'es, lisez le chapitre {\it Compilation / Hacker's guide}. 

* |tex-fonts-optional <encoding> <rm> <it> <bf> <ss> <si>|

  Similaire \`a |tex-fonts|, mais teste si les polices \TeX\ sont install\'ees sur le syst\`eme. Cela ne fait rien si aucune police n'est pr\'esente.
  
  Cette configuration est utilis\'ee par d\'efaut pour les polices Czech/Slovak et cyrillic pour \'eviter les erreurs \MP\ 
  sur les syst\`emes qui n'ont pas ces polices pr\'esentes.
  
  Comme les tests demandent du temps (ex\'ecuter pdf\TeX), 
  vous pouvez d\'esactiver compl\`etement ce comportement par d\'efaut en renseignant l'option |tex-fonts| dans le fichier INI.

\endlist


\subsubchapter XTherion.

Le fichier d'initialisation de XTherion est actuellement un script Tcl evalut\'e lorsque XTherion d\'emarre
Ce fichier est comment\'e ; Voir les commentaires pour conna\^itre les d\'etails.


%\subsubchapter Speed optimization.
%
%[Optionally creating \MP\ and \TeX\ format files.]

\subchapter Limitations.

\list
*  scrap size = $\approx 2.8 \times 2.8$ m dans l'\'echelle d'export (limitation \MP)
*  page size = 

   PDF map ou atlas: $\approx 5 \times 5$ m (limitation pdf\TeX)
   
   SVG map: pas de limites 
*  scraps count (nombre de scraps) = approx. 500--6000, d\'ependemment de la fr\'equence des sections
   
   limite actuelle \MP\ : $4 (scraps + sections) < 4096$ (cela pourrait \^etre augment\'e arbitrairement)

   limite pdf\TeX\ : $2\times pages + images + patterns +
                            6 (scraps + sections) < 32500$
\endlist


\subchapter Exemples de donn\'ees.

L'exemple suivant illustre l'utilisation basique des commandes Therion :

|encoding  utf-8

survey main -title "Grotte de test"
  
  survey first
    centreline
      units compass grad
      data normal from to compass clino length
                  1    2  100     -5    10
    endcentreline
  endsurvey

  survey second -declination [3 deg]
    centreline
      calibrate length 0 0.96
      data normal from to compass length clino
                  1    2  0       10     +10
    endcentreline
  endsurvey
 
  centreline
    equate 2@first 1@second
  endcentreline
 
  # les scraps sont generalement dans des fichiers *.th2 separes
  scrap s1 -author 2004 "Therion team"

    point 763 746 station -name 2@second
    point 702 430 station -name 2@first
    point 352 469 station -name 1@first
    point 675 585 air-draught -orientation 240 -scale large

    line wall -close on
      287 475
      281 354 687 331 755 367
      981 486 846 879 683 739
      476 561 293 611 287 475
    endline

  endscrap

  map m1 -title "Test map"
    s1
  endmap
 
endsurvey|

Le fichier de configuration associ\'e pourrait \^etre :

|encoding  utf-8
source test

layout l1
  scale 1 100
  layers off
endlayout

select m1@main

export model -fmt survex
export map -layout l1|

Si vous enregistrez le fichier de donn\'ees comme `test.th' et le fichier comme `thconfig' vous devriez \^etre capable de le compiler avec Therion.


%\subchapter Map symbols library.

%\input sym/symlib

\subchapter Historique.

\list
\everypar{\hangindent16pt\hangafter1}
* {\bf 1999}

  Oct : premi\`eres id\'ees concr\`etes

  Nov : d\'ebut de la programmation (scripts Perl et macros \MP)

  27 Dec:  Therion compile une topographie simple au format PostScript pour la premi\`ere fois
       (code source de 32 kB de Perl et de 7 kB de \MP\ et \TeX).
       Le mod\`ele topographique d\'eform\'e est substantiellement diff\'erent de l'actuel 
       (les positions des objets sont relatives \`a une vis\'ee particuli\`ere, et non \`a la position de toutes les stations du scrap).
       Cette version contient d\'ej\`a des options int\'eressantes comme les fonctions de transformation qui permettre \`a l'utilisateur
       de sp\'ecifier le format d'entr\'ee des donn\'ees topographiques, ou de diviser les grandes topographies en diff\'erentes maps ou
       multiples feuilles.
				
  30 Dec : Premi\`ere page internet (avec des donn\'ees d'exemple, mais sans le code source)
  
* {\bf 2000}\nobreak\par\nobreak
  Jan : xthedit (Tcl/Tk), une interface graphique \`a Therion

  18 Feb : d\'ebut de la reprogrammation (Perl)

  1 Apr : le premier PDF avec des liens hypertextes d'une topographie et d'un atlas

  Aug : exp\'erimentations avec PDF, pdf\TeX\ et \MP

* {\bf 2001}

  Nov : d\'ebut de la r\'eimpl\'ementation \`a partir de z\'ero : 
       Therion (C++ avec quelques scripts Perl h\'erit\'es des versions pr\'ec\'edentes) ; 
       notion de scrap ;
       \'editeur interactif de topographie en 2D, ThEdit en remplacement de xthedit (Delphi) 

  Dec : ThEdit exporte une topographie simple pour la premi\`ere fois

* {\bf 2002}

  Mar : Therion 0.1 ---
       Therion est capable de processer les donn\'ees topographiques de la grotte des Chauves Souris Mortes (Cave of Dead Bats).
       XTherion, l'\'editeur de texte d\'edi\'e \`a Therion (Tcl/Tk).

  27 Jul : Therion 0.2 ---
       Therion compile une topographie simple (consistent en deux scraps)
       pour la premi\`ere fois (code source de 800 kB)

  Aug : XTherion est \'etendu \`a l'\'edition de topographie 2D (en remplacement \`a ThEdit)

  Sep : Therion compile les premi\`eres donn\'ees r\'eelles d'une cavit\'e complexe. XTherion devient une extension du compiler.

%  Oct 5: Public presentation of Therion, held on Trango\v{s}ka

* {\bf 2003}\nobreak

  Mar : premi\`ere version finalis\'ee du Therion Book

  Apr : Therion inclus dans la distribution GNU/Linux Debian

  Jun : tous les scripts Perl sont r\'e\'ecrits en C++, Therion est un seul ex\'ecutable maintenant
         (m\^eme s'il utilise Survex et \TeX)
       
%  Nov 10: first idea of the Labyrinth project

* {\bf 2004}\Nobreak

  Mar : Therion 0.3 --- Therion exporte un mod\`ele 3D \`a partir de topographies 2D.
  L'algorithme de bouclage est inclus dans Therion.

* {\bf 2006}\Nobreak

  Oct : Therion 0.4 --- Nouveau viewer 3D (Loch).

* {\bf 2007}\Nobreak

  Feb : Therion 0.5 --- Support des images/dessins bitmap pour le morphing.
\endlist


\subchapter Futur.

Bien que Therion soit utilis\'e pour la production de topographie, il y a encore de nombreuses fonctions \`a impl\'ementer :

\subsubchapter G\'en\'eral.

\list
* information sur les bouclages dans la base de donn\'ees SQL
\endlist

\subsubchapter Topographies 2D.

\list
* compl\'eter les banques de symboles pr\'ed\'efinis
* g\'en\'erer un registre pour l'atlas
* utiliser MPlib \`a la place de \MP
\endlist


\subsubchapter Mod\`eles 3D.

\list
* am\'eliorer la mod\'elisation des parois des galeries/passages
\endlist

\subsubchapter XTherion.

\list
* am\'eliorer les capacit\'es d'\'edition 2D
\endlist

\subsubchapter Loch.

\list
* sch\'emas de couleurs
* arbre des survey pour s\'electionner les sous-surveys \`a afficher
* filtre spatial (par exemple, d\'ecouper par plan)
* support de surfaces multiples
\endlist

\subsubchapter Labyrinth.

\list
* nouveau GUI pour le futur (voir \www{https://labyrinth.speleo.sk})
\endlist



\endinput

\eject
\vbox{\hsize=93.2mm\baselineskip=12pt
\rightline{\eightit Nous marchons tous les deux et ne marchons pas dans les m\^emes rivi\`eres.}
\pic{herakl.pdf}
\vskip1pt
\eightrm 
\rightline{---Heraclitus of Ephesus, {\eightmit 6}$^{\grave{e}me}$/{\eightmit 5}$^{\grave{e}me}$ si\`ecle av. J.C.}
}


\chapter Traitement des donn\'ees.

Outre les fichiers de donn\'ees contenant des donn\'ees du relev\'e, Therion utilise un fichier de configuration contenant des instructions sur la pr\'esentation des donn\'ees.

\subchapter Fichier de configuration.

Le nom de fichier de configuration peut \^etre donn\'e comme argument \`a Therion. 
Par d\'efaut, Therion recherche le fichier nomm\'e |thconfig| dans le r\'epertoire de travail en cours. 
Il se lit comme tout autre fichier Therion (c�est-\`a-dire une commande par ligne ; les lignes vides ou commen\c{c}ant par `|#|' sont ignor\'ees ; 
les lignes termin\'ees par une barre oblique invers\'ee (backslash) se poursuivent \`a la ligne suivante). 
Une liste des commandes actuellement prises en charge est donn\'ee ci-apr\`es :

\subsubchapter `system'.

Permet d'ex\'ecuter des commandes syst\`eme lors de la compilation de Therion.\[Par exemple pour ouvrir ou refresh un lecteur PDF externe.] 
Normalement, Therion attend que le sous-processus soit termin\'e. 
Si vous souhaitez continuer la compilation sans interruption, utilisez l�expression |<command> &| sous Linux et l�expression |<command>| sous Windows.

\subsubchapter `encoding'.

|encoding| fonctionne comme la commande d'encodage dans les fichiers de donn\'ees---il sp\'ecifie les jeux de caract\`eres. 

\subsubchapter `language'.
\syntax
  \list
    * |language <xx_[YY]>| 
  \endlist
\endsyntax

\NEW{5.3}D\'efinit la langue de sortie pour les textes traduisibles.

\subsubchapter `cs'.

\syntax
  \list
    * |cs <systeme de coordonnees>| 
  \endlist
\endsyntax

\NEW{5.3}En dehors du bloc |layout|, la commande sp\'ecifie le syst\`eme de coordonn\'ees de l'export.
Il n'est pas possible de sp\'ecifier diff\'erents syst\`emes de coordonn\'ees pour diff\'erents exports 
(La derni\`ere occurence de |cs| est utilis\'ee pour tous les exports).

Si |cs| n'est pas d\'efinie dans le fichier de configuration, alors, la premi\`ere commande |cs| trouv\'ee par therion lors de la lecture des donn\'ees sera utilis\'ee pour les exports.

A l'int\'erieur d'un |layout|, cette option sp\'ecifie le syst\`eme de coordonn\'ees pour la localisation des donn\'ees associ\'ees (|origin|, 
|grid-origin|).

\subsubchapter `sketch-warp'.

\syntax
  \list
    * |sketch-warp <algorithm>| 
  \endlist
\endsyntax

Sp\'ecifie quel algorithme de morphing utiliser pour d\'eformer les scraps.
Plusieurs algorithmes sont possibles : |line|---par d\'efaut ; |plaquette|---invent\'e par Marco Corvi.

\subsubchapter `input'.

Fonctionne comme la commance |input| dans les fichiers de donn\'ees---Inclus d'autres fichiers

\subsubchapter `source'.

\description
   Sp\'ecifie les fichiers sources (donn\'ees) que Therion doit lire. 
   Vous pouvez sp\'ecifier plusieurs fichiers ici ; 
   un par ligne. Vous pouvez \'egalement les sp\'ecifier \`a l'aide de l'option de ligne de commande |-s| (voir ci-dessous).

   Il est \'egalement possible de taper (quelques petits extraits de code) directement dans le fichier de configuration en utilisant la syntaxe multiligne.
   
\enddescription

\syntax
  \hfill\break|source <nom_de_fichier>|\hfill\break
  or\hfill\break
  |source|\hfill\break
  {\qquad\it\dots therion commands\dots}\hfill\break
  |endsource|
\endsyntax

\arguments
*  |<nom_de_fichier>| 
\endarguments

\subsubchapter `select'.

\description
  S\'electionne les objets (relev\'es et dessins) \`a exporter. 
  Par d\'efaut, tous les objets du relev\'e sont s\'electionn\'es. 
  Si aucune topographie n'est s\'electionn\'ee, toutes les notes provenant des topographies s\'electionn\'ees sont s\'electionn\'ees par d\'efaut pour l'exportation de la topographie.

  S'il n'y a pas de scraps ou de topos dans les donn\'ees, la ligne de cheminement de toutes les topographies est export\'ee dans la topographie.\[ATTENTION : 
  s'il y a \'enorm\'ement de donn\'ees, ce qui est le cas, avec des cavit\'es ou des syst\`emes \`a fort d\'eveloppement, la m\'emoire est rapidement satur\'ee
  et le compilation crashe.]

  Lorsque vous exportez des topographies dans diff\'erentes projections, vous devez les s\'electionner s\'epar\'ement pour chaque projection
  
\enddescription

\syntax
  |select <object> [OPTIONS]|
\endsyntax

\arguments
*  |<object>| = tout relev\'e ou topographie identifi\'es par leur identifiant ID.
\endarguments

\options
  * |recursive <on/off>| = valide uniquement lorsqu'une topographie est s\'electionn\'ee. Si cette option est activ\'ee (par d\'efaut), 
                                       tous les sous-niveaux du relev\'e donn\'e sont s\'electionn\'es / d\'es\'electionn\'es de mani\`ere r\'ecursive.
  
  * |map-level <number>| = valide uniquement lorsqu'une topo est s\'electionn\'ee. 
                                           D\'etermine le niveau auquel l'extension de la topographie pour l'exportation d'atlas est arr\^et\'ee. 
                                           Par d\'efaut, 0 est utilis\'e. si `basic' est sp\'ecifi\'e, l�extension se fait jusqu�aux topographies de base. 
                                           {\it Remarque :} les aper\c{c}us de ne sont affich\'es que comme sp\'ecifi\'e dans les topographies du |map-level| en cours.

  * |chapter-level <number>| = valide uniquement si une topographie est s\'electionn\'ee. 
                                                D\'etermine le niveau (level) auquel, lors de l'exportation de l'atlas, l'extension de chapi-tre sera stopp\'ee. 
                                                Par d\'efaut, 0 est utilis\'e. Si vous sp\'ecifiez `|-|' ou `|.|', aucun chapitre n�est export\'e pour cette topographie. 
                                                Si l'option de pages de titre (|title-pages|) de la mise en page (|layout| ) est activ\'ee, chaque chapitre commence par une page de titre.
  
  \endoptions



\subsubchapter `unselect'.

\description
  D\'es\'electionne les objets \`a exporter.
\enddescription

\syntax
  |unselect <object> [OPTIONS]|
\endsyntax

\arguments
  Identique \`a la commande |select|.
\endarguments

\options
  Identique \`a la commande |select|.
\endoptions


\subsubchapter `text'.

\description
  Sp\'ecifie la traduction de tout texte Therion par d\'efaut dans le fichier de sortie.
\enddescription

\syntax
  |text <language ID> <texte therion> <mon texte>|
\endsyntax

\arguments
*  |<language ID>| = identifiant de langue ISO standard (par exemple |fr|, |en| ou |en_GB|)
*  |<therion text>| = texte Therion \`a traduire. Pour une liste des textes Therion et des traductions disponibles, voir le fichier |thlang/texts.txt|.
\endarguments


\subsubchapter `layout'.

\description
  Sp\'ecifie les caract\'eristiques de mise en page de l'export pour les topographies 2D. Les settings qui s'appliquent au mode atlas 
  sont marqu\'es  `A' ; au mode map `M'.
\enddescription

\syntax
|layout <id> [OPTIONS]
    copy <source layout id>
    cs <systeme de coordonnees>
    north <true/grid>
    scale <longueur sur l'image> <longueur reele>
    base-scale <longueur sur l'image> <longueur reele>
    units <metric/imperial>
    rotate <nombre>
    symbol-set <symbol-set>
    symbol-assign <point/line/area/group/special> <symbol-type> \ 
                                                  <symbol-set>
    symbol-hide <point/line/area/group/special> <symbol-type>
    symbol-show <point/line/area/group/special> <symbol-type>
    symbol-colour <point/line/area/group/special> <symbol-type> <colour>
    min-symbol-scale <scale>
    fonts-setup <tinysize> <smallsize> <normalsize> <largesize> \
                                                  <hugesize>
    size <width> <height> <units>
    overlap <valeur> <unites>
    page-setup <dimensions> <uniets>
    page-numbers <on/off>
    exclude-pages <on/off> <liste_des_pages>
    title-pages <on/off>
    nav-factor <facteur>
    nav-size <x-size> <y-size>
    transparency <on/off>
    opacity <valeur>
    surface <top/bottom/off>
    surface-opacity <valeur>
    sketches <on/off>
    layers <on/off>
    grid <off/top/bottom>
    grid-origin <x> <y> <x> <unites>
    grid-size <x> <y> <z> <unites>
    grid-coords <off/border/all>
    origin <x> <y> <z> <unites>
    origin-label <x-label> <y-label>
    own-pages <nombre>
    page-grid <on/off>
    legend <on/off/all>
    legend-columns <nombre>
    legend-width <n> <unites>
    map-comment <string>
    map-header <x> <y> <off/n/s/e/w/ne/nw/se/sw/center>
    map-header-bg <on/off>
    map-image <x> <y> <n/s/e/w/ne/nw/se/sw/center> <nom_de_fichier>
    statistics <explo/topo/carto/copyright all/off/nombre>
               <explo/topo-length on/off>
    scale-bar <longueur> <unites>
    survey-level <N/all>
    language <xx[_YY]>
    colour/color <item> <couleur>
    debug <on/all/first/second/scrap-names/station-names/off>
    doc-author <string>
    doc-keywords <string>
    doc-subject <string>
    doc-title <string>
    code <metapost/tex-map/tex-atlas>
    endcode
endlayout|
\endsyntax

\arguments
  |<id>| = identifiant du layout (utilis\'e par la commande |export|)
\endarguments

\comopt 
  * |copy <source layout id>| = d\'efinissez ici les propri\'et\'es qui ne sont pas modifi\'ees en fonction de la pr\'esentation source donn\'ee |layout|.
  
  {\it Options li\'ee \`a la pr\'esentation des topographies (map) :}\Nobreak

  * |scale <longueur de l'image> <longueur reele>| = d\'efini l'\'echelle de l'export de la topographie ou de l'atlas (M, A ; par d\'efaut : |1 200|)
  * |base-scale <longueur de l'image> <longueur reele>| = si c'est indiqu\'e, Therion va visuellement mettre \`a l'\'echelle la topographie par un facteur |scale/base-scale|.
                                                                                             Ceci poss\`ede le m\^eme effet que si la topographie imprim\'ee \`a l'\'echelle |base-scale| serait photo-r\'eduite \`a l'\'echalle |scale|. (M, A)
  * |rotate <valeur>| = effectue une rotation de la topographie de |<valeur>|$\circ$ (M, A ; par d\'efaut : |0|)
  * |units <metric/imperial>| = d\'efini les unit\'es de l'export (M, A ; par d\'efaut : |metric|)
  * |symbol-set <symbol-set>| = utilise la banque de symboles |symbol-set| pour tous les symboles de la topographie, s'ils sont disponibles.
                                                  Attention, le nom de la banque de symboles est casse sensitif. (M, A)
    
    Therion utilise les banques de symboles pr\'ed\'efinies suivantes :\par
    UIS (International Union of Speleology)\par
    ASF (Australian Speleological Federation)\par
    AUT (Austrian Speleological Association)\par
    CCNP (Carlsbad Caverns National Park)\NEW{5.4}\par
    NZSS (New Zealand Symbol Set)\NEW{5.4}\par
    SKBB (Speleoklub Bansk\'a Bystrica)
    
  * {\rightskip 0cm minus 4pt
    |symbol-assign <point/line/area/group/special> <symbol-type> <symbol-set>| = 
    affiche un symbole particulier dans la banque de symboles donn\'ee.
    Pour ce symbole, cette option pr\'edomine sur la d\'efinition de la banque de symboles d\'efinie par la commande
    |symbol-set|.\par}
    
    Si le symbole poss\`ede un sous-type, l'argument |<symbol-type>| doit \^etre sous une des formes suivantes :
    |type:subtype| ou simplement |type|, 
    qui assigne uns nouvelle banque de symboles \`a tous les sous-types du symbole en question.

    Les symboles suivant ne doivent pas \^etre utilis\'es avec cette option :
     point {\it section} (qui ne poss\`ede pas du tout de rendu sp\'ecifique) 
     et tout point et ligne d'\'etiquette ({\it label} (\'etiquette), 
    {\it remark} (remarque), {\it altitude}, {\it height} (hauteur), {\it passage-height} (hauteur du passage), 
    {\it station-name} (nom de station), {\it date}). 
    Voir le chapitre 
    {\it Modifier le layout / Personnaliser les \'etiquettes de texte} pour plus de d\'etails sur comment modifier l'apparence des \'etiquettes. (M, A)

    Group peut \^etre utilis\'e pour l'un de \'el\'ements suivants :
    all (tout), centerline (ligne de cheminement), sections, \NEW{5.3}water (eau),
    speleothems (sp\'el\'eoth\`emes), passage-fills (remplissage de passage/galerie), \NEW{5.4}ice (glace), sediments, equipment (\'equipement).
    
    Il existe deux symboles sp\'eciaux (special) : north-arrow (fl\`eche du nord), scale-bar (barre d'\'echelle).

  * |symbol-hide <point/line/area/group/special> <symbol-type>| = Ne pas afficher un symbole particulier ou un groupe de symboles sp\'ecifique.
    
    Vous pouvez utiliser |group cave-centerline|, |group surface-centerline|, 
    |point cave-station|, |point surface-station|
    et \NEW{5.4}|group text| dans les commandes |symbol-hide| et |symbol-show|.
    
    Utilisez |flag:<entrance/continuation/sink/spring/doline/dig>| comme 
    |<symbol-type>| pour masquer les stations topographiques avec un flag particulier
    (par exemple |symbol-hide point flag:entrance|).
    
    Cela peut aussi \^etre combin\'e avec |symbol-show|.(M, A)
  * |symbol-show <point/line/area/group/special> <symbol-type>| = affiche un symbole particulier ou un groupe de symboles s\'epcifique. 
                                                                                                          Peut \^etre combin\'e avec |symbol-hide|. (M, A)

  * \NEW{5.3}|symbol-colo[u]r <point/line/area/group/special> <symbol-type> <couleur>| = 
    modifie la couleur d'un symbole particulier ou d'un groupe de symboles sp\'ecifique.\[Note : la modification de la couleur s'applique actuellement au remplissage
    des pattern uniquement si (1) le format d'export est le PDF et (2)~si la version \MP\ est a minima 1.000] (M, A)

  * \NEW{5.4.1}|min-symbol-scale <scale>| = 
    d\'efini la |<scale>| minimale \`a partir de laquelle les points et les lignes sont affich\'ees sur la topographie. 
    Par exemple, avec |min-symbol-scale M|, aucun point ni line dont l'\'echelle est |S| ou |XS| ne sera affich\'e sur la topographie.
    |<scale>| poss\`ede le m\^eme format que l'option |scale| pour les points et les lignes.

  * \NEW{5.4.1}|fonts-setup <tinysize> <smallsize> <normalsize> <largesize> <hugesize>| =
    sp\'ecifie la taille du texte en points. 
    |<normalsize>| s'applique au point label (\'etiquette), 
    |<smallsize>| s'applique aux points remark (remarque) et autres points labels (\'etiquet-tes). 
    Chacun d'eux peut s'appliquer aux lignes label (\'etiquette) en accord avec son option |-size|.
    
    Les valeurs par d\'efaut sont respectivement :
    |8 10 12 16 24| pour les \'echelles jusqu'\`a 1:100; 
    |7 8 10 14 20| pour les \'echelles jusqu'\`a 1:200; 
    |6 7 8 10 14| pour les \'echelles jusqu'\`a 1:500 and 
    |5 6 7 8 10| pour les \'echelles plus petites que 1:500.

  {\it Options li\'ees au layout des pages :}\Nobreak

  * |size <largeur> <hauteur> <unites>| = d\'efini la taille de la topographie dans le mode atlas.
                                                          Si elle n'est pas sp\'ecifi\'ee, elle sera calcul\'ee \`a partir 
                                                          de |page-setup| et |overlap|.
                                                          Dans le mode topographie (map) cette option est appliqu\'ee si |page-grid| est d\'efini sur |on| (M, A ; par d\'efaut : |18 22.2 cm|)
  * |overlap <valeur> <unites>| = d\'efini la taille de recouvrement entre deux pages en unit\'e de page dans le mode Atlas.
                                                    D\'efini la taille de la marge dans le mode map (M, A ; par d\'efaut : |1 cm|)
  * |page-setup <dimensions> <unites>| = d\'efini les dimensions de la page dans cet ordre :
                                                                  largeur de la page (paper-width), hauteur de la page (paper-height), largeur de la page (page-width), hauteur de la page (page-height),
                                                                  marge gauche (left-margin) et  marge haute (top-margin). 
                                                                  Si ce n'est pas sp\'ecifi\'e, ce sera calcul\'e \`a partir de |size| et |overlap| (A ; par d\'efaut : |21 29.7 20 28.7 0.5 0.5 cm|)
  * |page-numbers <on/off>| = affiche ou non la num\'erotation des pages (A ; par d\'efaut : |true|)
  * |exclude-pages <on/off> <liste de pages>| = exclu des pages sp\'ecifiques de l'atlas.
                                                                            La liste contient les num\'eros de pages s\'epar\'es par une virgule ou un tiret (pour les intervalles). 
                                                                            Par exemple :~|2,4-7,9,23| signifie que les pages 2, 4, 5, 6, 7, 9 et 23 seront omises. Seules les pages de la topographie seront compt\'es 
                                                                            (d\'efinir |own-pages 0| et |title-pages off| pour obtenir le num\'ero correct de la page \`a exclure).
                                                                            La modification des options |own-pages| ou |title-pages| n'affecte pas l'exclusion des pages. (A)
  * |title-pages <on/off>| = affiche ou non le titre des pages avant chaque chapitre de l'atlas (A ; par d\'efaut : |off|)
  * |nav-factor <facteur>| = d\'efini le facteur de zoom du panneau navigation de l'atlas (A ; par d\'efaut : |30|)
  * |nav-size <x-size> <y-size>| = d\'efini le nombre de pages de l'atlas dans les deux directions du panneau de navigation (A ; par d\'efaut : |2 2|)
  * |transparency <on/off>| = d\'efini la transparence pour les galeries/passages (les galeries inf\'erieures sont alors visibles) (M, A ; par d\'efaut : |on|)
  * |opacity <valeur>| = d\'efini l'opacit\'e en \% set opacity value (valide si |transparency| est sur |on|). L'intervalle de valeur est 0--100. (M, A ; par d\'efaut : |70|)
  * |surface-opacity <valeur>| = d\'efini l'opacit\'e de la surface bitmap (valide si |transparency| est sur |on|).  L'intervalle de valeur est 0--100. (M, A ; par d\'efaut : |70|)
  * |surface <top/bottom/off>| = d\'efini la position de la surface bitmap au dessus / en dessous de la topographie. (M, A ; par d\'efaut : |off|)
  * |sketches <on/off>| = affiche ou non les dessins bitmaps scann\'es (morphing). (M, A ; par d\'efaut : |off|)
  * |layers <on/off>| = active/d\'esactive les couches PDF~1.5. (M, A ; par d\'efaut : |on|)
  * |grid <off/bottom/top>| = active/d\'esactive la grille (Les valeurs des coordonn\'ees peuvent \^etre optionnellement affich\'ees). (M, A ; par d\'efaut : |off|)
  * |cs <syst. de coordonnees>| = syst\`eme de coordonn\'ees pour |origin| et |grid-origin|
  * |north <true/grid>| = sp\'ecifie l'orientation par d\'efaut de la topographie (map).
                                     Par d\'efaut, la grille astronomique du nord `vrai' est utilis\'e. 
                                     Elle est ignor\'ee dans le cas d'utilisation du syst\`eme de coordonn\'ees locales.
  * |grid-origin <x> <y> <x> <unites>| = d\'efini les coordonn\'ees d'origine de la grille (M, A)
  * |grid-size <x> <y> <z> <unites>| = d\'efini la taille de la grille en unit\'es r\'eelles (M, A ; est \'egal par d\'efaut \`a la taille de la barre d'\'echelle |scale-bar|)
  * |grid-coords <off/border/all>| = sp\'ecifie o\`u ajouter les \'etiquettes de coordonn\'ees \`a la grille (M, A; par d\'efaut : |off|)
  * |origin <x> <y> <z> <unites>| = sp\'ecifie l'origine des pages de l'atlas (M, A)
  * |origin-label <x-label> <y-label>| = sp\'ecifie l'\'etiquette pour la page d'atlas qui poss\`ede le coin en bas \`a gauche comme coordonn\'ees d'origine donn\'ees.
                                                           Peut \^etre un nombre ou un charact\`ere. (M, A ; par d\'efaut : |0 0|)
  * |own-pages <number>| = d\'efini le nombre de pages personnelles \`a ajouter avant la premi\`ere page g\'en\'er\'ee automatiquement en mode atlas
                                             (utilis\'e actuellement pour corriger la num\'erotation des pages) (A ; par d\'efaut : |0|)
  * |page-grid <on/off>| = affiche le plan des pages (M ; par d\'efaut : |off|)

  {\it Options li\'ees \`a la l\'egende de la topographie (map) :}\Nobreak

  * |map-header <x> <y> <off/n/s/e/w/ne/nw/se/sw/center>| = 
    imprimme le cartou-che (header) de la topographie \`a une localisation sp\'ecifi\'ee par |<x> <y>|. 
    Le cartouche de la topographie contient des informations basiques a propos de la cavit\'e :
    nom (name), \'echelle (scale), fl\`eche du nord (north arrow), liste des topographes (list of surveyors) etc. 
    Il est enti\`erement personnalisable (voir le chapitre {\it Modifier le layout} pour plus de d\'etails).
    |<x>| est l'abcisse (gauche/droite ou E--W sur la page). 
    |<y>| est l'ordonn\'ee (haut/bas ou N--S sur la page).
    Les intervalles pour  |<x>| et |<y>| sont -100--200. 
    Le coin inf\'erieur-gauche de la topographie est |0 0|, 
    le coin sup\'erieur- droit est |100 100|. 
    Le cartouche est align\'e avec le coin sp\'ecifi\'e comme point d'ancrage |<off/n/s/e/w/ne/nw/se/sw/center>| 
    (M ; par d\'efaut : |0 100 nw|)
  * |map-header-bg <on/off>| = lorsque l'option est sur |on|, l'arri\`ere-plan du cartouche est rempli avec la couleur d'arri\`ere plan 
                                                 (par exemple pour cacher une grille topographi-que). (M ; par d\'efaut : |off|)
  * |map-image <x> <y> <n/s/e/w/ne/nw/se/sw/center> <nom_de_fichier>| = 
    inclus \- l'image\[Notez que vous pouvez inclure aussi une PDF, ce qui peut \^etre utilis\'e pour combiner plan et coupe d\'evelopp\'ee sur un m\^eme PDF avec un tr\`es bon rendu] 
    sp\'ecifi\'ee par |<nom_de_fichier>| sur la topographie \`a la localisation donn\'ee par |<x> <y>|. 
    Pour des d\'etails sur les coordonn\'ees et l'alignement, voir la description de l'option |map-header|.
  * |legend-width <n> <unites>| = largeur de la l\'egende (M, A ; par d\'efaut : |14 cm|)
  * |legend <on/off/all>| = affiche dans le cartouche la liste des symboles utilis\'es sur la topographie.
                                        Si l'option est |all|, tous les symboles de la banque de symboles utilis\'ee seront affich\'es. (M, A ; par d\'efaut : |off|)
  * |colo[u]r-legend <on/off>| = affiche ou non la couleur de fond des symboles dans la l\'egende lorsque l'option |map-fg| est donn\'ee pour l'altitude, le scrap o\`u la map (M, A)
  * |legend-columns <number>| = ajuste le nombre de colonnes pour la l\'egende (M, A ; par d\'efaut : |2|)
  * |map-comment <string>| = rajouter un commentaire optionnel au cartouche (M)
  * |statistics <explo/topo/carto/copyright all/off/number>| ou 
  * |statistics <explo/topo-length on/hide/off>| = affiche des statistiques basiques ; 
                                                                           si la valeur est |off|, les membres sont class\'es par \NEW{5.4}ordre alphab\'etique ;
                                                                           sinon, ils sont class\'es en fonction de leur contribution (= longueur) \`a l'exploration et/ou la topographie (M, A ; par d\'efaut : |off|)
  * |scale-bar <length> <units>| = d\'efini la longueur de la barre d'\'echelle (M, A)
  * |language <xx[_YY]>| = d\'efini la langue de l'export.
                                          Les langues disponibles sont list\'ees sur la page de licence / copyright. Voir les {\it Annexes} si vous d\'esirez
                                          ajouter ou personnaliser des traductions. (M, A)
  * |colo[u]r <item> <colour>| = personnalise la couleur d'un objet (item) sp\'ecifique sur la topographie (map-fg (avant-plan), map-bg (arri\`ere-plan), preview-above (au-dessus), preview-below (en-dessous), label). 
                                                 L'intervalle de couleur est 0--100 pour l'\'echelle de gris, et sous forme d'un triplet [0--100 0--100 0--100] pour l'\'echelle couleur RGB. 
    
    Pour |map-fg|, vous pouvez utilisez |altitude|, |scrap| ou |map| comme couleurs. 
    In this case the map is coloured according to altitude, scraps or maps.
    
    Pour |map-bg|, vous pouvez utiliser le mot-clef |transparent| pour omettre compl\`etement l'arri\`ere-plan de la page. 
    
    Pour les \'etiquettes (labels), vous pouvez mettre la couleur sur |on/off|. Si c'est sur |on|, les \'etiquettes sont coloris\'ees en utilisant la couleur du scrap associ\'e.
    
  * |debug <on/all/first/second/scrap-names/station-names/off>| = 
    dessine en uti-lisant diff\'erentes couleurs les scraps \`a diff\'erentes \'etapes de la transformation. 
    Cela permet de visualiser comment Therion distort le dessin.
    Voir la description de la commande |scrap| pour plus de d\'etails.
    Les points dont la distance est la plus modifi\'ee pendant la transformation sont affich\'es en orange.
    si |scrap-names| est sp\'ecifi\'e, alors le nom des scraps est affich\'e pour chaque scrap. 
    |station-names| affiche le nom de chaque station topographique.
  * |survey-level <N/all>| = |N| est le nombre de niveau de topographies \`a afficher pour le nom de la station topographique (M, A ; par d\'efaut : |0|).

  {\it Option li\'ees \`a l'export PDF :}\Nobreak

  * |doc-author <string>| = d\'efini l'auteur du document (M, A)
  * |doc-keywords <string>| = d\'efini les mots-clefs du document (M, A)
  * |doc-subject <string>| = d\'efini le sujet du document (M, A)
  * |doc-title <string>| = d\'efini le titre du document (M, A)

  {\it Personnalisation :}\Nobreak

  * |code <metapost/tex-map/tex-atlas>| = Ajoute ou red\'efini ici des macros \TeX\ et \MP.
                                                                   Cela permet \`a l'utilisateur de configurer de nombreux objets comme les symboles d\'efinis par l'utilisateur,
                                                                   o\`u le layout de la map (topographie) et de l'atlas au m\^eme endroit \&c. 
                                                                   Voir le chapitre {\it Modifier le layout} pour plus de d\'etails.
  * |endcode| = doit terminer un bloc de commandes \TeX\ et/ou \MP\
\endcomopt

\midinsert
%  \ifx\pdfoutput\undefined\else
%    \pdfximage {pic/page.pdf}%
%  \fi
%  \vbox to 482bp{\centerline{\hbox to 400bp{%
%    \ifx\pdfoutput\undefined
%      \epsfbox{mp/page.1}%
%    \else
%      \rlap{\pdfrefximage\pdflastximage}%
%      \convertMPtoPDF{mp/page.1}{1}{1}
%    \fi
%    \hss}}\vss
%  }
  \vbox to 482bp{\centerline{\hbox to 400bp{%
      \rlap{\pic{page.pdf}}%
      \MPpic{page.1}
    \hss}}\vss
  }
\endinsert


\subsubchapter `setup3d'.

\syntax
  \list
    * |setup3d <valeur>| 
  \endlist
\endsyntax

\NEW{5.3}Hack temporaire pour d\'efinir la distance d'\'echantillonnage (en m\`etres) 
lorsque nous g\'en\'erons un mod\`ele 3D par morceaux lin\'eraire \`a partir d'une enveloppe construite \`a partir de courbe de B\'ezier.


\subsubchapter `sketch-colors'.

\syntax
  \list
    * |sketch-colors <nombre-de-couleurs>| 
  \endlist
\endsyntax

\NEW{5.4} Cette option peut \^etre utilis\'ee pour r\'eduire la taille des images/dessins bitmaps pour les topographies (lorsque affich\'es)


\subsubchapter `export'.

\description
  Exporte les topographies s\'electionn\'ees (surveys ou maps). 
\enddescription

\syntax
  \list
    * |export <type> [OPTIONS]| 
  \endlist
\endsyntax

\arguments
  * |<type>| = Les exports suivants sont support\'es :

    |model| = mod\`ele 3D de la cavit\'e

    |map| = topographie en 2D sur une page uniquement

    |atlas| = atlas topographique en 2D sur plusieurs pages
    
    |cave-list| = tableau de synth\`ese des cavit\'es
    
    |survey-list| = tableau de synth\`ese des topographies (survey)
    
    |continuation-list| = liste des continuation possibles

    |database| = base de donn\'ees SQL avec la ligne de cheminement (centreline)
\endarguments

\penalty0

\options
  {\it commun :}\Nobreak
  * |encoding/enc <encoding>| = d\'efini l'encodage de l'export
  * |output/o <file>| = d\'efini le nom du fichier export\'e (avec l'extension). Si aucun nom n'est sp\'ecifi\'e, le pr\'efixe ``|cave.|'' est utilis\'e avec l'extension correspondante au format d'export.
    
    Si le nom de fichier d'export est donn\'e et qu'aucun format d'export n'est sp\'ecifi\'e, 
    le format d'export sera d\'etermin\'e par l'extension du nom de fichier.

  {\it model  (mod\`ele):}\Nobreak

  * |format/fmt <format>| = d\'efini le format d'export. Actuellement, les formats suivants sont accept\'es : |loch| (format natif ; fichier .loch ; par d\'efaut),
    |compass| (fichier .plt), |survex| (fichier .3d), |dxf|,
    |esri| (shapefiles 3D), |vrml|, |3dmf| et |kml| (Google Earth).
  * |enable <walls/[cave/surface-]centerline/splay-shots/surface/all>| et
  * |disable <walls/[cave/surface-]centerline/splay-shots/surface/all>| = 
     \hfil\break
     s\'electionne les caract\'eristiques \`a exporter, si le format le supporte. 
     |surface| n'est actuellement exporter qu'au format |Therion|.
  * |wall-source <maps/centerline/all>| = d\'efini la donn\'ee source \`a utiliser pour la mod\'elisation des parois 3D.

  {\it map/atlas :}\Nobreak
  
  * |format/fmt <format>| = d\'efini le format de la map (topographie). Actuellement |pdf|, |svg|, 
    |xhtml|\[SVG int\'egr\'e au fichier XHTML qui contient aussi la l\'egende],
    |survex|, |dxf|, |esri|\[shapefiles ESRI. Les diff\'erents fichiers du shapefile sont \'ecrits dans un r\'epertoire nomm\'e par le nom\_de\_fichier sp\'ecifi\'e.],
    |kml| (Google Earth), 
    |xvi|\[image vecteur Xtherion. Les images XVI peuvent \^etre utilis\'ees dans
    XTherion pour dessiner des topographies \`a l'\'echelle. |scale| (la r\'esolution d'image 100\,DPI est
    suppos\'ee) et |grid-size| du layout sont utilis\'es pour l'export.] et
    \NEW{5.3}|bbox|\[fichier texte contenant les coordonn\'ees g\'eographiques des coins inf\'erieur-gauche et sup\'erieur-droit de la topographie.]
    pour map (topographie) ; |pdf| uniquement est support\'e pour l'atlas.
  * |projection <id>| = identifiant unique qui sp\'ecifie le type de projection de la map (topographie).
    (Voir la commande |scrap| pour les d\'etails.) 
    
    Si aucune map n'est d\'efinie, alors les scraps de la projection donn\'ee seront export\'es.
    
    S'il n'y a pas de scrap avec la projection sp\'ecifi\'ee, alors Therion exportera la ligne de cheminement (centerline) des topographies (surveys) s\'electionn\'ees
  * |layout <id>| = d\'efini le layout pr\'ed\'efini pour la map ou l'altlas.
  * |layout-xxx| = o\`u |xxx| permet de rajouter des options de layout afin de modifier certaines propri\'et\'es dans la commande export.
  * |encoding/enc <encoding>| = defini l'encodage de l'export

  {\it communes pour les listes (list) :}\Nobreak

  * |format/fmt <format>| = d\'efini le format d'export pour continuation. Actuellement, les formats de sorties suivants sont disponibles :
     |html| (par d\'efaut), |txt|,
    \NEW{5.4}|kml|\[Pour cave-list et continuation-list.] et |dbf|.
    
  {\it continuation-list :}\Nobreak

  * |attributes <(on)/off>| = d\'efini si les attributs d\'efinis par l'utilisateur dans le tableau de continuations sont export\'es ou non
  * \NEW{5.3}|filter <(on)/off>| = d\'efini si les continuations sans commentaires/texte doivent \^etre filtr\'ees ou non.

  {\it cave-list :}\Nobreak

  * |location <on/(off)>| = d\'efini si les coordonn\'ees des entr\'ees doivent \^etre export\'ees ou non. 
  * \NEW{5.3}|surveys (on)/off| = exporte une liste brute des cavit\'es lorsque l'option est sur |off|. 
                                                    Sinon, toute la structure de la survey est agr\'eg\'ee avec les statistiques.
    
  {\it database :}\Nobreak
  
  * |format/fmt <format>| = actuellement |sql| et |csv|
  * |encoding/enc <encoding>| = d\'efini l'encodage d'export
\endoptions

{\it Synth\`ese des formats de fichiers export\'es :}
\nobreak\medskip
\bgroup
\leavevmode\kern1em\vbox{\advance\hsize-1em\halign{#\hfil\quad&#\hfil\cr
{\it type d'export}&{\it formats disponibles}\cr\noalign{\smallskip\hrule\smallskip}
model & loch, dxf, esri, compass, survex, vrml, 3dmf, kml\cr
map  & pdf, svg, xhtml, dxf, esri, survex, xvi, kml, \NEW{5.3}bbox\cr
atlas & pdf\cr
database & sql, \NEW{5.4}csv\cr
lists & html, txt, kml, dbf\cr}}\egroup


\subchapter Faire fonctionner Therion.

Maintenant, apr\`es avoir bien compris la structure des fichiers de donn\'ees et de configuration, nous sommes pr\^ets \`a utilsier Therion.
Habituellement, ceci est effectu\'e via les commandes en ligne en se positionnant dans le dossier de donn\'ees en tapant :

|therion|

La syntaxe compl\`ete est :

|therion [-q] [-L] [-l <log-file>]
        [-s <fichier_source>] [-p <chemin_de_recherche>]
        [-b/--bezier]
        [-d] [-x] [--use-extern-libs] [<cfg-file>]|

ou :

|therion [-h/--help]
        [-v/--version]
        [--print-encodings]
        [--print-environment]
        [--print-init-file]
        [--print-library-src]
        [--print-symbols]
        [--print-tex-encodings]
        [--print-xtherion-src]|


\penalty-200
\arguments
  |<cfg-file>| 
  Therion ne prend qu'un seul argument optionnel : le nom du fichier de configuration.
  Si aucun nom de fichier de configuration n'est fourni, le |thconfig| du r\'epertoire courant est utilis\'e.
  S'il n'y a pas de fichier |thconfig| (par exemple, le dossier courant n'est pas le dossier de donn\'ees)
  Therion sort et affiche un message d'erreur.
\endarguments

\options
* |-d| =
  force le mode d\'ebogage. 
  L'impl\'ementation actuelle cr\'ee un dossier temporaire nomm\'e |thTMPDIR| 
  (dans votre dossier temporaire syst\`eme)
   et n'efface pas les fichiers temporaires. 

%* |-g| =
%  Using this option you can generate a new configuration file.
%  If |cfg-file| is not specified therion will use the |thconfig|
%  file. If the destination file exists, it'll be overwritten.
%  Generate a new configuration file. This will be the given 
%  |<cfg-file>| if specified, or |thconfig| in the current directory if not. 
%  If the file already exists, it will be overwritten.
%        
* |-h, --help| =
        Affiche une aide rapide.

%* |-i| =
%        Ignore comments when writing |-g| or |-u| configuration file.
%
* |-L| =
        Ne cr\'ee par de fichier log. 
        Normalement, Therion Normally therion \'ecrit tous les messages dans un fichier
        therion.log dans le r\'epertoire courant.
        
* |-l <log-file>| =
        Change le nom du fichier log.
        
* |-p <search-path>| =
        Cette option est utilis\'ee pour d\'efinir le chemin de recherche (path)
        (ou une liste de paths s\'epar\'es par des deux-points) que Therion utilise pour trouver les fichiers sources
        (s'il ne les trouve pas dans le dossier de travail).

* |-q| =
        Lancer Therion en mode silencieux. Il affichera les warnings et les messages d'erreur dans STDERR.

* |--print-encodings| =
        Imprime une liste des encodages support\'es.
        
* |--print-tex-encodings| =
        Imprime une liste des encodages support\'es pour l'export PDF.
        
* |--print-init-file| =
        Imprime un fichier d'initialisation par d\'efaut. Pour plus de d\'etails, voir la section
        {\it Initialisation} dans les {\it Annexes}.

* |--print-environment| =
        Imprime les caract\'eristiques de l'environnement Therion.

* |--print-symbols| =
        Imprime une liste de tous les symboles support\'es par Therion dans un fichier |symbols.xhtml|.
        
* |-s <source-file>| =
        D\'efini le nom du fichier source.
        
%* |-u| =
%        Upgrade the configuration file.
%
* |--use-extern-libs| =
  	Ne copie pas les macros \TeX\ et \MP\ dans le dossier de travail \TeX\ et \MP\ 
  	doivent chercher les macros comme bon leur semble. A utiliser avec pr\'ecaution.

* |-v, --version| =
  	Afficher les informations de version.        
        
* |-x| =
  G\'en\'erer le fichier `.xtherion.dat' avec des informations additionnelles pour XTherion.
\endoptions


\subsubchapter XTherion---compiler.

XTherion rend l'utilisation de Therion plus facile, sp\'ecialement sur les syst\`emes sans consoles de lignes de commandes.
La fen\^etre de compilation est la fen\^etre par d\'efaut de XTherion.
Pour lancer Therion, il suffit d'ouvrir un fichier de configuration et de presser la touche `F9' ou le bouton `Compile'. 

XTherion affiche les messages de Therion dans la partie inf\'erieure de l'\'ecran.
Chaque message d'erreur est surlign\'e et un lien hypertexte vers le fichier source o\`u l'erreur a eu lieu est cr\'e\'e.

Apr\`es une premi\`ere compilation, des menus additionnels sont activ\'es, comme la {\it structure des Surveys} 
({\it Survey structure}) et la {\it structure des Maps} ({\it Map structure}). 
L'utilisateur peut s\'electionner confortablement des surveys ou des maps pour l'exportation en double-cliquant sur les objets dans l'arbre.
Un clic simple dans l'arbre de la {\it Survey structure} affichera des informations basiques sur la
survey dans le menu {\it Survey info}.

\endinput

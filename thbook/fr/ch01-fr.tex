\eject \ifodd\pageno\else\hbox{}\eject\fi

{\baselineskip=12pt
\setbox1=\hbox{\eightit NE LAISSEZ PERSONNE IGNORANT LA GEOMETRIE ENTRER ICI}
\copy1
\vskip4pt
\line{\pic[\wd1]{ageom.pdf}\hss}
\hbox to \wd1{\hfil\eightrm ---Inscription sur l'entr\'ee de l�Acad\'emie de Platon, $IV^e$ si\`ecle av. J.C.}
%\hbox to \wd1{\hfil\eightrm of Plato's Academy, {\eightmit 4}th century BC}
}

\chapter Introduction.

Therion est un outil de topographie souterraine. Son but est d'aider \`a :

\list
* archiver les donn\'ees topographi\'ees sur ordinateur sous une forme aussi proche que possible des notes et des esquisses d'origine, 
et de les r\'ecup\'erer de mani\`ere flexible et efficace ;
* dessiner et actualiser de beaux plans et coupes ;
* cr\'eer un mod\`ele 3D r\'ealiste de la grotte.
\endlist

Il fonctionne avec les syst\`emes d'exploitation Unix, Linux, MacOS\,X et Win32. Le code source et le programme d�installation de Windows sont disponibles sur la page Internet de Therion 
(\www{https://therion.speleo.sk}).

Therion est distribu\'e sous licence \www[GNU General Public License]
{https://www.gnu.org/}.

\subchapter Pourquoi Therion ?.

Dans les ann\'ees 90, nous avons beaucoup pratiqu\'e la sp\'el\'eologie et la topographie des grottes. 
Il existait certains programmes informatiques affichant des vues de cheminement et des stations apr\`es la fermeture de la boucle et l'\'elimination des erreurs. 
Ceux-ci ont \'et\'e d'une grande aide, en particulier pour les syst\`emes de cavit\'es de grande taille et complexes. 
Nous avons utilis\'e l'un d'eux, TJIKPR, comme couche de fond avec le calque des stations pour dessiner ensuite des topographies \`a la main. 
Apr\`es avoir achev\'e un vaste atlas de la � Grotte des chauves-souris � de 166 pages au d\'ebut de 1997, nous avons rapidement eu un probl\`eme : 
nous avons trouv\'e de nouveaux passages reliant des passages connus et les avons explor\'es. Apr\`es traitement dans TJIKPR, les nouvelles boucles ont influenc\'e la position des anciens relev\'es. 
La plupart des stations du relev\'e avaient maintenant une position l\'eg\`erement diff\'erente d�avant en raison de la modification de la r\'epartition des erreurs. 
Ainsi, nous devions soit dessiner \`a nouveau tout l'Atlas, soit accepter que l'emplacement de certains endroits ne soit pas pr\'ecis 
(dans le cas de boucles d'une longueur d'environ 1 km, il y avait parfois des erreurs d'environ 10 m) et d'essayer de fausser les nouveaux passages pour les adapter aux anciens.

Ces probl\`emes ont persist\'e lorsque nous avons essay\'e de dessiner des topographies \`a l'aide de certains programmes de CAO en 1998 et 1999. 
Il \'etait toujours difficile d'ajouter de nouveaux relev\'es sans adapter les anciens aux positions nouvellement calcul\'ees des stations topo dans toute la grotte. 
Nous n�avons trouv\'e aucun programme capable de dessiner une topo complexe correctement mise \`a jour (c�est-\`a-dire pas seulement des vues avec une enveloppe LRUD), 
dans laquelle les anciennes parties soient modifi\'ees en fonction des coordonn\'ees connues des stations du relev\'e le plus r\'ecent.

En 1999, nous avons commenc\'e \`a r\'efl\'echir \`a la cr\'eation d'un programme propre pour le dessin de topographies. 
Nous connaissions des programmes parfaitement adapt\'es \`a des sous-t\^aches particuli\`eres. Il y avait \MP,
un langage de programmation de haut niveau pour la description de graphiques vectoriels, 
Survex pour un excellent traitement des plans, et \TeX\ pour la compilation des r\'esultats. 
Nous n'avions qu'\`a les rassembler. \`a No�l 1999, une version minimaliste de Therion fonctionnait pour la premi\`ere fois. 
Cela ne comprenait que 32 kB de scripts Perl et de macros \MP, mais visait \`a d\'emontrer que nos id\'ees pouvaient \^etre mises en {\oe}uvre.

En 2000-2001, nous avons recherch\'e le format optimal des donn\'ees d�entr\'ee, 
le langage de programmation, le concept d�\'editeur de dessins interactif et des algorithmes internes avec l�aide de Martin Sluka (Prague) et de Martin Heller (Suisse). 
En 2002, nous avons pu sortir  la premi\`ere version de Therion r\'eellement utilisable, qui r\'epondait \`a nos exigences.


\subchapter Caract\'eristiques.

Therion est une application en ligne de commande. Il traite les fichiers d'entr\'ee, 
y compris les topos 2D, au format texte, et cr\'ee des fichiers avec des topos 2D ou un mod\`ele 3D en sortie.

La syntaxe des fichiers d'entr\'ee est d\'ecrite en d\'etail dans les chapitres suivants. 
Vous pouvez cr\'eer ces fichiers dans un \'editeur de texte brut quelconque tel que {\it ed} ou {\it vi}. Ils contiennent des instructions pour Therion comme :

|point 1303 1004 pillar| 

o\`u {\tt point} est un mot-cl\'e pour le symbole de point, 
suivi de ses coordonn\'ees et d'une sp\'ecification de type de symbole \`a afficher (ici un pilier).

L�\'edition manuelle de tels fichiers n�est pas facile - en particulier lorsque vous tracez des topos, 
devez penser en termes spatiaux (coordonn\'ees cart\'esiennes). Il existe donc une interface graphique sp\'eciale pour Therion appel\'ee XTherion. 
XTherion fonctionne comme un \'editeur de texte avanc\'e, un \'editeur de topos (o\`u les images sont dessin\'ees de mani\`ere totalement interactive) 
et un compilateur (qui ex\'ecute Therion sur les donn\'ees).

Cela peut para�tre compliqu\'e, mais cette approche pr\'esente de nombreux avantages :

\list
* Il existe une s\'eparation stricte des donn\'ees et de la visualisation. 
  Les fichiers de donn\'ees sp\'ecifient uniquement ce qui est o\`u, pas ce \`a quoi cela ressemble. 
  \MP ajoute la repr\'esentation visuelle dans les phases ult\'erieures du traitement des donn\'ees. 
  (Ce concept est tr\`es similaire \`a la repr\'esentation de donn\'ees XML.)

  Cela permet de modifier les symboles de dessin utilis\'es sans modifier les donn\'ees d'entr\'ee 
  ou de fusionner plusieurs dessins cr\'e\'es par diff\'erentes personnes ayant des styles diff\'erents 
  dans un seul dessin avec un jeu de symboles unifi\'e.

  Les dessins en 2D sont adapt\'es \`a des \'echelles de sorties sp\'ecifiques (niveau d'abstraction, 
  mise \`a l'\'echelle non lin\'eaire des symboles et des textes)
  
* Toutes les donn\'ees sont relatives aux positions des stations de la topographie. 
   les coordonn\'ees des stations de la topographie sont modifi\'ees au cours du processus de fermeture de la boucle, 
   toutes les donn\'ees pertinentes sont d\'eplac\'ees en cons\'equence, de sorte que la topographie est toujours \`a jour.

* Therion est ind\'ependant des syst\`emes d'exploitation, d'un encodage de caract\`eres ou d'un \'editeur de fichiers d'entr\'ee particulier ; 
  les fichiers d'entr\'ee resteront lisibles par un \^etre humain.

* Il est possible d�ajouter de nouveaux formats de sortie.

* Le mod\`ele 3D est g\'en\'er\'e \`a partir des images 2D pour obtenir un mod\`ele 3D r\'ealiste sans entrer trop de donn\'ees.

* bien que le support pour WYSIWYG soit limit\'e, vous obtenez ce que vous voulez.
\endlist

\subchapter Logiciels requis.

``Un programme doit faire une seule chose, mais le faire bien." (Ken Thompson) 
Nous utilisons donc des programmes externes utiles, li\'es aux probl\`emes de composition et de visualisation des donn\'ees. 
Therion peut alors s�acquitter de sa t\^ache beaucoup mieux que s�il s�agissait d�une application autonome 
permettant de calibrer votre imprimante ou votre scanner et d�envoyer un courrier \'electronique avec vos donn\'ees d�un seul clic.

Therion a besoin de :
\list
* Une distribution \TeX. N\'ecessaire uniquement si vous souhaitez cr\'eer des dessins 2D au format PDF ou SVG.
* {\it Tcl / Tk} avec {\it BWidget} et \'eventuellement l'extension tkImg. Ce n'est n\'ecessaire que pour XTherion.
* Des outils typographiques LCDF ({\it LCDF Typetools}) si vous souhaitez utiliser une configuration facile pour les polices personnalis\'ees dans les dessins PDF. 
* {\it Convertissez} et {\it identifiez} les utilitaires \`a partir de la distribution ImageMagick, si vous souhaitez utiliser la d\'eformation des esquisses de topographie.
* {ghostscript} pour cr\'eer des images calibr\'ees \`a partir de fichiers PDF g\'eor\'ef\'erenc\'es.
\endlist

Le programme d'installation Windows inclut tous les packages requis, \`a l'exception de {ghostscript}. Lisez l'{\it Annexe} si vous voulez compiler vous-m\^eme Therion.

Pour afficher des dessins et des mod\`eles, vous pouvez utiliser l'un des programmes suivants~:
\list
* tout visualisateur PDF ou SVG affichant des images 2D ;
* tout SIG prenant en charge les formats DXF ou {\it shapefile} pour l'analyse des dessins ;
* une  visionneuse 3D appropri\'ee pour les mod\`eles export\'es dans un format autre que le format par d\'efaut;
* tout client de base de donn\'ees SQL pour traiter la base de donn\'ees export\'ee.
\endlist


\subchapter Installation.

{\bf Installation \`a partir des sources (package therion-5.*.tar.gz):}

Le code source est une distribution primaire de Therion. Il doit \^etre compil\'e et install\'e conform\'ement aux instructions de l'{\it annexe}.

{\bf Installation pour Windows:}

Ex\'ecutez le programme d'installation et suivez les instructions. 

Il installe tous les \'el\'ements requis et cr\'ee des raccourcis vers XTherion et Therion Book.

\subsubchapter Environnement de travail.

Therion lit les param\`etres du fichier d'initialisation. 
Les param\`etres par d\'efaut devraient fonctionner correctement pour les utilisateurs utilisant uniquement les caract\`eres 
latins\[Sur le fichier PDF, Therion restitue la plupart des caract\`eres accentu\'es sous la forme d�une combinaison 
d�accents et d�un caract\`ere de base. Certains accents \'etranges peuvent \^etre omis. 
Des lettres accentu\'ees pr\'ecompos\'ees sont incluses pour les langues slovaque et tch\`eque.], \TeX\ standard et \MP.

Si vous souhaitez utiliser vos propres polices pour les caract\`eres latins ou non latins dans les fichiers PDF, 
modifiez le fichier d'initialisation. Des instructions sur la mani\`ere de proc\'eder sont donn\'ees en {\it annexe}.


\subchapter Comment cela fonctionne-t-il ?.

Alors, maintenant que les besoins de Therion sont clairs, voyons comment il interagit avec tous ces programmes :

\par\penalty-200
\midinsert
\centerline{\MPpic{schema.1}}
\endinsert

PAS DE PANIQUE ! 
Lorsque votre syst\`eme est correctement configur\'e, la plus grande partie est cach\'ee \`a l'utilisateur 
et tous les programmes n\'ecessaires sont automatiquement ex\'ecut\'es par Therion de fa\c{c}on transparente.

Pour travailler avec Therion, il suffit de savoir que vous devez cr\'eer des donn\'ees d'entr\'ee 
(la meilleure m\'ethode \'etant avec XTherion), puis ex\'ecuter Therion et afficher les fichiers de sortie 
(mod\`ele 3D, dessins, fichier journal) dans le programme appropri\'e. 

Pour en savoir plus \`a ce sujet, voici une br\`eve explication de l'organigramme 
ci-dessus. 
Les noms de programme sont en police romaine et les fichiers de donn\'ees en italique. 
Les fl\`eches indiquent le flux de donn\'ees entre les programmes. 
Les fichiers de donn\'ees temporaires ne sont pas affich\'es. 
Signification des couleurs : 

\list
* noir---Programmes Therion et macros (XTherion est \'ecrit Tcl/Tk,
  il a donc besoin de cet interpr\'eteur pour fonctionner)
* rouge---package \TeX\
* vert---fichiers rentrants cr\'e\'es par l�utilisateur et fichiers sortants cr\'e\'es par Therion
\endlist

Therion effectue lui-m\^eme la t\^ache principale. 
Il lit les fichiers d'entr\'ee, les interpr\`ete, trouve les boucles ferm\'ees et distribue les erreurs. 
Ensuite, toutes les autres donn\'ees (par exemple, dessins  2D) sont transform\'ees en fonction de la nouvelle position de la station.
Therion exporte les donn\'ees des dessins 2D au format \MP. 
\MP donne la forme r\'eelle aux symboles de carte abstraits en fonction des d\'efinitions de symbole de carte ; 
cela cr\'ee beaucoup de fichiers PostScript avec de petits scraps de la grotte. 
Ceux-ci sont lus et convertis dans un format de type PDF, qui forme les donn\'ees d'entr\'ee pour pdf\TeX\. 
Pdf\TeX\ effectue ensuite toute la composition et cr\'ee le fichier PDF final de la topographie de la grotte.

Therion exporte aussi le mod\`ele 3D (complet ou ligne de cheminement) dans diff\'erents formats.

Le cheminement peut \^etre export\'e pour un traitement ult\'erieur dans n�importe quelle base de donn\'ees SQL.


\subchapter Premi\`ere utilisation.

Apr\`es avoir expliqu\'e les principes de base de Therion, il est judicieux de l�essayer sur des exemples de donn\'ees.

\list
* T\'el\'echargez les exemples de donn\'ees \`a partir de la page Web Therion et d\'ecompressez-les quelque part sur le disque dur de votre ordinateur.
* Ex\'ecutez XTherion (sous Unix et MacOS\,X en tapant `xTherion' dans la ligne de commande, vous trouverez sous Windows un raccourci dans le menu {\it D\'emarrer}).
* Ouvrez le fichier `thconfig' \`a partir du r\'epertoire de donn\'ees de la fen\^etre `Compiler' de XTherion.
* Appuyez sur `F9' ou `compile' dans le menu pour ex\'ecuter Therion sur les donn\'ees---Vous obtiendrez des messages de Therion, \MP et \TeX\.
* Les fichiers PDF et le mod\`ele 3D sont cr\'e\'es dans le r\'epertoire de donn\'ees. 
\endlist

De plus, vous pouvez ouvrir des fichiers de donn\'ees topo (*.th) dans la fen\^etre `Editeur de texte' 
et des fichiers de donn\'ees cartographiques (*.th2) dans la fen\^etre `Editeur de carte' de XTherion. 
Bien que le format des donn\'ees puisse sembler d\'eroutant la premi\`ere fois, il vous sera expliqu\'e dans les chapitres suivants.

\endinput
